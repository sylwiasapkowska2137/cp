\documentclass[zad,zawodnik,utf8]{sinol}

  \title{Chrabąszcze}
  \author{Jacek Tomasiewicz}
  \pagestyle{fancy}
  \signature{jtom???}
  \id{chr}
  \iomode{stdin}
  \konkurs{XI obóz informatyczny}
  \etap{początkująca}
  \day{?}
  \date{??.09.2015}
  \RAM{32}

\begin{document}
  \begin{tasktext}% Ten znak % jest istotny!
Dwa chrabąszcze wspinają się po łodydze. Każdy chrabąszcz co sekundę skacze o pewną  odległość w kierunku czubka rośliny. Chcielibyśmy znać najmniejszą odległość w jakiej znajdą się od siebie chrabąszcze.

Odległość możemy liczyć tylko w  momencie, gdy chrabąszcze znajdują się na łodydze (nie możemy porównywać odległości w czasie skoku). Zakładamy, że łodyga jest tak długa, że chrabąszcze nigdy nie doskoczą do czubka  rośliny.

  \section{Wejście}
	Pierwszy wiersz wejścia zawiera trzy liczby całkowite $n, a, b$ ($1 \leq n \leq  500\,000, 1 \leq a, b \leq 10^9$), oznaczające odpowiednio liczbę skoków, położenie  pierwszego i drugiego chrabąszcza. Następnych $n$ wierszy opisuje skoki chrabąszczy w  kolejnych sekundach.

Opis skoku zawiera dwie liczby całowite $x$, $y$ ($1 \leq x, y \leq 1\,000$), oznaczające odpowiednio długość skoku  pierwszego i drugiego chrabąszcza.

  \section{Wyjście}
Wyjście powinno zawierać jedną liczbę całkowitą, równą minimalnej odległości, w której  znajdą się dwa chrabąszcze.

     \makecompactexample

\medskip
\noindent
\textbf{Wyjaśnienie do przykładu:} Najbliżej położone chrabąszcze będą na pozycjach ($7,9$).

  \end{tasktext}
\end{document}