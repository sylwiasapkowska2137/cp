\documentclass[zad,zawodnik,utf8]{sinol}
  \title{Obwód}
  \id{obw}
  \signature{jtom???}
  \author{Jacek Tomasiewicz} % Autor zadania
  \pagestyle{fancy}
  \iomode{stdin}
  \konkurs{XI obóz informatyczny}
  % HINT: Pola konkurs, etap, day, date uzupelnia kierownik konkursu.
  \etap{początkująca}
  \day{1}
  \date{21.09.2015}
  \RAM{32} % HINT: To pole uzupelnia opracowujacy
 
\begin{document}
  \begin{tasktext}%
Jaś chce narysować prostokąt o minimalnym obwodzie, którego pole powierzchni wynosi dokładnie
$p$. Zakładamy, że długości boków prostokąta mogą być tylko liczbami całkowitymi.
Pomóż Jasiowi i znajdź minimalny obwód, jaki można osiągnąć.

  \section{Wejście}
Pierwszy i jedyny wiersz wejścia zawiera jedną liczbę całkowitą $p$ ($1 \leq p \leq 10^9$), oznaczającą
pole powierzchni szukanego prostokąta.

  \section{Wyjście}
Pierwszy i jedyny wiersz wyjścia powinien zawierać jedną liczbę całkowitą, równą minimalnemu
obwodowi, jaki może uzyskać Jaś.

     \makecompactexample    
	 
\medskip
\noindent
\textbf{Wyjaśnienie do przykładu:} Prostokąt o minimalnym obwodzie będzie miał wymiary $5 \times 6$.

  \end{tasktext}
\end{document}