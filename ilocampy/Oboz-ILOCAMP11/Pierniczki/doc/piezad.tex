\documentclass[zad,zawodnik,utf8]{sinol}
  \title{Pierniczki}
  \author{Jacek Tomasiewicz}
  \pagestyle{fancy}
  \signature{jtom???}
  \id{pie}
  \iomode{stdin}
  \konkurs{XI obóz informatyczny}
  \etap{średnia}
  \day{0}
  \date{20.09.2015}
  \RAM{64}
  \history{JTom, pomysł i redakcja}{2012.07.07}{1.00}

\begin{document}
  \begin{tasktext}% Ten znak % jest istotny!
Daniel ułożył wieżyczkę z $n$ pierniczków, każdy o pewnym smaku $s_i$. Każdy pierniczek ma inny smak. Daniel może wyjąć dowolnego pierniczka i wstawić go na samym szczycie, jednak smak pierniczka zmniejsza się o jeden. 

Danielowi zależy, aby wszystkie pierniczki były ułożone od najsmaczniejszego. Dokładniej $s_i$ musi być mniejsze od $s_{i+1}$, gdzie $i$ jest pierniczkiem znajdującym się na szczycie pierniczkowej wieżyczki.

Zastanawiamy się, ile minimalnie ruchów musi zrobić Daniel. Za ruch uznajemy przełożenie pierniczka na szczyt wieżyczki. Pierniczki możemy przestawiać kilka razy, jednak każde przełożenie liczymy oddzielnie.

  \section{Wejście}
Pierwszy wiersz wejścia zawiera jedną liczbę całkowitą $n$ ($1 \leq n \leq 500\,000$), oznaczającą liczbę pierniczków. Kolejny wiersz zawiera $n$ liczb całkowitych $s_1, s_2, \ldots, s_n$ ($1 \leq s_i \leq n, s_i <> s_j$, dla $i <> j$), gdzie $s_i$ oznacza smak $i$-tego pierniczka ($s_1$ jest pierniczkiem znajdującym się na szczycie wieżyczki).

W testach wartych co najmniej $50\%$ punktów zachodzi dodatkowy warunek $n \leq 10\,000$.

  \section{Wyjście}
Pierwszy i jedyny wiersz wyjścia powinien zawierać jedną liczbę całkowitą, równą minimalnej liczbie ruchów, które musi wykonać Daniel.

     \makecompactexample

	\medskip
	\noindent
	\textbf{Wyjaśnienie do przykładu:} Daniel może przestawić 4 pierniczek uzyskując ($2, 1, 2, 4$), potem 1 pierniczek uzyskując ($1, 1, 2, 4$) i 2 pierniczek uzyskując ($0, 1, 2, 4$).

  \end{tasktext}
\end{document}