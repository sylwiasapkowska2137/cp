\documentclass[zad,zawodnik,utf8]{sinol}
  \title{Lekcja matematyki}
  \signature{jtom???}
  \id{lek}
  \konkurs{XI obóz informatyczny}
  \etap{początkująca}
  \day{?}
  \date{??.09.2015}
  \RAM{32}
  \iomode{stdin}
  \pagestyle{fancy}
  \author{Jacek Tomasiewicz}
  \history{2011.08.13}{Jacek Tomasiewicz, redakcja zadania}{1.00}

\begin{document}
  \begin{tasktext}%
   Nauczyciel matematyki napisał na tablicy bardzo dużą liczbę. Zadaniem uczniów jest usunięcie z danej liczby dokładnie jednej cyfry, tak aby liczba po sklejeniu była jak największa.

Jasio nie radzi sobie z tym zadaniem. Pomóż mu, i znajdź największą liczbę, jaką można utworzyć.

  \section{Wejście}
    	Pierwszy wiersz wejścia zawiera jedną liczbę całkowitą $n$ ($10 \leq n \leq 10^{100000}$), oznaczającą liczbę, którą napisała nauczycielka na tablicy.

  \section{Wyjście}
   Pierwszy i jedyny wiersz wyjścia powinien zawierać jedną liczbę całkowitą, równą największej liczbie jaką można utworzyć po usunięciu dokładnie jednej cyfry.

  \makecompactexample


  \end{tasktext}
\end{document}