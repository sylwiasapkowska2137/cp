\documentclass[opr,utf8]{sinol}
  \signature{jtom010}            
  \title{Trójkąty}  
  \id{tro}                       
  \iomode{stdin}
  \author{Jacek Tomasiewicz}         
  \history{2011.02.17}{JTom, przygotowanie opracowania autorskiego}{1.00}
  
  
\begin{document}
  \begin{tasktext}% 
    \section{Rozwiązanie}
      
	\begin{itemize}
	\item
	$O(n^3)$ --- przeglądamy każdą trójkę patyczków i z tych, których da się 
	zbudować trójkąt wybieramy ten o minimalnym obwodzie.

	\item
	$O(n^2)$ --- całość sortujemy. Następnie dwa krótsze boki ($a, b$) wybieramy 
	na $n^2$ sposobów, a bok dłuższy ($c$) najlepiej wybrać jak najkrótszy, 
	czyli następny w kolejności za bokiem $b$.

	\item
	$O(n*log n)$ --- całość sortujemy. ($A \leq B \leq C$). $B$ i $C$ muszą 
	być koło siebie, $A$ możemy wyszukać binarnie lub jeśli znajdziemy już 
	pewne $A$ spełniające warunek $A \leq B \leq C$ i tworzące trójkąt, to 
	w następnych przypadkach opłaca nam się szukać tylko $A'< A$. 

	\end{itemize}

  \end{tasktext}
\end{document}
