\documentclass[zad,zawodnik,utf8]{sinol}
  \title{Dwie części}
  \id{dwi}
  \signature{jtom???}
  \author{Jacek Tomasiewicz} % Autor zadania
  \pagestyle{fancy}
  \iomode{stdin}
  \konkurs{XI obóz informatyczny}
  % HINT: Pola konkurs, etap, day, date uzupelnia kierownik konkursu.
  \etap{początkująca}
  \day{?}
  \date{??.09.2015}
  \RAM{64} % HINT: To pole uzupelnia opracowujacy
 
\begin{document}
  \begin{tasktext}%
Jaś znalazł w domu długą taśmę. Bez chwili namysłu napisał na taśmie pewien ciąg liczb całkowitych.
Jaś może przeciąć taśmę w pewnym miejscu, tylko wtedy gdy na obydwu połówkach
istnieje lider oraz wartości tych liderów są takie same.

Liderem nazywamy element, który występuje więcej niż $\frac{k}{2}$ razy, 
gdzie $k$ jest liczbą rozpatrywanych elementów. 
Chcielibyśmy poznać, na ile sposobów Jaś może przeciąć taśmę.

  \section{Wejście}
Pierwszy wiersz wejścia zawiera jedną liczbę całkowitą $n$ ($2 \leq n \leq 300\,000$), oznaczającą
liczbę liczb na taśmie. Drugi wiersz wejścia zawiera $n$ liczb całkowitych $a_1, a_2, \ldots, a_n$
($-10^9 \leq a_i \leq 10^9$), gdzie $a_i$ oznacza $i$-tą liczbę napisaną na taśmie.

  \section{Wyjście}
Pierwszy i jedyny wiersz wyjścia powinien zawierać jedną liczbę całkowitą, równą liczbie
sposobów, na które Jaś może przeciąć taśmę.

     \makecompactexample    

  \end{tasktext}
\end{document}