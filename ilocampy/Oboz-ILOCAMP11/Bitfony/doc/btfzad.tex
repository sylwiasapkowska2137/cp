\documentclass[zad,zawodnik,utf8]{sinol}
  \title{Bitfony}
  \author{Jacek Tomasiewicz}
  \pagestyle{fancy}
  \signature{jtom???}
  \id{btf}
  \iomode{stdin}
  \konkurs{XI obóz informatyczny}
  \etap{początkująca}
  \day{?}
  \date{??.09.2015}
  \RAM{32}

\begin{document}
  \begin{tasktext}% Ten znak % jest istotny!
W Bajtocji istnieją bitfony, które umożliwiają połączenie głosowe (i tekstowe) z tylko jedną, przypisaną do bitfonu osobą.

W firmie pracuje $n$ osób, każda z nich dysponuje jednym bitfonem. Jeśli pewna osoba $a$ chce wysłać wiadomość do osoby $b$, to wysyła poprzez swój bitfon wiadomość (jej treść oraz nazwę adresata). Wiadomość porusza się pomiędzy bitfonami do momentu, aż trafi do $b$ lub wróci do $a$. 

Jesteś programistą w firmie i chcesz wysłać wiadomość do pewnej osoby. Zastanawiasz się, czy wiadomość dojdzie do wybranej osoby, jeśli tak, po jakim czasie. 

  \section{Wejście}
	
Pierszy wiersz wejścia zawiera jedną liczbę całkowitą $n$ ($1 \leq n \leq 10^6$), oznaczającą liczbę osób w firme. 

Drugi wiersz wejścia zawiera $n$ liczb całkowitych $x_1, x_2, \ldots, x_n$ ($1 \leq x_i \leq n$), gdzie $x_i$ oznacza numer osoby, z którą może połączyć się $i$-ta osoba.

Trzeci wiersz wejścia zawiera dwie liczby całkowite $a, b$ ($1 \leq a, b \leq n, a \neq b$), oznaczające, że osoba $a$, chce wysłać wiadomośc do osoby $b$.

  \section{Wyjście}
	Pierwszy i jedyny wiersz wyjścia powinien zawierać jedną liczbę całkowitą, równą czasowi, po którym dojdzie wiadomość lub wartość $-1$, jeśli wiadomość nigdy nie dojdzie.

     \exampleimg{bitofon.eps}
     \makestandardexample

\medskip
\noindent
\textbf {Wyjaśnienie do przykładu:} Wiadomość dojdzie i będzie szła następująco $1 \rightarrow 2 \rightarrow 3$.

  \end{tasktext}
\end{document}