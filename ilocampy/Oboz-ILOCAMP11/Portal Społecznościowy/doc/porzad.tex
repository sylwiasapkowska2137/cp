\documentclass[zad,zawodnik,utf8]{sinol}
  \title{Portal Społecznościowy}
  \author{}
  \pagestyle{fancy}
  \signature{jtom???}
  \id{por}
  \iomode{stdin}
  \konkurs{XI obóz informatyczny}
  \etap{olimpijska}
  \day{3}
  \date{23.09.2015}
  \RAM{128}

\begin{document}
  \begin{tasktext}% Ten znak % jest istotny!
W obecnych czasach popularność na serwisach społecznościowych jest dla niektórych ważniejsza niż wszystko inne. Blogerzy, modele, czy aktorzy walczą o bycie obserwowanym. Każdy chce być najbardziej popularny. Nasze zadanie jest właśnie o tym.

Mamy $N$ kont na pewnym portalu społecznościowym. Informacje o kontach są podane w postaci: ,,konto $X$ obserwuje konto $Y$''. Chcemy teraz dowiedzieć się, czy w niepustej sumie przedziałów znajduje się konto gwiazdy. W danym podzbiorze konto gwiazdy to takie, które jest obserwowane przez wszystkie pozostałe konta (z tego podzbioru), podczas gdy samo nie obserwuje nikogo z pozostałych (z tego podzbioru). Zakładamy, że żadne konto nie może obserwować samego siebie.

  \section{Wejście}
W pierwszej linii wejścia znajduje się jedna liczba całkowita $N$ ($1 \leq N \leq 1\,000$), oznaczająca liczbę kont na portalu społecznościowym.

W bloku kolejnych $N$ linii wejścia znajduje się po $N$ liczb całkowitych $o_{i,j}$ ($0 \leq o_{i,j} \leq 1, 1 \leq i, j \leq N$) stanowiących opis, kto kogo obserwuje. ($i$ oznacza numer wiersza w bloku, natomiast $j$ oznacza numer kolumny, $o_{i,j} = 1$ oznacza, że konto $j$-te jest obserwowane przez konto $i$-te).

W kolejnej linii wejścia znajduje się jedna liczba całkowita $M$ ($1 \leq M$), oznaczająca liczbę zapytań o podzbiory kont portalu.

Następnie na wejściu znajduje się opis $M$ zapytań. Każde z nich opisane jest w następujący sposób: w pierwszej linii zapytania znajduje się jedna liczba całkowita $L$ ($1 \leq L$), oznaczająca liczbę przedziałów, których suma tworzy rozważany podzbiór. W każdej z kolejnych $L$ linii wejścia znajduje się opis tych przedziałów w postaci dwóch liczb całkowitych $a_i, b_i$ ($1$~$\leq$~$a_i$~$\leq$~$b_i$~$\leq$~$N$), oznaczających odpowiednio numer początkowego i końcowego konta z $i$-tego przedziału składającego się na dane zapytanie. Łączna liczba przedziałów ze wszystkich zapytań nie przekracza $1\,000\,000$.

  \section{Wyjście}
W $M$ liniach wyjścia powinny znaleźć się odpowiedzi dla poszczególnych podzbiorów. Odpowiedzią jest numer konta gwiazdy (z rozpatrywanego podzbioru) lub wartość $-1$ jeśli takie konto nie istnieje.

     \makecompactexample

\medskip
  \noindent
  \textbf{Wyjaśnienie do przykładu:} W pierwszym podzbiorze konto numer 6 jest obserwowane przez konta 1, 2, 3 i 5, zaś samo nie obserwuje żadnego innego konta. W drugim podzbiorze nie istnieje konto spełniające warunki bycia gwiazdą.

  \end{tasktext}
\end{document}