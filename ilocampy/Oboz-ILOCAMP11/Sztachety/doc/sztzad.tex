\documentclass[zad,zawodnik,utf8]{sinol}

\title{Sztachety}
\id{szt}
\author{Michał Majewski} % Autor zadania
\pagestyle{fancy}
\iomode{stdin}
\konkurs{XI obóz informatyczny}
\etap{olimpijska}
\day{?}
\date{??.09.2015}
\RAM{32}
 
\begin{document}
\begin{tasktext}%
Pewien znany fotograf podróżuje po świecie szukając nietypowych inspiracji, a następnie zachwyca miłośników fotografii swoimi dziełami. W tym roku wybrał się na podróż w poszukiwaniu ciekawych płotów. Po zebraniu zdjęć doszedł do wniosku, że najbardziej podobają mu się zdjęcia płotów z kolorowymi sztachetami. Ale czy tak kolorowe zdjęcia zostaną dobrze odebrane? Po dłuższych rozważaniach postanowił przyciąć niektóre fotografie, aby zawierały tylko fragmenty płotów z oryginalnych zdjęć.

W wyciętym fragmencie powinno się znaleźć dokładnie $k$ sztachet o unikalnym kolorze (których kolor występuje dokładnie raz w tym fragmencie).

Zadanie okazało się czasochłonne i trudne, a zdjęć do obróbki jest bardzo dużo. Spróbuj pomóc naszemu artyście znajdując odpowiednie fragmenty jego zdjęć.

\section{Wejście}
W pierwszym wierszu wejścia znajdują się dwie liczby całkowite $n$, $k$ ($1 \leq n, k \leq 1\,000\,000$) oznaczające odpowiednio liczbę sztachet w płocie oraz liczbę z treści zadania.

W drugim wierszu wejścia znajduje się $n$ liczb całkowitych $x_1, x_2, \ldots, x_n$ ($1 \leq x_i \leq 1\,000\,000$), gdzie $x_i$ oznacza kolor $i$-tej sztachety.

\section{Wyjście}
Na wyjściu powinna znaleźć się jedna liczba całkowita oznaczająca długość najdłuższego (spójnego) fragmentu płotu, który spełnia warunki z zadania. Jeśli nie da się wybrać takiego fragmentu wypisz $-1$.

\makecompactexample

\end{tasktext}
\end{document}