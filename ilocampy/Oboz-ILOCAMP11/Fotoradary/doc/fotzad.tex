\documentclass[zad,zawodnik,utf8]{sinol}
  \title{Fotoradary}
  \author{Mateusz Puczel}
  \pagestyle{fancy}
  \signature{mp???}
  \id{fot}
  \iomode{stdin}
  \konkurs{XI obóz informatyczny}
  \etap{olimpijska}
  \day{1}
  \date{21.09.2015}
  \RAM{64}

\begin{document}
  \begin{tasktext}% Ten znak % jest istotny!
Od pewnego czasu w Bajtocji zauważono coraz więcej nielegalnych wyścigów nocnych,
które stały się poważnym zagrożeniem dla bezpieczeństwa mieszkańców.
Z tego powodu szef Bajtockiej Policji postanowił zamontować na ulicach fotoradary w celu zebrania niezbędnych dowodów.
  
W mieście jest $n$ skrzyżowań połączonych $n - 1$ ulicami. Każda ulica ma długość $1$.
Wyścig może odbyć się między dowolnymi dwoma różnymi skrzyżowaniami na ulicach łączących te skrzyżowania.
Z powodu ograniczonego budżetu Bajtocka Policja może zamontować maksymalnie $k$ fotoradarów na tych $n - 1$ ulicach. 
Fotoradary powinny być zamontowane w taki sposób, aby długość najdłuższej trasy niepokrytej żadnym fotoradarem była jak najkrótsza.

  \section{Wejście}
W pierwszym wierszu wejścia znajdują się dwie liczby całkowite $n, k$ ($1 \leq k < n \leq 10^6$), 
oznaczające kolejno liczbę skrzyżowań w Bajtocji oraz maksymalną liczbę fotoradarów, które można zamontować.

W każdym z kolejnych $n - 1$ wierszy znajdują się dwie liczby całkowite $a, b$ ($1 \leq a, b \leq n$, $a \neq b$),
oznaczające, że istnieje ulica pomiędzy skrzyżowaniami $a$ i $b$.

  \section{Wyjście}
W pierwszym i jedynym wierszu wyjścia powinna znaleźć się jedna liczba całkowita, oznaczająca długość najdłuższej trasy, na której
może odbywać się wyścig i na żadnej z ulic tej trasy nie stoi fotoradar.
    
    \makecompactexample


  \end{tasktext}
\end{document}