\documentclass[zad,zawodnik,utf8]{sinol}
  \title{Łańcuch}
  \signature{jtom031}
  \id{lan}
  \konkurs{XI obóz informatyczny}
  \etap{początkująca}
  \day{?}
  \date{??.09.2015}
  \RAM{32}
  \iomode{stdin}
  \pagestyle{fancy}
  \author{Jacek Tomasiewicz}
  \history{2011.08.07}{Jacek Tomasiewicz, pomysł i redakcja zadania}{1.00}

\begin{document}
  \begin{tasktext}% Ten znak %~jest istotny!
    Bajtek znalazł długi łańcuch składający się z~$n$ ogniw.
    Uznał jednak, że nie jest mu potrzebny jeden łańcuch,
    lecz dokładnie trzy o~dowolnych długościach (oczywiście każdy składający się z~co najmniej jednego ogniwa).

Bajtek może niszczyć dowolne ogniwa, dzięki czemu może dzielić łańcuch na części.
    Każde ogniwo posiada pewną wytrzymałość $w_i$, którego zniszczenie wymaga od Bajtka $w_i$ sekund.

    Chcielibyśmy znać minimalny czas, jaki Bajtek musi poświęcić, aby podzielić łańcuch na trzy części.

    \section{Wejście}
      Pierwszy wiersz standardowego wejścia zawiera jedną liczbę całkowitą $n$ ($5~\leq~n~\leq~300\,000$),
      oznaczającą liczbę ogniw, z~których składa się łańcuch.
      Kolejny wiersz zawiera $n$ liczb całkowitych $w_1, w_2, \ldots, w_n$ ($1 \leq w_i \leq 10^9$),
      gdzie $w_i$ oznacza wytrzymałość $i$-tego ogniwa.

    \section{Wyjście}
      Pierwszy i~jedyny wiersz wyjścia powinien zawierać jedną liczbę całkowitą,
      równą minimalnej liczbie sekund, potrzebnych Bajtkowi na podział łańcucha.

    \makecompactexample

  \medskip
  \noindent
  \textbf{Wyjaśnienie do przykładu:} 
  Bajtek zniszczy drugie i~piąte ogniwo.

  \end{tasktext}
\end{document}