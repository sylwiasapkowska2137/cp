\documentclass[zad,zawodnik,utf8]{sinol}

\title{Liczby Kozlova}
\id{koz}
\author{Maciej Hołubowicz} % Autor zadania
\pagestyle{fancy}
\iomode{stdin}
\konkurs{XI obóz informatyczny}
\etap{olimpijska}
\day{1}
\date{21.09.2015}
\RAM{64}
 
\begin{document}
\begin{tasktext}%

Kozlov to sławny rosyjski matematyk, ma na koncie wiele istotnych twierdzeń i dowodów.
Liczby Kozlova to takie liczby naturalne, które nie są podzielne przez żaden kwadrat liczby naturalnej większej od jedynki.
Twoim zadaniem jest znalezienie $k$-tej co do wielkości liczby Kozlova.


  \section{Wejście}
W pierwszym i jedynym wierszu wejścia znajduje się liczba całkowita $k$ ($1 \leq k \leq 10^9 $).

  \section{Wyjście}
Na wyjściu powinna się znaleźć jedna liczba całkowita równa $k$-tej co do wielkości liczbie Kozlova.

\makecompactexample

\end{tasktext}
\end{document}
