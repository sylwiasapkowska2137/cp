\documentclass[zad, zawodnik, utf8]{sinol}

\usepackage{graphicx}
\usepackage{amsthm}
\newcounter{def}
\newcounter{tw}
\newcounter{lem}
\newcounter{wn}
\newcounter{obs}

\usepackage{listings}
\lstset{
  basicstyle=\small,
  keywordstyle=\bfseries,
  numbers=left,
  language=pascal,
  xleftmargin=1.2em,
  frame=TBLR,
  mathescape=true,
  numberstyle=\footnotesize,
 }

\newcommand{\definicja}[1]{\vskip 0.2cm \noindent {\bf Definicja.\thedef.} \stepcounter{def} \emph{#1} \vskip 0.3cm}
\newcommand{\twierdzenie}[1]{\vskip 0.2cm \noindent {\bf Twierdzenie.\thetw.} \stepcounter{tw} \emph{#1} \vskip 0.3cm}
\newcommand{\lemat}[1]{\noindent {\vskip 0.2cm \bf Lemat.\thelem.} \stepcounter{lem} \emph{#1} \vskip 0.3cm}
\newcommand{\wniosek}[1]{\noindent {\vskip 0.2cm \bf Wniosek.\thewn.} \stepcounter{wn} \emph{#1} \vskip 0.3cm}
\newcommand{\obserwacja}[1]{\noindent {\vskip 0.2cm \bf Obserwacja.\theobs.} \stepcounter{obs} \emph{#1} \vskip 0.3cm}
\newcommand{\intuicja}[1]{\noindent {\vskip 0.2cm \bf Intuicja.} #1 \vskip 0.3cm}
\newcommand{\dowod}[1]{\begin{proof} #1 \end{proof}}
\begin{document}
 \signature{jtom???}
  \id{zer}
  \pagestyle{fancy}
  \title{Zerowe sumy}
  \author{Jacek Tomasiewicz}
  \iomode{stdin}
  \konkurs{XI obóz informatyczny}
  \etap{początkująca}
  \day{3}
  \date{19.11.2011}
  \RAM{32}
  \history{2011.08.12}{Mateusz Litwin, przygotowanie rozwiązania}{1.00}



\begin{tasktext}%

\setcounter{def}{1}
\setcounter{tw}{1}
\setcounter{wn}{1}
\setcounter{obs}{1}

\section{Rozwiązanie wolne $O(n^2)$}

Dla każdej pary liczb sprawdzamy, czy sumuje się do 0. Wybieramy najbardziej odległą. Wszystkich par liczb jest $n^2$, stąd złożoność takiego rozwiązania to $O(n^2)$. Za takie rozwiązanie można było uzyskać $50\%$ punktów.

\section{Rozwiązanie wzorcowe $O(n)$}

W poprzednim rozwiązaniu nie wykorzystujemy, że wartości liczb są z niedużego zakresu. Możemy dla każdej wartości $x$ zapisać informację o najbardziej lewym oraz prawym elemencie. Łatwo zauważyć, że elementami sumującymi się do 0, będą wartości $-x$ i $x$. Dla każdego z nich w czasie stałym wyznaczamy dwa najbardziej odległe (wykorzystując wcześniej wygenerowane lewe i prawe elementy).

Złożoność takiego rozwiązania to $O(n)$.

\end{tasktext}
\end{document}