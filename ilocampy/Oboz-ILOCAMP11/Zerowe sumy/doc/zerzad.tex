\documentclass[zad,zawodnik,utf8]{sinol}
  \title{Zerowe sumy}
  \author{Jacek Tomasiewicz}
  \pagestyle{fancy}
  \signature{jtom???}
  \id{zer}
  \iomode{stdin}
  \konkurs{XI obóz informatyczny}
  \etap{początkująca}
  \day{1}
  \date{21.09.2015}
  \RAM{32}
  \history{2011.11.19}{JTom, pomysł i~redakcja}{1.00}

\begin{document}
  \begin{tasktext}% Ten znak % jest istotny!
Mamy dane $n$ liczb całkowitych w tablicy. Naszym zadaniem jest znalezienie dwóch liczb (niekoniecznie różnych), których suma jest równa 0. Dodatkowo chcielibyśmy, aby znalezione liczby były jak najbardziej oddalone od siebie. 

  \section{Wejście}
	
Pierwszy wiersz wejścia zawiera jedną liczbę całkowitą $n$ ($1 \leq n \leq 1\,000\,000$), oznaczającą ilość liczb. Kolejny wiersz zawiera $n$ liczb całkowitych $l_1, l_2, \ldots, l_n$ ($-10^6 \leq l_i \leq 10^6$), gdzie $l_i$ oznacza wartość $i$-tej liczby w tablicy.

W testach wartych $50\%$ punktów zachodzi dodatkowy warunek $n \leq 1\,000$.

  \section{Wyjście}
	Pierwszy i jedyny wiersz wyjścia powinien zawierać jedną liczbę całkowitą, równą największej odległości pomiędzy parą liczb, sumujących się do zera lub wartość $-1$, gdy takiej pary nie ma.

     \makecompactexample

\medskip
\noindent
\textbf {Wyjaśnienie do przykładu:} Pary sumujące się do zera to: ($-2, 2$), ($3, -3$), ($0, 0$). Najbardziej odległa jest para ($-2, 2$).

  \end{tasktext}
\end{document}