\documentclass[zad,zawodnik,utf8]{sinol}

\title{Dzielny Przemo}
\id{dzi}
\author{JT} % Autor zadania
\pagestyle{fancy}
\iomode{stdin}
\konkurs{XI obóz informatyczny}
\etap{olimpijska}
\day{4}
\date{24.09.2015}
\RAM{128}
 
\begin{document}
\begin{tasktext}%
Mamy dany ciąg $n$ liczb całkowitych $x_1, x_2, \ldots, x_n$.
Przemo chce znaleźć najdłuższy spójny fragment, tak aby można było go podzielić na dwie niepuste części,
lewą i prawą. Jedynym warunkiem jest to, aby w lewej części była tylko jedna wartość maksymalna, a w prawej tylko jedna wartość minimalna.

  \section{Wejście}
W pierwszym wierszu wejścia znajduje się jedna liczba całkowita $n$ ($2 \leq n \leq 10^6$),
oznaczająca liczbę elementów ciągu.
W kolejnym wierszu wejścia znajduje się $n$ liczb całkowitych $x_1, x_2, \ldots, x_n$ ($-10^9 \leq x_i \leq -10^9$), gdzie $x_i$ oznacza $i$-tą liczbę w ciągu.

  \section{Wyjście}
Na wyjściu powinna znaleźć się jedna liczba całkowita, równa maksymalnej długości szukanego fragmentu.
Jeżeli taki fragment nie występuje, to wynikiem powinna być wartość \texttt{0}.

\makecompactexample    

\end{tasktext}
\end{document}