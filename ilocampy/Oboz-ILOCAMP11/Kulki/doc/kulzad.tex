\documentclass[zad,zawodnik,utf8]{sinol}

\title{Kulki}
\id{kul}
\author{} % Autor zadania
\pagestyle{fancy}
\iomode{stdin}
\konkurs{XI obóz informatyczny}
\etap{średnia}
\day{?}
\date{??.09.2015}
\RAM{32}
 
\begin{document}
\begin{tasktext}%
Młody Maciuś pojechał w odwiedziny do dziadków na wieś. Tematy dorosłych nie za bardzo go interesowały, więc zamiast siedzieć z innymi postanowił poszukać sobie ciekawszego zajęcia. W jednym z pokoi wypatrzył rząd czerwonych i białych kulek. Postanowił zrobić porządek i tak je poprzestawiać, aby wszystkie czerwone kulki leżały jedna obok drugiej. Kulki okazały się bardzo ciężkie, więc chciałby tego dokonać najmniejszą liczbą ruchów polegających na zamianie dwóch sąsiednich kulek miejscami.

  \section{Wejście}
W pierwszej linii wejścia znajduje się jedna liczba całkowita $n$ ($1 \leq n \leq 1\,000\,000$) oznaczająca liczbę kulek.
W drugiej linii wejścia znajduje się słowo złożone z $n$ liter, tak że litera $i$ opisuje kolor kuli $i$. Litera \texttt{B} oznacza kolor biały, a \texttt{C} - czerwony. 

  \section{Wyjście}
Na wyjściu powinna znaleźć się jedna liczba całkowita, oznaczająca ile minimalnie ruchów musi wykonać Maciuś, aby wszystkie czerwone kulki znalazły się obok siebie.

\makecompactexample
  \medskip
  \noindent
  \textbf{Wyjaśnienie do przykładu:} 
  Zamieniamy miejscami kolejno kulki na pozycjach $4$ i $5$; $3$ i $4$; $8$ i $9$; $9$ i $10$.

\end{tasktext}
\end{document}