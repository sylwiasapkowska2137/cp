\documentclass[zad,zawodnik,utf8]{sinol}
  \title{Chodnik}
  \author{Jacek Tomasiewicz}
  \pagestyle{fancy}
  \signature{jtom???}
  \konkurs{XI obóz informatyczny}
  \id{cho}
  \iomode{stdin}
  \etap{początkująca}
  \day{?}
  \date{??.09.2015}
  \RAM{32}

\begin{document}
  \begin{tasktext}% Ten znak % jest istotny!
Edek napisał kredą na chodniku wszystkie liczby od 1 do $n$ w losowej kolejności. Następnie poszedł do sklepu. Po powrocie zauważył, że brakuje jednej liczby. Pomóż Edkowi i powiedz, której liczby brakuje!

  \section{Wejście}
Pierwszy wiersz wejścia zawiera jedną liczbę całkowitą $n$ ($1 \leq n \leq 500\,000$), oznaczającą ilość liczb, które wypisał Edek.

Kolejny wiersz zawiera $n-1$ liczb całkowitych $l_1, l_2, \ldots, l_{n-1}$ ($1 \leq l_i \leq n)$, gdzie $l_i$ oznacza $i$-tą liczbę na chodniku (po powrocie Edka ze sklepu).

  \section{Wyjście}
Pierwszy i jedyny wiersz wyjścia powinien zawierać jedną liczbę całkowitą, równą wartości liczby, której brakuje na chodniku.

     \makecompactexample


  \end{tasktext}
\end{document}