\documentclass[zad, zawodnik, utf8]{sinol}

\usepackage{graphicx}
\usepackage{amsthm}
\newcounter{def}
\newcounter{tw}
\newcounter{lem}
\newcounter{wn}
\newcounter{obs}

\usepackage{listings}
\lstset{
  basicstyle=\small,
  keywordstyle=\bfseries,
  numbers=left,
  language=pascal,
  xleftmargin=1.2em,
  frame=TBLR,
  mathescape=true,
  numberstyle=\footnotesize,
 }

\newcommand{\definicja}[1]{\vskip 0.2cm \noindent {\bf Definicja.\thedef.} \stepcounter{def} \emph{#1} \vskip 0.3cm}
\newcommand{\twierdzenie}[1]{\vskip 0.2cm \noindent {\bf Twierdzenie.\thetw.} \stepcounter{tw} \emph{#1} \vskip 0.3cm}
\newcommand{\lemat}[1]{\noindent {\vskip 0.2cm \bf Lemat.\thelem.} \stepcounter{lem} \emph{#1} \vskip 0.3cm}
\newcommand{\wniosek}[1]{\noindent {\vskip 0.2cm \bf Wniosek.\thewn.} \stepcounter{wn} \emph{#1} \vskip 0.3cm}
\newcommand{\obserwacja}[1]{\noindent {\vskip 0.2cm \bf Obserwacja.\theobs.} \stepcounter{obs} \emph{#1} \vskip 0.3cm}
\newcommand{\intuicja}[1]{\noindent {\vskip 0.2cm \bf Intuicja.} #1 \vskip 0.3cm}
\newcommand{\dowod}[1]{\begin{proof} #1 \end{proof}}
\begin{document}
 \signature{jtom???}
  \id{spr}
  \pagestyle{fancy}
  \title{Kule sprężyste}
  \author{Jacek Tomasiewicz}
  \iomode{stdin}
  \konkurs{XI obóz informatyczny}
  \etap{średnia}
  \day{}
  \date{sierpień 2012}
  \RAM{32}
  \history{2011.12.29}{Jacek Tomasiewicz, przygotowanie rozwiązania}{1.00}


\begin{tasktext}%

\setcounter{def}{1}
\setcounter{tw}{1}
\setcounter{wn}{1}
\setcounter{obs}{1}

\section{Rozwiązanie wzorcowe $O(n)$}

Moment, w którym nie będzie juz żadnych zderzeń będzie taki, że pewna początkowa liczba kul porusza się tylko w lewo, a reszta prawo, czyli \texttt{LL..LP..PP}.

Jak wyliczać liczbę zderzeń i końcowy kierunek kul. Możemy poruszać się od lewej do prawej. W przypadku natrafienia na:
\begin{itemize}
\item kulę \texttt{P} -- dodajemy ją do kolejki (w kolejce będą tylko kule poruszające się w prawo),
\item kulę \texttt{L} -- pierwsza z kolejki kula zmieni swój kierunek (poprzez ciąg zderzeń kolejnych kul z kolejki). Kula ta zmieni swój kierunek na dobre, więc możemy ją usunąć. Rozpatrywana kula \texttt{L} też zmieni swój kierunek, więc musimy ją dodać do kolejki. Liczbę zderzeń zwiększamy o liczbę elementów w kolejce.
\end{itemize}

\end{tasktext}
\end{document}