\documentclass[zad,zawodnik,utf8]{sinol}
  \title{Pary}
  \signature{ajas}
  \id{par}
  \etap{olimpijska}
  \day{3}
  \konkurs{XI obóz informatyczny}
  \date{23.09.2015}
  \RAM{128}
  \iomode{stdin}
  \pagestyle{fancy}
  \author{Adrian Jaskółka}
  \history{2011.11.18}{Adrian Jaskółka-wszystko}{1.00}

\begin{document}
  \begin{tasktext}%
Pani BitoKazimiera jest nauczycielką informatyki w jednym z Bajtockich gimnazjów. W jej ulubionej klasie ma $n$ uczniów. Każdy z uczniów ma pewnego najlepszego przyjaciela w klasie.

	Pani BitoKazimiera chciała zabrać swoją ulubioną klasę na wycieczkę do Muzeum BitoKopernika, aby uczniowie poznali dokoniania ich rodaka 
	(np. odkrycie, że to BitoSłońce obraca się wokół BitoZiemi, a nie na odwrót). W tym celu postanowiła połączyć uczniów w jak największą liczbę par, 
	tak aby w każdej parze znajdował się uczeń oraz jego najlepszy przyjaciel. Jeżeli miała wiele możliwości podziału, to chciałaby to zrobić tak,
    aby w jak największej liczbie par była dziewczynka oraz chłopczyk.

	Twoim zadaniem jest policzenie ile można najwięcej utworzyć par oraz w jak największej liczbie par może być po dziewczynce i chłopczyku.
	Oczywiście każdy uczeń może należeć tylko do jednej pary.
	
	Możesz założyć, że żadna osoba nie jest swoim najlepszym przyjacielem.

  \section{Wejście}
  	
	W pierwszym wierszu wejścia znajduje się jedna liczba całkowita $n$ oznaczająca liczbę uczniów ($2~\leq~n~\leq~500\,000$). W kolejnych $n$ wierszach znajdują się po dwie liczby oznaczające kolejno
	najlepszego przyjaciela $i$-tej osoby oraz płeć $i$-tej osoby ($1$ dla dziewczynki oraz $2$ dla chłopczyka).

  \section{Wyjście}
  	
	Wypisz dwie liczby oznaczające największą możliwą liczbę par jakie możemy utworzyć, oraz ile najwięcej spośród tych par może być parą z dziewczynką i chłopcem.

  \makecompactexample

\medskip
\noindent
\textbf{Wyjaśnienie do przykładu:}
Łączymy w pary ucznia 2 z uczennicą 3 i uczennicę 4 z uczniem 5.

  \end{tasktext}
\end{document}
