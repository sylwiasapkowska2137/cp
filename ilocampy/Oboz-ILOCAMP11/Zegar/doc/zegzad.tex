\documentclass[zad,zawodnik,utf8]{sinol}

\title{Zegar}
\id{zeg}
\author{Jacek Tomasiewicz} % Autor zadania
\pagestyle{fancy}
\iomode{stdin}
\konkurs{XI obóz informatyczny}
\etap{początkująca}
\day{?}
\date{??.09.2015}
\RAM{32}
 
\begin{document}
\begin{tasktext}%
Jaś ma dostać od rodziców prezent, bardzo się niecierpliwi i nie może doczekać tej chwili. Ciągle pyta mamę: ,,Czy to już?''. Mama ma tego serdecznie dość, więc postanowiła zdradzić Jasiowi dokładnie o której godzinie spotka go wyczekiwane szczęście. Jaś teraz zastanawia się, ile jeszcze minimalnie czasu musi upłynąć, aby na zegarze wybiła dokładnie ta godzina. Ty zaś mu w tym pomożesz.

  \section{Wejście}
W pierwszej linii wejścia znajduje się obecny czas w formacie \texttt{HH}:\texttt{MM} (gdzie \texttt{HH} oznacza godziny a \texttt{MM} minuty, $0 \leq \texttt{HH} \leq 23, 0 \leq \texttt{MM}
\leq 59$). W drugiej linii wejścia znajduje się czas, w którym Jaś spodziewa się otrzymać od rodziców prezent, również w formacie \texttt{HH}:\texttt{MM}, opisanym wcześniej.

Po godzinie \texttt{23:59} na zegarze pojawia się godzina \texttt{00:00}. Czas obecny i czas, w którym Jaś spodziewa się otrzymać od rodziców prezent, nie muszą należeć do tego samego dnia. 

  \section{Wyjście}
W pierwszej i jedynej linii wyjścia powinna znaleźć się liczba, oznaczająca ile minut musi upłynąć od pierwszego czasu, aby zegar wskazał czas drugi.

\makecompactexample

\end{tasktext}
\end{document}