\documentclass[zad,zawodnik,utf8]{sinol}

\title{Unikaty}
\id{uni}
\author{} % Autor zadania
\pagestyle{fancy}
\iomode{stdin}
\konkurs{XI obóz informatyczny}
\etap{średnia}
\day{1}
\date{21.09.2015}
\RAM{32}
 
\begin{document}
\begin{tasktext}%
Piotruś jest bardzo bystrym uczniem, ale potrafi mocno narozrabiać. Ostatnio za karę dostał bardzo długą taśmę, na której znajdował się ciąg liter. Jego zadanie polegało na znalezieniu fragmentu, w którym jest dokładnie $k$ różnych liter. Pani przedszkolanka dobrze zna zdolności Piotrusia i utrudniła zadanie, żeby nie poszło mu zbyt szybko. Ustaliła, że ma to być możliwie najdłuższy fragment.

\section{Wejście}
W pierwszym wierszu wejścia znajdują się dwie liczby $n$, $k$ ($1 \leq k \leq 26$, $1 \leq n \leq 1\,000\,000$) oznaczające odpowiednio liczbę liter na taśmie oraz liczbę z treści zadania.

W drugim wierszu wejścia znajduje się $n$ małych liter alfabetu angielskiego reprezentujących ciąg liter znajdujący się na taśmie.

\section{Wyjście}
Na wyjściu powinna znaleźć się jedna liczba całkowita oznaczają długość szukanego fragmentu.
Możesz założyć, że zawsze istnieje taki fragment.

\makecompactexample
  \medskip
  \noindent
  \textbf{Wyjaśnienie do przykładu:} 
  Najdłuższy fragment zawierający dokładnie 2 różne litery to: \texttt{uucu}.

\end{tasktext}
\end{document}