\documentclass[zad,zawodnik,utf8]{sinol}

\title{Prezenty}
\id{bud}
\author{Maciej Hołubowicz} % Autor zadania
\pagestyle{fancy}
\iomode{stdin}
\konkurs{XI obóz informatyczny}
\etap{olimpijska}
\day{4}
\date{24.09.2015}
\RAM{64}
 
\begin{document}
\begin{tasktext}%
Franek ma wielu przyjaciół. W jego mieście jest $S$ sklepów, w których można kupić $P$ różnych typów przedmiotów.
Franek bardzo lubi przeglądać oferty kupna różnych towarów. 
Dana oferta mówi o tym, że w jakimś ze sklepów można kupić $X$ przedmiotów danego typu w cenie $C$.
Franek niektórych ze swoich przyjaciół lubi bardziej, a niektórych mniej. 
W zależności od tego jak bardzo kogo lubi chciałby kupić mu daną liczbę prezentów na urodziny.
Franek nie jest jednak głupi i chciałby wybrać jak najtańsze przedmioty.
Franek nie chce też danej osobie kupić za dużo przedmiotów jednego typu, bo wtedy prezent będzie słaby.
Dokładniej, Franek może kupić co najwyżej $K[i]$ przedmiotów typu $i$.
Franek nie chce również kupić za dużo przedmiotów w jednym sklepie, żeby przyjaciele nie myśleli, że był zbyt leniwy i nie chciał dla nich odwiedzić wielu sklepów.
Dokładniej, w $i$-tym sklepie nasz bohater może kupić nie więcej niż $M[i]$ przedmiotów. 


  \section{Wejście}
W pierwszym wierszu wejścia znajdują się trzy liczby całkowite $m, S, P$ ($1 \leq m \leq 10^3, 1 \leq S, P \leq 100$), oznaczające kolejno liczbę ofert, liczbę sklepów oraz liczbę typów przedmiotów.

W drugim wierszu wejścia jest $S$ liczb całkowitych opisujących tablicę $M$. $i$-ta liczba mówi o tym, jak dużo przedmiotów może kupić Franek w $i$-tym sklepie. Liczby nie przekraczają $200$ i są dodatnie.

W trzecim wierszu wejścia jest $P$ liczb całkowitych opisujących tablicę $K$. $i$-ta liczba mówi o tym jak dużo egzemplarzy $i$-tego przedmiotu może kupić Franek. Te liczby również nie przekraczają $200$ i są dodatnie.

W kolejnych $m$ wierszach znajdują się cztery liczby całkowite $a, b, X$ oraz $C$, będące ofertą sprzedaży przedmiotu $a$ w sklepie $b$ w ilości $X$ ($ 1 \leq X \leq 10^4$) o koszcie $C$ ($1 \leq C \leq 10^4$).

W kolejnym wierszu znajduje się liczba całkowita $q$ ($1 \leq q \leq 10^3$) będącą liczbą zapytań o kupienie danej liczby przedmiotów spełniających wymagania Franka.

W $q$ kolejnych wierszach znajduje się po jednej liczbie całkowitej $Y$ ($1 \leq Y \leq 10^9$), oznaczającej że Franek chce kupić $Y$ przedmiotów według opisanych zasad.

  \section{Wyjście}
Na wyjściu powinno się znaleźć $q$ wierszy, w każdym wierszu jedna liczba całkowita, oznaczająca minimalny koszt kupienia przedmiotów dla każdego z zapytań. Jeśli nie da się kupić tylu przedmiotów, wypisz $-1$.

\makecompactexample

\end{tasktext}
\end{document}
