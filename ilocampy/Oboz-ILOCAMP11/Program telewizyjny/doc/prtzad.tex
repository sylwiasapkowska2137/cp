\documentclass[zad,zawodnik,utf8]{sinol}
  \title{Program telewizyjny}
  \signature{jtom???}
  \id{prt}
  \konkurs{XI obóz informatyczny}
  \etap{średnia}
  \day{?}
  \date{??.09.2015}
  \RAM{32}
  \iomode{stdin}
  \pagestyle{fancy}
  \author{Jacek Tomasiewicz}
  \history{2012.02.21}{Jacek Tomasiewicz, pomysł i redakcja zadania}{1.00}

\begin{document}
  \begin{tasktext}%
    W \textit{BajtTV} trwają obecnie prace nad nową ramówką. Redaktorzy otrzymali
    $n$ propozycji programów. Każdy jest opisany przez jego ramy czasowe
    (czas rozpoczęcia emisji programu $b_i$ oraz czas jej zakończenia $e_i$)
    i~stopień atrakcyjności dla telewidzów $a_i$. Szefostwo chce, aby w
    ramówce znajdowało się co najmniej $k$ kanałów. Jednocześnie atrakcyjność
    najsłabszego emitowanego programu, ma być jak największa. Programy nie mogą się pokrywać, w szczególności jeden program nie może się zaczynać w momencie zakończenia drugiego.

    Redaktor naczelny poprosił Ciebie, informatyka pracującego w \textit{BajtTV},
    o napisanie programu, który wyznaczy ramówkę, tak aby najsłabszy program
    miał jak największą atrakcyjność.

    \section{Wejście}
    Pierwszy wiersz standardowego wejścia zawiera dwie liczby całkowite $n$ i
    $k$ ($1 \leq n \leq 10^5$, $1 \leq k \leq n$) oznaczające odpowiednio
    liczbę propozycji programowych oraz minimalną ich liczbę w ramówce.

    Kolejne $n$ wierszy standardowego wejścia zawiera po trzy liczby całkowite
    $b_i$, $e_i$, $a_i$ ($1 \leq b_i < e_i \leq 10^6)$, $1 \leq a_i \leq 10^6$,
    $1 \leq i \leq n$) oznaczających odpowiednio czas rozpoczęcia emisji, jej
    zakończenia oraz atrakcyjność $i$-tego programu.

    \section{Wyjście}
    Pierwszy i jedyny wiersz standardowego wyjścia powinien zawierać jedną
    liczbę całkowitą, równą atrakcyjności najsłabszego programu w optymalnej
    ramówce. Natomiast, jeżeli nie da się ułożyć żadnej ramówki, to należy wypisać $-1$.

    \makecompactexample

  \end{tasktext}
\end{document}