\documentclass[zad,zawodnik,utf8]{sinol}
  \title{Robot}
  \id{rob}
  \signature{jtom???}
  \author{Jacek Tomasiewicz} % Autor zadania
  \pagestyle{fancy}
  \iomode{stdin}
  \konkurs{XI obóz informatyczny}
  % HINT: Pola konkurs, etap, day, date uzupelnia kierownik konkursu.
  \etap{początkująca}
  \day{?}
  \date{??.09.2015}
  \RAM{32} % HINT: To pole uzupelnia opracowujacy
 
\begin{document}
  \begin{tasktext}%
Mamy daną tablicę $n$ liczb całkowitych. Na pozycji $k$ jest ustawiony robot, który musi wykonać $m$ ruchów.
Ruch robota polega na przesunięciu się do sąsiedniej komórki tablicy.
W każdej komórce robot zabiera liczbę, zastępując ją zerem.
Celem robota jest zebranie liczb o jak największej całkowitej sumie, 
a Twoim celem jest jego zaprogramowanie.

  \section{Wejście}
Pierwszy wiersz wejścia zawiera trzy liczby całkowite $n, k, m$ ($0 \leq k, m < n \leq 300\,000$), 
oznaczające odpowiednio liczbę elementów tablicy, pozycję robota oraz liczbę ruchów robota. 
Drugi wiersz wejścia zawiera $n$ liczb całkowitych $t_0, t_1, \ldots, t_{n-1}$ ($0 \leq t_i \leq 1\,000$), 
gdzie $t_i$ oznacza wartość $i$-tej komórki tablicy.

Możesz założyć, że w testach wartych około $40\%$ punktów zachodzi dodatkowy warunek $n \leq 5\,000$.

  \section{Wyjście}

Pierwszy i jedyny wiersz wyjścia powinien zawierać jedną liczbę całkowitą, 
równą maksymalnej sumie, jaką może uzbierać robot.

     \makecompactexample    

	 \medskip
	 \noindent
	 \textbf{Wyjaśnienie do przykładu:} Robot może odwiedzić komórki o indeksach ($3, 4, 5, 6$) 
	 i zebrać następujące liczby: $1 + 5 + 0 + 3 + 9 = 18$.
	 
  \end{tasktext}
\end{document}