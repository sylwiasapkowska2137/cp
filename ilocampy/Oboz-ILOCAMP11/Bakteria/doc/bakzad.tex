\documentclass[zad,zawodnik,utf8]{sinol}
  \title{Bakteria}
  \author{Jacek Tomasiewicz}
  \pagestyle{fancy}
  \signature{jtom???}
  \id{bak}
  \iomode{stdin}
  \konkurs{XI obóz informatyczny}
  \etap{początkująca}
  \day{1}
  \date{21.09.2015}
  \RAM{32}

\begin{document}
  \begin{tasktext}% Ten znak % jest istotny!
W Bajtocji istnieje system wczesnego ostrzegania przed zakażeniem bakteriologicznym.
Właśnie odkryto zalążek takiego zakażenia na kwadratowym jeziorze rozmiaru $n \times n$.

Znamy położenie zalążka i wiemy, że w ciągu jednej doby potrafi zainfekować przylegające do niego komórki. 
Dokładniej jeśli pewna komórka na pozycji $a, b$ jest zainfekowana, to w ciągu doby infekują się komórki na następujących pozycjach: $(a-1, b),(a+1,b),(a, b-1), (a, b+1)$.
Jak szybko bakteria rozprzestrzeni się na całe jezioro?

  \section{Wejście}
W pierwszym i jedynym wierszu wejścia znajdują się trzy liczby całkowite $n, x, y$ ($1 \leq x, y \leq n \leq 10^9$), 
oznaczające odpowiednio wymiary jeziora oraz położenie zalążka bakterii.

  \section{Wyjście}
Wyjście powinno zawierać jedną liczbę całkowitą, równą liczbie dni po których zostanie zainfekowane całe jezioro.

     \makecompactexample


  \end{tasktext}
\end{document}