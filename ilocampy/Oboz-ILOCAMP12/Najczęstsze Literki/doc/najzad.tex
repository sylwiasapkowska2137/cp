\documentclass[zad,zawodnik,utf8]{sinol}

\title{Najczęstsza literka}
\id{naj}
\author{} % Autor zadania
\pagestyle{fancy}
\iomode{stdin}
\konkurs{XII obóz informatyczny}
\etap{początkująca}
\day{0}
\date{17.01.2016}
\RAM{32}

\begin{document}
\begin{tasktext}%
Jaś postanowił, że codziennie będzie dopisywał jedną literkę na koniec słowa, początkowo pustego. Postanowił też, że każdego dnia będzie zapisywał, która literka występuje najczęściej w dotychczas napisanym słowie (jeśli kilka literek występuje tak samo często, to wybieramy wcześniej położoną w alfabecie literkę). Czy nie zechciałbyś pomóc Jasiowi?

  \section{Wejście}
Pierwszy wiersz wejścia zawiera jedną liczbę całkowitą $n$ ($1 \leq n \leq 10^6 $) oznaczającą liczbę dni, przez które Jaś dopisuje literki. Kolejny wiersz zawiera słowo składające się z n liter alfabetu angielskiego — jest to słowo, które powstanie po n dniach dopisywania literek.


  \section{Wyjście}
Pierwszy i jedyny wiersz wyjścia powinien zawierać jedno słowo składające się z $n$ liter alfabetu angielskiego. Na $i$-tej pozycji słowa powinna znaleźć się literka, która występowała najczęściej w słowie z $i$-tego dnia.

\makecompactexample

\end{tasktext}
\end{document}
