\documentclass[opr]{sinol}
  \signature{jtom031}
  \title{�a�cuch}  
  \id{lan}                       
  \iomode{stdin}
  \author{Jacek Tomasiewicz}         
  \history{2011.07.19}{JTom, przygotowanie opracowania autorskiego}{1.00}
  
  
\begin{document}
  \begin{tasktext}% 
    \section{Rozwi�zania poprawne}
      
	\begin{itemize}
	\item
	$O(n^2)$ --- sprawdzamy ka�d� par� liczb, kt�re nie s� s�siadami i wybieramy z nich minimum,

	\item
	$O(n)$ --- znajdujemy cztery najmniejsze liczby i z nich szukamy najmniejszej pary liczb, kt�ra podzieli �a�cuch na trzy cz�ci.

	\end{itemize}

    \section{Rozwi�zania niepoprawne}
	\begin{itemize}
	\item
	$O(n)$ --- podobne jak wzorcowe $O(n)$, tylko �e szukamy pary lub tr�jki najmniejszych liczb.
	\end{itemize}

  \end{tasktext}
\end{document}
