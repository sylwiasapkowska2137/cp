\documentclass[zad,zawodnik,utf8]{sinol}

\title{Ludzie-Komputery}
\id{lkp}
\author{Piotr Gawryluk} % Autor zadania
\pagestyle{fancy}
\iomode{stdin}
\konkurs{XII obóz informatyczny}
\etap{początkująca}
\day{4}
\date{21.01.2016}
\RAM{64}

\begin{document}
  \begin{tasktext}%

Agent Romek został wyznaczony do szpiegowania w Kraju Ludzi-Komputerów. Najpierw jednak musi uzyskać paszport, co nie jest wcale takim prostym zadaniem. Jak sama nazwa wskazuje, ludzie w tym państwie mają zmodyfikowane mózgi, aby liczyli tak szybko, jak komputer.
Aby uzyskać paszport, Romek musi rozwiązać następujące zadanie: mając ciąg liczb całkowitych, wybrać z niego najmniejszy spójny podciąg, który jeśli zostanie posortowany niemalejąco, to cały ciąg liczb będzie posortowany niemalejąco. Pomóż mu i podaj długość takiego podciągu, reszta nie powinna sprawić już mu większego problemu.
  
\section{Wejście}

Pierwszy wiersz wejścia zawiera jedną liczbę całkowitą $n$ ($1 \leq n \leq 10^{6}$), 
oznaczjącą długość ciągu który otrzymał Romek. W drugiej linii wejścia znajduje się $n$ liczb całkowitych 
$x_{i}$ ($ 1 \leq x_{i} \leq 10^{6}, 1 \leq i \leq n $), oznaczających odpowiednio i-tą liczbę w ciągu z zadania Romka.

  \section{Wyjście}

Pierwszy i jedyny wiersz wyjscia powinien zawierać jedną liczbę całkowitą, równą długości najkrótszego spójnego podciągu, który należy posortować niemalejąco tak, aby cały ciąg był posortowany niemalejąco.

     \makecompactexample

  \end{tasktext}
\end{document}