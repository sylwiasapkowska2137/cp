\documentclass[zad,zawodnik,utf8]{sinol}

\title{Graf i ty}
\id{git}
\author{Mateusz Puczel} % Autor zadania
\pagestyle{fancy}
\iomode{stdin}
\konkurs{XII obóz informatyczny}
\etap{zaawansowana}
\day{3}
\date{20.01.2016}
\RAM{128}

\begin{document}
\begin{tasktext}%
Masz dany graf o $n$ wierzchołkach ponumerowanych kolejnymi liczbami naturalnymi od $1$ do $n$, w którym dwa wierzchołki $a$ i $b$ są połączone dwukierunkową krawędzią wtedy i tylko wtedy, gdy $a$ jest dzielnikiem $b$ lub $b$ jest dzielnikiem $a$.
Jeśli $b$ jest dzielnikiem $a$, to waga tej krawędzi wynosi $\frac{a}{b}$, a jeśli $a$ jest dzielnikiem $b$, to waga krawędzi wynosi $\frac{b}{a}$.
Masz również ciąg $q$ zapytań o długość najkrótszej ścieżki w tym grafie pomiędzy pewnymi dwoma wierzchołkami.

  \section{Wejście}
W pierwszym wierszu wejścia znajdują się dwie liczby całkowite $n$ i $q$ ($1 \leq n, q \leq 10^6$), oznaczające kolejno liczbę wierzchołków w grafie oraz liczbę zapytań.
W każdym z kolejnych $q$ wierszy znajdują się dwie liczby $a$ i $b$ ($1 \leq a, b \leq n$), oznaczające numery wierzchołków, między którymi chcemy znaleźć długość najkrótszej ścieżki.
  \section{Wyjście}
Na wyjściu powinno pojawić się $q$ wierszy, a w każdym z nich jedna liczba całkowita -- długość najkrótszej ścieżki między wierzchołkami z kolejnych zapytań. 
  \makecompactexample

\end{tasktext}
\end{document}
