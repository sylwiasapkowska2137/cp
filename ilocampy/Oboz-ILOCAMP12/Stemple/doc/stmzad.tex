\documentclass[zad,zawodnik,utf8]{sinol}

\title{Stemple}
\id{stm}
\author{Mateusz Chołołowicz} % Autor zadania
\pagestyle{fancy}
\iomode{stdin}
\konkurs{XII obóz informatyczny}
\etap{zaawansowana}
\day{2}
\date{19.01.2016}
\RAM{64}

\begin{document}
  \begin{tasktext}% Ten znak % jest istotny!
Bonifacy chce pomalować swój drewniany szablon. Zamierza w tym celu użyć swoich kolorowych stempli.

Szablon składa się z $n$ pionowych pasków o szerokości $1$ cm i pewnych wysokościach.
Dolne krawędzie pasków ułożone są w jednej linii. Każdy z jego $m$ stempli jest prostokątem i ma unikalny kolor.

Malowanie szablonu polega na odciskaniu stempli w określonych miejscach tak, aby pokryć farbą całą powierzchnię figury.
Miejsce, w którym został przyłożony stempel, zostaje nasiąknięte farbą.
Chłopiec ma następujące wymagania:

1) Szablon nie może być pokryty kolorem innym niż kolory jego stempli, co mogłoby powstać po odciśnięciu na pewnym
fragmencie dwóch różnych pieczątek i wymieszaniu kolorów.

2) Szablon podczas malowania leży na cennej, antycznej podkładce. Chłopiec woli stwierdzić, że nie da się pomalować
szablonu zgodnie z jego zasadami, niż usilnie przykładać stemple do szablonu, wychodząc poza pole figury i brudzić
podkładkę.

3) Stemple muszą być przykładane do dolnej krawędzi szablonu określonym bokiem (Bonifacy nie może obracać stempli).

4) Przecinając szablon w dowolnym miejscu prostopadle do jego podstawy, oś przecięcia nie może przebiegać
przez więcej niż jeden kolor.

Bonifacy chciałby dokonać $minimalnej$ liczby przyłożeń pewnych stempli do szablonu, ponieważ jest to bardzo
czasochłonne. Zorientował się, że być może istnieje wiele sposobów zrealizowania jego zadania, zatem poprosił
Cię o pomoc. Chłopiec poinformował Cię, że możesz używać jego stempli wielokrotnie.

 \section{Wejście}

W pierwszym wierszu wejścia znajdują się dwie liczby całkowite $n, m$ ($1 \leq n, m \leq 1\,000$), oznaczające
odpowiednio długość szablonu w centymetrach oraz liczbę stempli Bonifacego.
Kolejny wiersz zawiera $n$ dodatnich liczb całkowitych, reprezentujących szerokości szablonu na kolejnych
jednocentymetrowych odcinkach figury.
W kolejnych $m$ wierszach znajdują się pary liczb całkowitych $a_i$, $b_i$, oznaczających, że $i$-ty stempel
chłopca ma wymiary $a_i$ na $b_i$, gdzie $b_i$ to długość boku przykładanego równolegle do podstawy szablonu.

  \section{Wyjście}
Jeżeli nie da się pomalować szablonu przy użyciu podanych stempli zgodnie z wymaganiami Bonifacego,
na standardowe wyjście należy wypisać jedno słowo \texttt{NIE}. W przeciwnym razie,
w pierwszym wierszu wyjścia należy wypisać minimalną liczbę wykonanych odciśnięć stempli, które pokryją
cały szablon Bonifacego i nie zabrudzą leżącej pod nim podkładki. W drugim wierszu powinny znaleźć się
numery użytych stempli w kolejności od lewej do prawej. Stemple numerujemy zgodnie z kolejnością pojawienia
się ich na wejściu, poczynając od numeru jeden. Jeżeli istnieje kilka możliwości wypisania drugiego wiersza,
należy wypisać ten, który jest wcześniejszy leksykograficznie.
     \makecompactexample

  \end{tasktext}
\end{document}