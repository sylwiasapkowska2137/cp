\documentclass[zad,zawodnik,utf8]{sinol}

\title{Krzyżyk}
\id{krz}
\author{} % Autor zadania
\pagestyle{fancy}
\iomode{stdin}
\konkurs{XII obóz informatyczny}
\etap{początkująca}
\day{2}
\date{19.01.2016}
\RAM{32}

\begin{document}
  \begin{tasktext}% Ten znak % jest istotny!
Bajtek lubi rysować. Do swoich obrazków używa różnych wzorków. Ostatnio narysował pewien obraz ze znaczków \texttt{\#}. Okazało się jednak, że jego rozmiar nie przypadł Bajtkowi do gustu. Chciałby obejrzeć kopie swego dzieła w różnej skali, aby wybrać najlepszą z nich. Nie potrafi tego jednak szybko zrobić, a więc poprosił Ciebie o pomoc.

  \section{Wejście}
Jedna liczba całkowita $n$ ($2 \leq n \leq 300$), oznaczająca skalę o której przedstawienie, prosi Cię Bajtek, a dokładniej długość boku krzyżyka.

  \section{Wyjście}
Rysunek składający się ze znaczków \texttt{\#}, przedstawiający dzieło Bajtka o podanym na wejściu rozmiarze (patrz przykład).

  \section{Przykład}
   \twocol{%
       \noindent Dla danych wejściowych:
       \includefile{../in/\ID0.in}
       \noindent poprawnym wynikiem jest:
       \includefile{../out/\ID0.out}
     }{%
       \noindent natomiast dla danych wejściowych:
       \includefile{../in/\ID1.in}
       \noindent poprawnym wynikiem jest:
       \includefile{../out/\ID1.out}
     }
  \end{tasktext}
\end{document}
