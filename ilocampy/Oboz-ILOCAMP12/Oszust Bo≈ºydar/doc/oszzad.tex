\documentclass[zad,zawodnik,utf8]{sinol}

\title{Oszust Bożydar}
\id{osz}
\author{Mateusz Chołołowicz} % Autor zadania
\pagestyle{fancy}
\iomode{stdin}
\konkurs{XII obóz informatyczny}
\etap{zaawansowana}
\day{1}
\date{18.01.2016}
\RAM{64}
 
\begin{document}
  \begin{tasktext}% Ten znak % jest istotny!
Bożydar rzuca przed siebie $n$ kamieniami, tak aby rzucić jak najbliżej wyznaczonej linii, oddalonej od niego
o $x$ metrów. Znamy odległości wszystkich rzutów chłopca.
Końcowy rezultat to sumaryczna odległość najlepszych $k$ rzutów od miejsca, w którym znajduje się linia 
(im mniejszy wynik tym lepiej). Bożydar jest kłamczuszkiem i chce przesunąć linię, tak aby zminimalizować 
swój wynik. Należy wypisać najlepszy wynik jaki może uzyskać oszust.

 \section{Wejście}
	
Pierwszy wiersz wejścia zawiera trzy liczby całkowite $n$, $x$, $k$ ($1 \leq k \leq n \leq 10^6$, 
$1 \leq x \leq 10^{12}$), oznaczające kolejno liczbę kamieni, którymi rzuca Bożydar, początkową odległość linii
od chłopca oraz liczbę rzutów wliczających się do wyniku. W następnym wierszu wejścia znajduje się $n$ liczb 
całkowitych $d_1, d_2, \ldots, d_n$ ($1 \leq d_i \leq 10^{12}$), równych odległościom kolejnych rzutów Bożydara.

  \section{Wyjście}
	Pierwszy i jedyny wiersz wyjścia powinien zawierać jedną liczbę całkowitą, równą wartości najlepszego
	(najmniejszego) wyniku, który może uzyskać Bożydar, odpowiednio przesuwając linię.

     \makecompactexample

  \end{tasktext}
\end{document}