\documentclass[zad,zawodnik,utf8]{sinol}

\title{Przechadzka}
\id{pse}
\author{Mateusz Chołołowicz} % Autor zadania
\pagestyle{fancy}
\iomode{stdin}
\konkurs{XII obóz informatyczny}
\etap{zaawansowana}
\day{4}
\date{21.01.2016}
\RAM{128}
 
\begin{document}
  \begin{tasktext}% Ten znak % jest istotny!
Król Jakubosław uwielbia przechadzać się po swoim królestwie, którego miasta łącznie z drogami
pomiędzy nimi tworzą $drzewo$. Miasta są ponumerowane liczbami od $1$ do $n$. 
Król podczas przechadzki nigdy nie spaceruje dwukrotnie pomiędzy tą samą
parą miast.


Jakubosław chce teraz zaplanować swoją najbliższą wycieczkę. To, co jest dla niego ważne przy jej wyborze,
to miasto, z którego wyruszy, długość trasy oraz miasto, w którym przechadzka się skończy. Ze względu na to,
że miast w królestwie Jakubosława jest wiele, król prosi Cię o pomoc. Przesyła Ci $m$ propozycji, w których podaje
miasto początkowe przechadzki oraz liczbę $d$ - pożądaną długość trasy rozpoczynającej się w tym mieście. 
Twoim zadaniem jest zaproponowanie mu
dowolnego miasta, które jest odległe o $d$ od miasta startowego, w którym mógłby zakończyć podróż. 
Jeżeli w królestwie nie istnieje żadne miasto odległe o $d$ od miasta wyznaczonego na początek trasy, należy poinformować o tym króla.

 \section{Wejście}
    
W pierwszym wierszu wejścia znajduje się dwie liczby całkowite $n$ i $m$ ($1 \leq n, m \leq 5 \cdot 10^5$), oznaczające odpowiednio 
liczbę miast w królestwie Jakubosława oraz liczbę propozycji. 

W nastepnych $n-1$ wierszach znajduje się opis połączeń pomiędzy miastami
w królestwie. Każdy z tych wierszy to para liczb $a_i$, $b_i$ ($1 \leq a_i, b_i \leq n, a_i \neq b_i$), która oznacza, że istnieje
bezpośrednia, dwukierunkowa droga pomiędzy miastami o numerach $a_i$ oraz $b_i$.

W kolejnych $m$ wierszach znajdują się propozycje wycieczek wysłane przez Króla. Każda z nich to dwie liczby $v_j$ oraz $d_j$
($1 \leq v_j \leq n, 0 \leq d_j < n$), oznaczające, że należy podać dowolne miasto odległe o $d_j$
od miasta o numerze $v_j$, w którym król mógłby zakończyć swą przechadzkę.

  \section{Wyjście}
    Na standardowe wyjście należy wypisać dokładnie $m$ wierszy. Każdy wiersz to odpowiedź na kolejną propozycję Jakubosława.
    Jest to numer miasta, w którym król może zakończyć swoją wycieczkę lub $-1$ jeżeli nie istnieje żadne miasto odległe o $d_j$
    od miasta początkowego.
    
    \section{Przykład}
   \twocol{
       \noindent Dla danych wejściowych
       \includefile{../in/\ID0.in}
     }{
       \noindent jednym z poprawnych wyników jest na przykład
       \includefile{../out/\ID0.out}
     }
    

  \end{tasktext}
\end{document}