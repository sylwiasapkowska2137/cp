\documentclass[zad,zawodnik,utf8]{sinol}

\title{Kwadraty}
\id{kwa}
\author{} % Autor zadania
\pagestyle{fancy}
\iomode{stdin}
\konkurs{XII obóz informatyczny}
\etap{początkująca}
\day{1}
\date{18.01.2016}
\RAM{32}

\begin{document}
  \begin{tasktext}% Ten znak % jest istotny!
Adrian bawi się kwadratami zbudowanymi z zapałek. Kwadraty są ułożone w rzędzie, a na każdej zapałce znajduje się jedna liczba całkowita ze zbioru $\{1, 2, 3, 4\}$. Wiemy, że na jednym kwadracie nie ma dwóch zapałek z tymi samymi wartościami.

Adrian może obracać kwadraty -- zawsze o 90 stopni w dowolną stronę. Zauważ, że jeden kwadrat można obrócić kilka razy.

Adrian chciałby poobracać kwadraty w taki sposób, aby stykające się boki dwóch sąsiednich kwadratów miały zawsze tą samą wartość. Zależy mu także, aby liczba obrotów była jak najmniejsza. Czy pomożesz Adrianowi?
 
  \section{Wejście}
 Pierwszy wiersz standardowego wejścia zawiera jedną liczbę całkowitą $n$ ($1 \leq n \leq 500\,000$), oznaczającą liczbę wszystkich kwadratów. Następnych $n$ wierszy zawiera opis kwadratów.

Każdy wiersz zawiera $4$ liczby całkowite $a, b, c, d$ ($1 \leq a, b, c, d \leq 4$), oznaczających kolejno wartość górnej zapałki kwadratu, dolnej, lewej i prawej.

  \section{Wyjście}
 Na standardowe wyjście należy wypisać jedną liczbę całkowitą, równą mminimalnej liczbie obrotów, które należy dokonać, aby stykające się boki kwadratów miały tą samą wartość. 

     \exampleimg{kwa0.eps}
     \makestandardexample

  \end{tasktext}
\end{document}