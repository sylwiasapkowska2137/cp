\documentclass[zad,zawodnik,utf8]{sinol}

\title{Hurtownia}
\id{hur}
\author{} % Autor zadania
\pagestyle{fancy}
\iomode{stdin}
\konkurs{XII obóz informatyczny}
\etap{olimpijska}
\day{2}
\date{19.01.2016}
\RAM{64}

\begin{document}
  \begin{tasktext}% Ten znak % jest istotny!
Bajtazar chce założyć w Bitocji nowy ośrodek swojej firmy budowlanej. Jego działalność jest bardzo znana na całym świecie jednak Bitocja nie posiada jeszcze żadnej placówki tej organizacji. Hurtownie te słyną z olbrzymich ośrodków, które dzięki swej wielkości i jakości potęgują dochody Bajtazara. Miasta w Bitocji są jednak małe, a Bajtazar potrzebuje aż trzech hurtowni, aby placówka odpowiednio prosperowała, gdyż zadaniem każdej, jest zająć się inną dziedziną budownictwa. Król państwa nie chce, żeby wszystkie $3$ budynki znajdowały się w jednym mieście, gdyż przemysł znacznie by je zanieczyścił. Bajtazar musi więc rozmieścić swoje hurtownie w trzech różnych miastach, przy czym ze względów ekonomicznych i taktycznych, zależy mu aby miasta w których rozpocznie budowę swojego ośrodka, były możliwie blisko siebie. Dokładniej nasz bohater chciałby, aby suma odległości pomiędzy każdą parą miast w których powstaną hurtownie, była jak najmniejsza. Miast w Bitocji jest bardzo dużo, a jeszcze więcej jest opcji wyboru optymalnych $3$ miast dla Bajtazara, dlatego zostałeś poproszony o pomoc!

  \section{Wejście}
W pierwszym wierszu standardowego wejścia znajdują się $2$ liczby całkowite $n$ oraz $m$, ($3 \leq n \leq m \leq 10^5$) oznaczające ilość miast w Bitocji oraz liczbę łączących je dróg. Kolejne $m$ wierszy zawiera opis sieci dróg Bitocji. Każdy wiersz opisuje jedną drogę i składa się z trzech liczb całkowitych $a_i$, $b_i$, $c_i$ ($1 \leq a_i, b_i \leq n$, $1 \leq c_i \leq 10^9$), oznaczających kolejno: miasta które łączy $i$-ta droga oraz długość tej drogi. Możesz założyć, że z każdego miasta da się dojechać pośrednio do dowolnego innego oraz że wszystkie drogi są dwukierunkowe.

  \section{Wyjście}
Na standardowe wyjście należy wypisać dokładnie jedną liczbę całkowitą, równą najmnieszej możliwej sumie odległości pomiędzy każdą parą spośród $3$ wybranych miast.

    \makecompactexample
  \end{tasktext}
\end{document}
