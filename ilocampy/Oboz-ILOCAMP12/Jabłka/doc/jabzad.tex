\documentclass[zad,zawodnik,utf8]{sinol}

\title{Jabłka}
\id{jab}
\author{Mateusz Puczel} % Autor zadania
\pagestyle{fancy}
\iomode{stdin}
\konkurs{XII obóz informatyczny}
\etap{olimpijska}
\day{1}
\date{18.01.2016}
\RAM{32}
 
\begin{document}
\begin{tasktext}%
Na stole leży $n$ jabłek ułożonych w rzędzie. Każde jabłko jest pewnego gatunku. Przemek i Jakub chcieliby podzielić się jabłkami tak,
aby każdy z nich otrzymał tyle samo jabłek każdego gatunku. W tym celu wybierają $m$ przedziałów i dla każdego z nich zastanawiają się, czy jabłka w tym przedziale
można podzielić zgodnie z ich wymaganiami. Ponieważ jabłek jest bardzo dużo, poprosili Ciebie, abyś napisał program, który rozwiąże ich problem.

  \section{Wejście}
W pierwszym wierszu wejścia znajdują się dwie liczby całkowite $n$, $m$ ($1 \leq n, m \leq 10^6$), oznaczające odpowiednio liczbę jabłek na stole
oraz liczbę zapytań Przemka i Jakuba.

W drugim wierszu wejścia znajduje się $n$ liczb całkowitych $x_1, x_2, \ldots, x_n$ ($1 \leq x_i \leq 10^9$), oznaczające gatunki kolejnych jabłek.

W każdym z kolejnych $m$ wierszy znajdują się dwie liczby całkowite $a$, $b$ ($1 \leq a \leq b \leq n$), oznaczające odpowiednio początek i koniec przedziału kolejnych zapytań.
  \section{Wyjście}
Na wyjściu powinno pojawić się $m$ wierszy. W każdym z tych wierszy powinno pojawić się słowo \texttt{TAK}, jeżeli dla kolejnych zapytań da się
podzielić jabłka tak, aby Przemek i Jakub dostali tyle samo jabłek każdego typu. W przeciwnym wypadku należy wypisać \texttt{NIE}.
\makecompactexample

\end{tasktext}
\end{document}