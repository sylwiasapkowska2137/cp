\documentclass[zad,zawodnik,utf8]{sinol}

\title{Wagony}
\id{wag}
\author{Jacek Tomasiewicz} % Autor zadania
\pagestyle{fancy}
\iomode{stdin}
\konkurs{XII obóz informatyczny}
\etap{początkująca}
\day{0}
\date{17.01.2016}
\RAM{32}
 
\begin{document}
  \begin{tasktext}%
    Na Bajtockim peronie stoi długi pociąg składający się z lokomotywy i $n$ wagonów. Każdy wagon ma określoną klasę, będącą liczbą całkowitą. W Bajtocji przyjęło się, że im wyższa klasa, tym lepsza. Wagony o klasie nieparzystej są dla niepalących, a o klasie parzystej dla palących.

Bajtek chce wybrać wagon dla niepalących, jak najlepszej klasy. Bajtek może wsiadać tylko do wagonów od $a$ do $b$, licząc od czoła lokomotywy (1 jest wagonem tuż obok lokomotywy).

    \section{Wejście}
    Pierwszy wiersz wejścia zawiera jedną liczbę całkowitą $n$ ($1 \leq  n \leq 300\,000)$, oznaczającą liczbę wagonów. Drugi wiersz wejścia zawiera $n$ liczb całkowitych $w_1, w_2, \ldots, w_n$ ($-10^5 \leq w_i \leq 10^5)$, gdzie $w_i$ oznacza klasę $i$-tego wagonu. Trzeci wiersz wejścia zawiera dwie liczby całkowite $a, b$ ($1 \leq a \leq b \leq n$), oznaczające że Bajtek może wsiadać do wagonów od $a$ do $b$, licząc od czoła lokomotywy.

    \section{Wyjście}
Pierwszy i jedyny wiersz wyjścia powinien zawierać jedną liczbę całowitą, równą maksymalnej klasie wagonu, do którego może wsiąść  Bajtek. Możesz założyć, że zawsze istnieje taki wagon.

    \makecompactexample
    
    \medskip
    \noindent
    \textbf{Wyjaśnienie do przykładu:}
    Bajtek może wsiąść do wagonów z pozycji od $3$ do $5$, czyli do wagonów o klasie ($8, 5, 7$). Najlepszą klasą niepalącą jest 7.
  \end{tasktext}
\end{document}