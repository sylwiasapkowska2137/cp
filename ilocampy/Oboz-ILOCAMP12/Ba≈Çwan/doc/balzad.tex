\documentclass[zad,zawodnik,utf8]{sinol}

\title{Bałwan}
\id{bal}
\author{Mateusz Puczel} % Autor zadania
\pagestyle{fancy}
\iomode{stdin}
\konkurs{XII obóz informatyczny}
\etap{olimpijska}
\day{2}
\date{19.01.2016}
\RAM{128}
 
\begin{document}
\begin{tasktext}%
W czasie obozu informatycznego, po zajęciach z programowania nadszedł czas wolny, w którym Przemek postanowił ulepić bałwana.
Przygotował $n$ kul śnieżnych o różnych rozmiarach, a następnie ułożył je w rzędzie. Przemek może podnieść kulę i położyć
ją na innej kuli o większym rozmiarze, która leży w tym rzędzie na prawo od kuli podnoszonej. Chłopiec chciałby w ten sposób
ulepić jak największego bałwana -- wiele kul może leżeć na sobie, o ile każda kula jest mniejsza niż wszystkie kule pod nią. Wielkością bałwana jest wówczas liczba kul użytych do ulepienia go. Przemek może zmienić rozmiar jednej z kul -- dolepić
do niej więcej śniegu lub pozbyć się go przed przystąpieniem do lepienia w celu zmaksymalizowania wielkości bałwana.

  \section{Wejście}
W pierwszym wierszu wejścia znajduje się jedna liczba całkowita $n$ ($1 \leq n \leq 10^5$), oznaczająca liczbę kul śnieżnych.

W drugim wierszu wejścia znajduje się $n$ liczb całkowitych $x_1, x_2, \ldots, x_n$ ($1 \leq x_i \leq 10^9$), oznaczających rozmiary kolejnych kul w rzędzie.

  \section{Wyjście}
Na wyjściu powinna pojawić się jedna liczba całkowita -- maksymalna wielkość ulepionego bałwana.
  \makecompactexample

\end{tasktext}
\end{document}