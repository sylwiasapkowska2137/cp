\documentclass[zad,zawodnik,utf8]{sinol}

\title{Rękawiczki}
\id{rek}
\author{} % Autor zadania
\pagestyle{fancy}
\iomode{stdin}
\konkurs{XII obóz informatyczny}
\etap{początkująca}
\day{1}
\date{18.01.2016}
\RAM{32}

\begin{document}
  \begin{tasktext}% Ten znak % jest istotny!
Bitoasia ma $n$ rękawiczek o różnych kolorach. Rękawiczki  tego samego koloru są nierozróżnialne. Przykładowo, dwie  czarne rękawiczki tworzą parę, którą Bitoasia może założyć  na mróz.

Bitoasi odcieli prąd, przez co w mieszkaniu jest bardzo  ciemno i nie można odróżnić koloru rękawiczek. 
Pomóż Bitoasi i powiedz, ile minimalnie rękawiczek musi wziać ze sobą, aby miała pewność, że wśród nich będzie para.

 \section{Wejście}
	
Pierwszy wiersz wejścia zawiera jedną liczbę całkowitą $n$  ($1 \leq n \leq 300\,000$), oznaczającą liczbę rękawiczek.  Kolejny wiersz zawiera $n$ liczb całkowitych $k_1, k_2, \ldots, k_n$ ($1 \leq k_i \leq 10^9$), gdzie $k_i$ oznacza  kolor $i$-tej rękawiczki.

Możesz założyć, że w testach wartych $70\%$ puntków  zachodzi warunek $k_i \leq 10^6$, a w testach wartych $40\%$ punktów zachodzi $n \leq 5\,000$.

  \section{Wyjście}
Pierwszy i jedyny wiersz wyjścia powinien zawierać jedną  liczbę całkowitą, równą minimalnej liczbie rękawiczek, które powinna wziąć Bitoasia. Jeśli biorąc wszystkie rękawiczki nie będzie pary, wynikiem powinna być wartość \texttt{-1}.

     \makecompactexample

  \end{tasktext}
\end{document}