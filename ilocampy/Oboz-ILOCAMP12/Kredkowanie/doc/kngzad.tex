\documentclass[zad,zawodnik,utf8]{sinol}

\title{Kredkowanie}
\id{kng}
\author{Mateusz Puczel} % Autor zadania
\pagestyle{fancy}
\iomode{stdin}
\konkurs{XII obóz informatyczny}
\etap{zaawansowana}
\day{2}
\date{19.01.2016}
\RAM{128}
 
\begin{document}
\begin{tasktext}%
Karolina jest Małą dziewczynką, która lubi informatyczne zabawy. Teraz do dyspozycji ma drzewo o $n$ wierzchołkach. Zabawa Karoliny polega na wybieraniu pary wierzchołków i pokolorowaniu kredkami wszystkich niepokolorowanych jeszcze wierzchołków na ścieżce pomiędzy nimi. Karolina ma jednak ograniczoną ilość kredek i chciałaby wiedzieć, jak dużo już z nich zużyła, więc zastanawia się, ile wierzchołków zostało już przez nią pokolorowanych.
Karolina, mimo tego, że lubi informatyczne zabawy, jeszcze nie umie programować tak dobrze, aby rozwiązać ten problem. Poza tym musi dokupić kredki. Ale za to wybrała właśnie Ciebie do pomocy!

  \section{Wejście}
W pierwszym wierszu wejścia znajdują się dwie liczby całkowite $n, q$ ($1 \leq n, q\leq 500\,000$) oznaczające liczbę wierzchołków drzewa i liczbę par wierzchołków które wybierała do kolorowania.

W kolejnych $n - 1$ wierszach znajdują się dwie liczby $a, b$ ($1 \leq a, b \leq n, a \neq b$), oznaczające krawędź między wierzchołkami $a$ i $b$.

W kolejnych $q$ wierszach znajdują się dwie liczby $a, b$ ($1 \leq a, b \leq n$) oznaczające pary wierzchołków malowane kolejno przez Karolinę.

  \section{Wyjście}
Na wyjściu powinno się znaleźć $q$ liczb całkowitych oznaczających ilość pokolorowanych wierzchołków po każdym wybraniu pary wierzchołków do malowania.

\makecompactexample

\end{tasktext}
\end{document}
