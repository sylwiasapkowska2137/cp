\documentclass[zad,zawodnik,utf8]{sinol}

\title{Operatory}
\id{ope}
\author{} % Autor zadania
\pagestyle{fancy}
\iomode{stdin}
\konkurs{XII obóz informatyczny}
\etap{olimpijska}
\day{2}
\date{19.01.2016}
\RAM{8}

\begin{document}
\begin{tasktext}%
Przemek napisał na tablicy ciąg $n$ znaków w postaci \texttt{COCOCO...COC}, gdzie \texttt{C} jest cyfrą, a \texttt{O} jest operatorem ze zbioru $\{$'\texttt{=}', '\texttt{<}', '\texttt{>}', '\texttt{+}', '\texttt{-}'$\}$.

Przemek chce wybrać pewną liczbę rozłącznych podsłów podanego ciągu, które są poprawnymi wyrażeniami. Poprawne wyrażenie to spójne podsłowo podanego ciągu, które zaczyna i kończy się cyfrą, zawiera dokładnie jeden operator porównania ze zbioru $\{$'\texttt{=}', '\texttt{<}', '\texttt{>}'$\}$ i jest prawdziwe matematycznie.

Przemek zastanawia się ile może maksymalnie wybrać poprawnych wyrażeń, które są rozłącznymi podsłowami podanego ciągu. Pomóż mu obliczyć tę wartość.

  \section{Wejście}
W pierwszym wierszu wejścia znajdują się jedna liczba całkowita nieparzysta $n$ ($1 \leq n \leq 10^7$), oznaczająca długość ciągu napisanego przez Przemka.

W drugim wierszu znajduje się $n$ znaków reprezentujących ciąg napisany przez Przemka. Znaki na pozycjach nieparzystych są cyframi, a znaki na pozycjach parzystych są operatorami.

Możesz dodatkowo założyć, że w testach wartych przynajmniej:\newline
- $10\%$ punktów zachodzi: $n \leq 10$\newline
- $30\%$ punktów zachodzi: $n \leq 10^3$\newline
- $50\%$ punktów zachodzi: $n \leq 10^5$\newline
- $70\%$ punktów zachodzi: $n \leq 5 \cdot 10^6$\newline
- $90\%$ punktów zachodzi: $n \leq 9 \cdot 10^6$

  \section{Wyjście}
Na wyjściu powinna znaleźć się jedna liczba całkowita, oznaczająca maksymalną możliwą do uzyskania liczbę rozłącznych poprawnych wyrażeń.

  \section{Przykład}
   \twocol{%
       \noindent Dla danych wejściowych:
       \includefile{../in/\ID0.in}
     }{%
       \noindent poprawnym wynikiem jest:
       \includefile{../out/\ID0.out}
     }
   \twocol{%
       \noindent natomiast dla danych wejściowych:
       \includefile{../in/\ID1.in}
     }{%
       \noindent poprawnym wynikiem jest:
       \includefile{../out/\ID1.out}
     }

\medskip
\noindent
\textbf{Wyjaśnienie do przykładu:}

W pierwszym przykładzie można wybrać tylko jedno poprawne wyrażenie: \texttt{7-5<3}.

W drugim przykładzie wybieramy następujące podsłowa: \texttt{5<6} oraz \texttt{4<5}.

\medskip
\noindent
\textbf{Uwaga:} Limit pamięci podany w treści zadania dotyczy sumarycznego zapotrzebowania na pamięć, a więc zawiera rozmiar kodu wykonywalnego, stosu, sterty, itp. Można wysłać rozwiązanie na sprawdzarkę, aby sprawdzić czy nie przekracza ono limitu pamięci na testach przykładowych.

\end{tasktext}
\end{document}