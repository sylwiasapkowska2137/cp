\documentclass[zad,zawodnik,utf8]{sinol}

\title{Kaktusy}
\id{kak}
\author{Jacek Tomasiewicz} % Autor zadania
\pagestyle{fancy}
\iomode{stdin}
\konkurs{XII obóz informatyczny}
\etap{zaawansowana}
\day{0}
\date{17.01.2016}
\RAM{32}
 

\begin{document}
  \begin{tasktext}% Ten znak % jest istotny!
Bitoadam ma ustawionych w rzędzie na balkonie $n$ kaktusów różnych wielkości. Bitoadam postanowił, że może wyrzucić (usunąć z rzędu) część kaktusów. Chciały, aby pozostałe kaktusy tworzyły pewien ciąg, który na początku cały czas rośnie, a potem cały czas maleje. Uznał, że będzie to wyglądało bardzo estetycznie. 

Bitoadam nie chciałby wyrzucić za dużo kaktusów. Poprosił Ciebie o pomoc, abyś znalazł minimalną liczbę kaktusów, po których usunięciu dostaniemy kaktusy w kolejności \emph{rosnąco-malejącej.}\footnote{Przykładem ciągu rosnąco-malejacego może być: ($3, 5, 4, 2$), ($5, 4, 3$), ale także ciag jednoelementowy ($2$). Ciągiem, który nie jest rosnąco-malejący jest przykładowo: ($3, 5, 4, 5$).}

  \section{Wejście}
	
Pierwszy wiersz wejścia zawiera jedną liczbę całkowitą $n$ ($1 \leq n \leq 500\,000$), oznaczającą liczbę kaktusów.
Drugi wiersz wejścia zawiera $n$ liczb całkowitych $k_1, k_2, \ldots, k_n$ ($1 \leq k_i \leq 10^9$), gdzie $k_i$ oznacza wysokość $i$-tego kaktusa

Możesz założyć, że w testach wartych $40\%$ puntków zachodzi dodatkowy warunek $n \leq 5\,000$.

  \section{Wyjście}
Pierwszy i jedyny wiersz wyjścia powinien zawierać jedną liczbę całkowitą, równą minimalnej liczbie kaktusów, które powinien usunąć Bitoadam.

	\makecompactexample

  \end{tasktext}
\end{document}