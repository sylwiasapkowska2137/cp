\documentclass[zad,zawodnik,utf8]{sinol}

\title{Mikstura}
\id{mik}
\author{Maciej Hołubowicz} % Autor zadania
\pagestyle{fancy}
\iomode{stdin}
\konkurs{XII obóz informatyczny}
\etap{olimpijska}
\day{3}
\date{20.01.2016}
\RAM{512}

\begin{document}
\begin{tasktext}%
Przemek jest magikiem. Ma przepis na magiczną miksturę, która składa się z $n$ składników.
Przepis mówi, że aby stworzyć miksturę trzeba wymieszać $x_i$ jednostek substancji typu $i$.
Przemek chce iść do magicznego sklepu, aby kupić roztwory do stworzenia substancji.
Każda jednostka roztworu zawiera pewną ilość każdej substancji oraz ma jakąś cenę.
Życie magika nie jest jednak sielanką, więc Przemek chciałby wydać jak najmniej pieniędzy na swoją miksturę.
Przemek może kupić w sklepie dowolną ilość każdego z roztworów. W szczególności może być to każda nieujemna liczba rzeczywista.

Oblicz minimalną kwotę wydanych pieniędzy, którą będzie musiał wydać Przemek, aby stworzyć miksturę.

  \section{Wejście}
W pierwszym wierszu wejścia znajdują się dwie liczby całkowite $n, k$ ($1 \leq n, k\leq 10$), oznaczające kolejno liczbę różnych substancji oraz liczbę dostępnych w sklepie roztworów.

W drugim wierszu wejścia znajduje się $n$ liczb całkowitych $x_1, x_2, ..., x_n$ ($0 \leq x_i \leq 10$), oznaczających ilości składników potrzebnych do przygotowania mikstury.

W kolejnych $k$ wierszach znajdują się opisy rozworów. Opis roztworu zawiera $n$ liczb całkowitych $y_1, y_2, ..., y_n$ ($0 \leq y_i \leq 10$) opisujących ilości składników w roztworze oraz liczbę całkowitą $c$ ($0 \leq c \leq 10$) oznaczającą cenę roztworu.

W $30\%$ testów zachodzi $n, k \leq 4$.

  \section{Wyjście}
Na wyjściu powinna się znaleźć jedna liczba rzeczywista oznaczająca minimalny koszt mikstury lub $-1$, gdy nie jest możliwe utworzenie takiej mikstury.
Rozwiązanie będzie akceptowane, jeśli błąd bezwzględny nie przekroczy $10^{-9}$.

\makecompactexample

\end{tasktext}
\end{document}
