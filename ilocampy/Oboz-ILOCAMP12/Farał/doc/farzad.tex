\documentclass[zad,zawodnik,utf8]{sinol}

\title{Farał}
\id{far}
\author{Mateusz Puczel} % Autor zadania
\pagestyle{fancy}
\iomode{stdin}
\konkurs{XII obóz informatyczny}
\etap{zaawansowana}
\day{3}
\date{20.01.2016}
\RAM{128}


\begin{document}
  \begin{tasktext}% Ten znak % jest istotny!
W Bajtocji zbliża się XXIV maraton.
Rafał bardzo lubi biegać i chciałby wziąć udział w maratonie, lecz ze względu na zbyt małe umiejętności oraz zbyt dużą masę
nie jest w stanie przebiec go na całej długości. Pomimo to Rafał nie odpuszcza i przygotował sobie program treningowy, który przygotuje go
do udziału.

Program treningowy zawiera $n$ odcinków o różnym stopniu trudności do pokonania. Rafał ma wytrzymałość $k$. 
Każdego dnia przebiega odcinki w ustalonej kolejności (takiej samej każdego dnia), a po każdym przebiegnięciu odcinka jego wytrzymałość zmniejsza się o poziom trudności tego odcinka.
Rafał kończy trening, gdy odcinek, który go czeka ma większą trudność niż Rafał wytrzymałość.
Wówczas przyjeżdża po niego Adam i zabiera go do domu.
W rezultacie poziom trudności każdego odcinka, który przebiegł Rafał danego dnia zmniejsza
się o $1$, a jednocześnie wytrzymałość Rafała zwiększa się o liczbę odcinków, którą przebiegł danego dnia.

Rafał chciałby wiedzieć, którego dnia będzie w stanie pokonać każdy odcinek, aby mógł z większą pewnością wystartować w maratonie.
Ponieważ Rafał w tej chwili biega, Ty napiszesz program, który obliczy, którego dnia Rafał przebiegnie wszystkie odcinki.
Jeżeli trudność odcinka spadnie poniżej $0$, staje się on dla Rafała tak prosty, że zwiększa na tym odcinku swoją wytrzymałość.

  \section{Wejście}
W pierwszym wierszu wejścia znajdują się dwie liczby naturalne $n, k$ ($1 \leq n \leq 10^5, 1 \leq k \leq 10^9$), oznaczające kolejno liczbę odcinków do pokonania oraz początkową wytrzymałość Rafała.

Kolejny wiersz zawiera $n$ liczb całkowitych $a_1, a_2, \ldots, a_n$ ($0 \leq a_i \leq 10^9$), oznaczające trudności kolejnych odcinków, które Rafał będzie pokonywał.

  \section{Wyjście}
Na wyjsciu powinna pojawić sie jedna liczba całkowita -- dzień, którego Rafał przebiegnie wszystkie odcinki lub słowo \texttt{NIE} w przypadku gdy nie da się ukończyć programu treningowego. 
Rafał zaczyna dnia numer $1$.

   \makecompactexample

  \section{Wyjaśnienie do przykładu}
Pierwszego dnia Rafał pokonuje pierwszy odcinek, jego wytrzymałość wzrasta do $7$, a trudności odcinków wyniosą kolejno: $3$ $4$ $3$ $2$.
\\Drugiego dnia pokonuje $2$ odcinki, wytrzymałość wzrasta do $9$, a trudności odcinków wynoszą kolejno: $2$ $3$ $3$ $2$.
\\Trzeciego dnia Rafał pokona $3$ odcinki, wytrzymałość wzrośnie do $12$, a trudności odcinków wyniosą kolejno: $1$ $2$ $2$ $2$.
\\Czwartego dnia Rafałowi starczy wytrzymałości aby pokonać wszystkie odcinki.

  \end{tasktext}
\end{document}
