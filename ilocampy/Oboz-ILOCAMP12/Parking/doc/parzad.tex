\documentclass[zad,zawodnik,utf8]{sinol}

\title{Parking}
\id{par}
\author{Jacek Tomasiewicz} % Autor zadania
\pagestyle{fancy}
\iomode{stdin}
\konkurs{XII obóz informatyczny}
\etap{początkująca}
\day{3}
\date{20.01.2016}
\RAM{32}

\begin{document}
  \begin{tasktext}% Ten znak % jest istotny!
Na parkingu jest $n^2$ miejsc parkingowych, na których ustawionych jest obecnie $n$ samochodów, co ciekawe dokładnie 1 samochód w każej kolumnie. Każdy z samochodów jest ustawiony przodem lub tyłem w kierunku północnym. 

Znając położenie każdego z samochodów, chcielibyśmy wiedzieć, czy możliwe jest ustawienie wszystkich samochodów w jednym rzędzie. Zakładamy, że samochody mogą jechać tylko prosto przed siebie (bez cofania, skręcania).

 \section{Wejście}
	
Pierwszy wiersz wejścia zawiera jedną liczbę całowitą $n$ ($1 \leq n \leq 1\,000\,000$), oznaczającą liczbę samochodów. Kolejny wiersz zawiera $n$ liczb całkowitych $s_1, s_2, \ldots, s_n$ ($1 \leq s_i \leq n$), gdzie $s_i$ oznacza numer rzędu (rzędy numerujemy od północy parkingu) samochodu położonego w $i$-tej kolumnie. 

Kolejny wiersz zawiera $n$ liczb całkwowitych $z_1, z_2, \ldots, z_n$ ($0 \leq z_i \leq 1$), gdzie $z_i$ oznacza zwrot $i$-tego samochody. Jeśli $z_i$ jest równe 0, to samochód ustawiony jest przodem w kierunku północnym, 1 -- w kierunku południowym.

W testach wartych $40\%$ punktów zachodzi dodatkowy warunek $n \leq 1\,000$.

  \section{Wyjście}
	Pierwszy i jedyny wiersz wyjścia powinien zawierać jedno słowo \texttt{TAK}, jeśli można ustawić samochody w jednym rzędzie, lub słowo \texttt{NIE} w przeciwnym przypadku.

     \makecompactexample

\medskip
\noindent
\textbf {Wyjaśnienie do przykładu:} Wszystkie samochody mogą ustawić się w 3 rzędzie.

  \end{tasktext}
\end{document}