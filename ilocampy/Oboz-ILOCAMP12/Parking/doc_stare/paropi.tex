\documentclass[opr]{sinol}

\usepackage{graphicx}
\usepackage{amsthm}
\newcounter{def}
\newcounter{tw}
\newcounter{lem}
\newcounter{wn}
\newcounter{obs}

\usepackage{listings}
\lstset{
  basicstyle=\small,
  keywordstyle=\bfseries,
  numbers=left,
  language=pascal,
  xleftmargin=1.2em,
  frame=TBLR,
  mathescape=true,
  numberstyle=\footnotesize,
 }

\newcommand{\definicja}[1]{\vskip 0.2cm \noindent {\bf Definicja.\thedef.} \stepcounter{def} \emph{#1} \vskip 0.3cm}
\newcommand{\twierdzenie}[1]{\vskip 0.2cm \noindent {\bf Twierdzenie.\thetw.} \stepcounter{tw} \emph{#1} \vskip 0.3cm}
\newcommand{\lemat}[1]{\noindent {\vskip 0.2cm \bf Lemat.\thelem.} \stepcounter{lem} \emph{#1} \vskip 0.3cm}
\newcommand{\wniosek}[1]{\noindent {\vskip 0.2cm \bf Wniosek.\thewn.} \stepcounter{wn} \emph{#1} \vskip 0.3cm}
\newcommand{\obserwacja}[1]{\noindent {\vskip 0.2cm \bf Obserwacja.\theobs.} \stepcounter{obs} \emph{#1} \vskip 0.3cm}
\newcommand{\intuicja}[1]{\noindent {\vskip 0.2cm \bf Intuicja.} #1 \vskip 0.3cm}
\newcommand{\dowod}[1]{\begin{proof} #1 \end{proof}}
\begin{document}
 \signature{jtom???}
  \id{par}
  \title{Parking}
  \author{Jacek Tomasiewicz}
  \iomode{stdin}
  \etap{pocz�tkuj�ca}
  \day{5}
  \date{03.12.2011}
  \RAM{32}
  \history{2011.08.12}{Jacek Tomasiewicz, przygotowanie rozwi�zania}{1.00}

\begin{tasktext}%

\setcounter{def}{1}
\setcounter{tw}{1}
\setcounter{wn}{1}
\setcounter{obs}{1}

\section{Rozwi�zanie wzorcowe $O(n)$}

Na pocz�tku obliczamy po�o�enie dw�ch samochod�w:
\begin{itemize}
	\item p�nocny -- samoch�d zwr�cony w kierunku p�nocnym, po�o�ony najbli�ej p�nocy (o najmniejszym numerze rz�du)
	\item po�udniowy -- samoch�d zwr�cony w kierunku po�udniowym, po�o�ony najbli�ej po�udnia (o najwi�kszym numerze rz�du).
\end{itemize}

Kiedy nie mo�emy ustawi� samochod�w w jednym rz�dzie? Tylko wtedy, gdy numer rz�du samochodu p�nocnego jest mniejszy od numeru rz�du samochodu po�udniowego. 

\begin{lstlisting}
wczytaj(n, s[])
poludniowy := 0
polnocny := n + 1
for i := 1 to n do
   wczytaj(z)
   if (z == 1) poludniowy = max(poludniowy, s[i])
   if (z == 0) polnocny = min(polnocny, s[i])
if (polnocny < poludniowy)
   wypisz(NIE)
else
   wypisz(TAK)
\end{lstlisting}

Z�o�ono�� czasowa i pami�ciowa takiego rozwi�zania to $O(n)$.

\end{tasktext}
\end{document}