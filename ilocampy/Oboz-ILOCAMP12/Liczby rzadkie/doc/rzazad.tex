\documentclass[zad,zawodnik,utf8]{sinol}

\title{Liczby rzadkie}
\id{rza}
\author{Michał Majewski} % Autor zadania
\pagestyle{fancy}
\iomode{stdin}
\konkurs{XII obóz informatyczny}
\etap{zaawansowana}
\day{3}
\date{20.01.2016}
\RAM{32}
 
\begin{document}
  \begin{tasktext}%
Dzi\'s b\k{e}dziemy zajmowa\'c si\k{e} liczbami rzadkimi. Aby zdefiniowa\'c, czym one w og\'ole s\k{a}, b\k{e}dziemy potrzebowali zapisa\'c liczb\k{e} naturaln\k{a} w jej reprezentacji binarnej. Gdy mamy ju\.z ci\k{a}g zer i jedynek to sprawa staje si\k{e} prosta, je\'sli w tym ci\k{a}gu pewne dwie jedynki stoj\k{a} obok siebie, liczba rzadk\k{a} nie jest, w przeciwnym razie~---~jest.

  \section{Wej\'scie}
W pierwszym i jedynym wierszu standardowego wej\'scia znajduje si\k{e} jedna liczba naturalna w systemie dziesi\k{e}tnym $n$ ($1 \leq n \leq 10^{18}$).

  \section{Wyj\'scie}

Na wyj\'sciu powinna znale\'z\'c si\k{e} jedna liczba naturalna w systemie dziesi\k{e}tnym, b\k{e}d\k{a}ca najmniejsz\k{a} liczb\k{a} rzadk\k{a} wi\k{e}ksz\k{a} od~podanego~$n$.
     \makecompactexample    
   
\medskip
\noindent

  \end{tasktext}
\end{document}


100111
543210
32 + 4 + 2 + 1 