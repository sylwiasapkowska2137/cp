\documentclass[zad,zawodnik,utf8]{sinol}

\title{Sprężyna}
\id{spr}
\author{} % Autor zadania
\pagestyle{fancy}
\iomode{stdin}
\konkurs{XII obóz informatyczny}
\etap{początkująca}
\day{2}
\date{19.01.2016}
\RAM{32}

\begin{document}
  \begin{tasktext}% Ten znak % jest istotny!

Przemek ma $n$ sprężyn.
Każdą z nich rozciągnął na pewną długość i zastanawia się teraz, jakie długości będą miały jeśli je puści.
Przemek zna rozciągliwość każdej ze sprężyn i wie,
że każda z nich będzie się poruszać po puszeczeniu w następujący sposób.
Najpierw zmniejszy się o $k$ później zwiększy się o $k-1$, zmniejszy się o $k-2$, \ldots, aż się zatrzyma.

Znajdź, jaką długość będzie miała każda ze sprężyn, jak się zatrzyma.

  \section{Wejście}
W pierwszym wierszu wejścia znajduje się jedna liczba całkowita $n$ ($1 \leq n \leq 300\,000$), oznaczająca liczbę wszystkich sprężyn.
W kolejnych $n$ wierszach znajdują się opisy sprężyn. Każdy wiersz zawiera dwie liczby całkowte $d, k$ ($1 \leq k \leq d \leq 10^9$),
oznaczające początkową długość sprężyny, oraz o ile na początku zmiejszy sie długość sprężyny.


  \section{Wyjście}
Wyjście powinno zawierać $n$ wierszy będących odpowiedziami dla kolejnych sprężyn.
Odpowiedzią powinna być jedna liczba całkowita, równa końcowej długości sprężyny.

     \makecompactexample


  \end{tasktext}
\end{document}