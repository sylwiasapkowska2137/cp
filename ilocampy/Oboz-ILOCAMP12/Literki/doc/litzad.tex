\documentclass[zad,zawodnik,utf8]{sinol}

\title{Literki}
\id{lit}
\author{Jacek Tomasiewicz} % Autor zadania
\pagestyle{fancy}
\iomode{stdin}
\konkurs{XII obóz informatyczny}
\etap{początkująca}
\day{3}
\date{20.01.2016}
\RAM{32}

\begin{document}
  \begin{tasktext}% Ten znak % jest istotny!
Mamy dane słowo składające się z $n$ liter. Chcielibyśmy wykreślić pewien początkowy fragment i pewien  końcowy, tak aby pozostała część słowa zaczynała się i kończyła tą samą literką.

Chcielibyśmy przy tym, aby suma wyciętych fragmentów była jak najmniejsza. 

  \section{Wejście}
Pierwszy i jedyny wiersz wejścia zawiera jendą liczbę całkowitą $n$ ($1 \leq n \leq 1\,000\,000$),  oznaczającą długość słowa. Kolejny wiersz wejścia zawiera słowo składające się z $n$ małych liter  alfabetu angielskiego.

  \section{Wyjście}
Wyjście powinno zawierać jedną liczbę całkowitą, równą minimalnej długości sumie fragmentów, które  powinniśmy usunąć.

     \makecompactexample

\medskip
  \noindent
  \textbf{Wyjaśnienie do przykładu:} 
Możemy usuwąć pierwszą i dwie ostatnie literki ze słowa.

  \end{tasktext}
\end{document}