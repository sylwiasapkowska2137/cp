\documentclass[zad,zawodnik,utf8]{sinol}

\title{Tablica}
\id{tab}
\author{Mateusz Chołołowicz} % Autor zadania
\pagestyle{fancy}
\iomode{stdin}
\konkurs{XII obóz informatyczny}
\etap{zaawansowana}
\day{1}
\date{18.01.2016}
\RAM{64}
 
\begin{document}
  \begin{tasktext}% Ten znak % jest istotny!
Franek ma tablicę dwuwymiarową o wysokości $n$ i szerokości $m$. W każdej komórce tablicy zapisana jest
jedna liczba całkowita.
Franek postanowił zamalować pewne komórki tablicy. Chce wybrać komórki do zamalowania tak, aby
suma liczb w zamalowanych komórkach była jak największa. Ponadto jeśli chłopieć zamalowuje w $i$-tym od góry rzędzie
komórki o indeksach od $a$-tego do $b$-tego włącznie, to musi też zamalować w rzędzie $i+1$-szym komórki 
zawierające przedział [$max(1, a-1)$, $min(b+1,m)$].
Pomóż Przemkowi obliczyć największą możliwą sumę zamalowanych komórek, jaką może uzyskać.

 \section{Wejście}
    
W pierwszym wierszu wejścia znajdują się dwie liczby całkowite $n$ i $m$ ($1 \leq n, m \leq 1000$), oznaczające odpowiednio
wysokość i szerokość tablicy Franka.

Każdy z kolejnych $n$ wierszy zawiera $m$ liczb całkowitych 
$a_{i,j}$ ($-10^9 \leq a_{i,j} \leq 10^9$ dla $1 \leq i \leq n$, $1 \leq j \leq m$).
Reprezentują one zawartość tablicy Franka w kolejności od najwyższego do najniższego wiersza.

  \section{Wyjście}
    Na wyjściu powinna znaleźć się jedna liczba całkowita, oznaczająca maksymalną możliwą do uzyskania sumę zamalowanego obszaru.

     \makecompactexample

  \end{tasktext}
\end{document}