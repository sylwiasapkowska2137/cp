\documentclass[zad,zawodnik,utf8]{sinol}

\title{Baki z wodą}
\id{bak}
\author{Mateusz Chołołowicz} % Autor zadania
\pagestyle{fancy}
\iomode{stdin}
\konkurs{XII obóz informatyczny}
\etap{olimpijska}
\day{3}
\date{20.01.2016}
\RAM{64}
 
\begin{document}
  \begin{tasktext}% Ten znak % jest istotny!
Na hawajskiej wyspie z nieznanych przyczyn stoi wieża złożona z $n$ baków na wodę, ustawionych jeden na drugim. 
Dla uproszczenia dalszych opowieści, ponumerujmy baki od góry liczbami od $1$ do $n$.
Wieża jest podzielona na $k$ grup baków. Wszystkie baki należące do tej samej grupy posiadają taką samą pojemność.
Grupy tworzą rozłączne przedziały, to znaczy, że pomiędzy dowolnymi dwoma bakami z pewnej grupy, nie znajduje
się żaden bak, który do niej nie należy. Zatem możemy ponumerować te grupy kolejnymi liczbami od $1$ do $k$ zaczynając od szczytu wieży.
Na samym dole, pod ostatnim bakiem, znajduje się ogromny, przepotężny, 
monstrualny, tytaniczny bak o nieskończonej pojemności, który jest zakopany głęboko w ziemi. 

 
Początkowo wszystkie baki w wieży są puste. Hawajscy ziomale postanowili nalewać wodę do baków (i jak pewne
się domyślasz, zwrócili się do Ciebie z prośba o pomoc). Wspomnieni ludzie
będą dolewać wodę do baków $m$-krotnie. Za każdym razem wleją dokładnie $l$ litrów wody do każdego baku należącego do
pewnego spójnego przedziału wieży. Jeżeli woda nie mieści się już w pewnym baku, to jej nadmiar spływa do baq, 
który znajduje się bezpośrednio pod nim. Jeżeli ilość wody w baku o numerze $n$, przekroczy jego pojemność, 
woda spływa z niego do baku o nieskończonej pojemności. Ziomale chciałyby wiedzieć
trzy rzeczy. Po pierwsze proszą o policzenie liczby baków, które będą pełne po zakończeniu wszystkich operacji.
Po drugie interesuje ich numer pierwszej operacji, po której pewna ilość wody spłynęła do nieskończonego baku.
Po trzecie potrzebują też informacji, ile litrów wody spłynęło łącznie do nieskończonego baku po zakończeniu wszystkich operacji. Do dzieła!

 \section{Wejście}
    
W pierwszym wierszu wejścia znajdują się trzy liczby całkowite $n$, $k$, $m$ ($1 \leq n \leq 10^9, 1 \leq m,k \leq 3 \cdot 10^5$), 
oznaczające kolejno liczbę wszystkich baków w wieży, liczbę grup, z których składa się wieża oraz liczbę operacji wlewania
wody do baków. 

W każdym z kolejnych $k$ wierszy, znajdują się dwie liczby całkowite, opisujące kolejne grupy baków. Pierwsza z nich to liczba
baków wchodzących w skład danej grupy. Liczebności wszystkich grup sumują się oczywiście do $n$. Druga liczba z każdej pary
oznacza pojemność każdego z baków należących do tej grupy. Wartości te nie będą większe niż $10^9$.

W następnych $m$ wierszach znajdują się opisy kolejnych dolań wody do wieży. Każdy wiersz składa się z trzech
liczb całkowitych $a_i$, $b_i$, $w_i$ ($1 \leq a_i \leq b_i \leq n, 1 \leq w_i \leq 1\,000$), oznaczających, że do \textit{każdego}
baku należącego do przedziału $[a_i, b_i]$, dolewanych jest $w_i$ litrów wody.

  \section{Wyjście}
    Na standardowe wyjście należy wypisać jeden wiesz, zawierający trzy liczby: liczbę baków, które będą pełne po wykonaniu
    wszystkich operacji; numer dolania, po którym do nieskończonego baku poraz pierwszy wpłynie pewna ilość wody; 
    a także ilość litrów, które do niego spłyną po dokonaniu wszystkich operacji. Jeżeli ostatnia wartość jest równa zero, 
    wtedy druga powinna wynosić -1.
    
     \makecompactexample

  \end{tasktext}
\end{document}