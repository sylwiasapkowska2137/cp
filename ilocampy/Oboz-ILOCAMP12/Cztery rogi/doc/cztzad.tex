\documentclass[zad,zawodnik,utf8]{sinol}

\title{Cztery rogi}
\id{czt}
\author{Jacek Tomasiewicz} % Autor zadania
\pagestyle{fancy}
\iomode{stdin}
\konkurs{XII obóz informatyczny}
\etap{olimpijska}
\day{0}
\date{17.01.2016}
\RAM{32}
 
\begin{document}
  \begin{tasktext}% Ten znak % jest istotny!
Mamy daną tablicę $n \times n$, wypełnioną wartościami: $0$ lub $1$. 
Chcielibyśmy znaleźć  prostokąt zaczepiony w 4 komórkach, których wartości (w każdym rogu) są równe 1. 
Dodatkowo prostokąt powinien mieć największe możliwe pole powierzchni.

Zakładamy, że dwa rogi mogą być zaczepione w tej samej komórce tablicy.

 \section{Wejście}
	
Pierwszy wiersz wejścia zawiera jedną liczbę całkowitą $n$ ($1 \leq n \leq 2\,000$),  
oznaczającą wielkość tablicy. W~$n$~kolejnych wierszach znajduje się opis tablicy. 
W każdym wierszu po $n$ wartości: $0$ lub $1$.

Możesz założyć, że w testach wartych $60\%$ punktów zachodzi warunek $n \leq 500$, 
a w testach wartych $30\%$ punktów zachodzi $n \leq 100$.

  \section{Wyjście}
	Pierwszy i jedyny wiersz wyjścia powinien zawierać jedną liczbę całkowitą, równą wartości największego pola powierzchni.
	Jeśli nie da się znaleźć żadnego prostokąta, to odpowiedzią powinna być wartość \texttt{0}.

     \makecompactexample

  \end{tasktext}
\end{document}