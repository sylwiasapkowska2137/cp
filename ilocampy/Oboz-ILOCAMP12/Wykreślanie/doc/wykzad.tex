\documentclass[zad,zawodnik,utf8]{sinol}

\title{Wykreślanie}
\id{wyk}
\author{} % Autor zadania
\pagestyle{fancy}
\iomode{stdin}
\konkurs{XII obóz informatyczny}
\etap{początkująca}
\day{2}
\date{19.01.2016}
\RAM{128}

\begin{document}
\begin{tasktext}%
Mamy danych $n$ liczb całkowitych. Możemy wykreślić maksymalnie jedną liczbę, tak aby reszta z dzielenia (modulo) iloczynu wszystkich pozostałych liczb przez $x$ była jak największa.

  \section{Wejście}
Pierwszy wiersz wejścia zawiera dwie liczby całkowite $n$, $x$ ($2 \leq n \leq 300\,000, 1 \leq x \leq 10^9$), oznaczające odpowiednio
ilość liczb oraz wartość (modulo) z treści zadania

Drugi wiersz wejścia zawiera $n$ liczb całkowitych $l_1, l_2, \ldots, l_n$ ($1 \leq l_i \leq 10^6$), gdzie $l_i$ oznacza wartość $i$-tej liczby.
  \section{Wyjście}
  Wyjście powinno zawierać jedną liczbę całkowitą, równą maksymalnej wartości reszty z dzielenia iloczynu przez $x$.
  \makecompactexample

\end{tasktext}
\end{document}