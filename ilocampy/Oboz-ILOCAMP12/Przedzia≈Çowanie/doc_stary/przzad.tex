\documentclass[zad,zawodnik]{sinol}
  \title{Przedzia�owanie}
  \author{Jacek Tomasiewicz}
  \pagestyle{fancy}
  \signature{jtom???}
  \id{prz}
  \iomode{stdin}
  \etap{�rednia}
  \day{1}
  \date{21.01.2013}
  \RAM{32}
  \history{2011.02.20}{JTom, pomys� i~redakcja}{1.00}

\begin{document}
  \begin{tasktext}% Ten znak % jest istotny!
Dla danego zbioru $n$ liczb ca�kowitych, wyznacz minimaln� liczb� przedzia��w, kt�re go pokrywaj� i nie pokrywaj� �adnej liczby ca�kowitej spoza tego zbioru.

  \section{Wej�cie}
W pierwszym wierszu wej�cia znajduje si� jedna liczba ca�kowita $t$ ($1 \leq t \leq 20$), oznaczaj�ca liczb� zestaw�w testowych. Dalej opisywane s� zestawy danych.

Pierwszy wiersz zestawu zawiera jedn� liczb� ca�kowit� $n$ $(1\leq n \leq 500\,000$), oznaczaj�c� liczb� elment�w zbioru. 
Drugi wiersz wej�cia zawiea $n$ liczb ca�kowitych $l_1, l_2, \ldots, l_n$ ($1 \leq l_i \leq 10^6$), gdzie $l_i$ oznacza warto�� $i$-tego elementu.

  \section{Wyj�cie}
Wyj�cie powinno zawiera� $t$ wierszy b�d�cych odpowiedziami dla kolejnych zestaw�w danych. W ka�dym wierszu powinna by� jedna liczba ca�kowita, r�wna minimalnej liczbie przedzia��w, pokrywaj�cej zbi�r liczb.


     \makecompactexample


  \end{tasktext}
\end{document}