\documentclass[zad,zawodnik,utf8]{sinol}

\title{Monety}
\id{mon}
\author{Jacek Tomasiewicz} % Autor zadania
\pagestyle{fancy}
\iomode{stdin}
\konkurs{XII obóz informatyczny}
\etap{początkująca}
\day{4}
\date{21.01.2016}
\RAM{32}
 
\begin{document}
  \begin{tasktext}%
Bitoasia ułożyła w rzędzie $n$ monet. Każda z nich pokazuje orła lub reszkę.
Dwie sąsiednie monety nazywamy \textit{przyległymi} jeśli pokazują tę samą stronę.
 
Bitoasia \textbf{musi} odwrócić dokładnie jedną monetę.
Chciałaby to zrobić w taki sposób, aby liczba par przyległych monet w całym ciągu była jak największa. 

  \section{Wejście}
Pierwszy wiersz wejścia zawiera jedną liczbę całkowitą $n$ ($1 \leq n \leq 300\,000$), 
oznaczającą liczbę monet. Drugi wiersz wejścia zawiera $n$ liczb całkowitych 
$m_1, m_2, \ldots, m_n$ ($m_i \in \{0, 1\}$), gdzie $m_i$ oznacza stronę $i$-tej monety: 0 --- orzeł, a 1 --- reszka.

  \section{Wyjście}

Pierwszy i jedyny wiersz wyjścia powinien zawierać jedną liczbę całkowitą, 
równą maksymalnej liczbie par przyległych monet.

     \makecompactexample    
	 
\medskip
\noindent
\textbf{Wyjaśnienie do przykładu:} Po odwróceniu trzeciej monety, przeległe będą monety: (1, 2), (2, 3), (3, 4) i (5, 6).

  \end{tasktext}
\end{document}