\documentclass[zad,zawodnik,utf8]{sinol}

\title{Wartość bezwzględna}
\id{wrt}
\author{Mateusz Chołołowicz} % Autor zadania
\pagestyle{fancy}
\iomode{stdin}
\konkurs{XII obóz informatyczny}
\etap{zaawansowana}
\day{2}
\date{19.01.2016}
\RAM{64}
 
\begin{document}
  \begin{tasktext}% Ten znak % jest istotny!
Dany jest ciąg liczb całkowitych. Chcemy wybrać z niego 3 elementy tak, aby wartość $|s - m|$ była jak największa,
gdzie $s$ oznacza średnią arytmetyczną tych liczb, a $m$ - medianę.

 \section{Wejście}
    
W pierwszym wierszu wejścia znajduje się jedna liczba całkowita $n$ ($1 \leq n \leq 10^6$), oznaczająca 
liczbę elementów w ciągu. Kolejny wiersz zawiera $n$ dodatnich liczb całkowitych, reprezentujących kolejne wartości
elementów podanego ciągu. Liczby te nie będą większe niż $10^9$.

  \section{Wyjście}
    Należy wypisać największą możliwą wartość bezwzględną z różnicy pomiędzy średnią arytmetyczną a medianą 
    pewnego trzyelementowego podzbioru podanego ciągu. Wynik powinien być wypisany z dokładnością do dwóch
    miejsc po przecinku.
     \makecompactexample

  \end{tasktext}
\end{document}