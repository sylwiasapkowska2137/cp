\documentclass[zad,zawodnik,utf8]{sinol}

\title{Dzielniki}
\id{dzi}
\author{Maciej Hołubowicz} % Autor zadania
\pagestyle{fancy}
\iomode{stdin}
\konkurs{XII obóz informatyczny}
\etap{olimpijska}
\day{3}
\date{20.01.2016}
\RAM{128}

\begin{document}
\begin{tasktext}%
Masz dane dwie liczby całkowite $x$ i $y$. Podaj najmniejszą liczbę naturalną, która ma $x^y$ dzielników.\newline
Liczba $x$ jest liczbą pierwszą.\newline \newline
Wynik podaj modulo $10^9+7$.

  \section{Wejście}
W pierwszym i jedynym wierszu wejścia znajdują się dwie liczby całkowite $x, y$ ($1 \leq x, y \leq 10^5$).\newline \newline
Dla 30\% testów zachodzi $x=2$.\newline
W innym podzbiorze testów wartym 20\% punktów zachodzi $x, y \leq 1000$.

  \section{Wyjście}
Na wyjściu powinna znaleźć się jedna liczba całkowita, mnimalna liczba, która ma $x^y$ dzielników podana modulo $10^9+7$.

\makecompactexample

\end{tasktext}
\end{document}
