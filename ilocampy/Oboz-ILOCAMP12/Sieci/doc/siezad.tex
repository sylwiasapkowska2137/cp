\documentclass[zad,zawodnik,utf8]{sinol}



\title{Sieci}

\id{sie}

\author{??} % Autor zadania

\pagestyle{fancy}

\iomode{stdin}

\konkurs{XII obóz informatyczny}

\etap{olimpijska}

\day{4}

\date{21.01.2016}

\RAM{128}

 

\begin{document}

\begin{tasktext}%

Sieć lokalna na obozie \texttt{ILOCAMP 2016 zima} ma strukturę drzewa, tzn. nie ma cykli i łączy $n$ urządzeń za pomocą $n-1$ kabli. Ponadto każdy kabel łączy dokładnie dwa różne urządzenia.



Z powodu starych kabli i urządzeń sieć jest bardzo wolna. Administrator zdefiniował dla każdego kabla wartość $t_i$ oznaczającą czas transmisji pakietu między urządzeniami $a_i$ i $b_i$ za pomocą $i$-tego kabla.

Dla dwóch komputerów $a$ i $b$ czas transmisji pakietu z jednego komputera do drugiego, to suma wartości $t_i$ dla wszystkich kabli na ścieżce z $a$ do $b$. Jedną z ważnych charakterystyk sieci jest maksymalny czas transmisji pakietu z jednego komputera do innego. Oznaczmy tą charakterystykę jako $m$.



Administrator pod naciskami Bitoasi narzekającej na wolny transfer danych, zdecydował ulepszyć sieć i zmniejszyć jej charakterystykę $m$. W tym celu chce zamienić niektóre stare kable na nowoczesne ultraszybkie kable, którymi czas przesyłu wynosi $0$. Zamiana $i$-tego kabla na kabel ultraszybki kosztuje nas $p_i$.



Pomóż mu znaleźć taki podzbiór kabli, że po ich zamianie na kable ultraszybkie, charakterystyka sieci $m$ się zmniejszy i osiągniemy to jak najmniejszym kosztem. Zauważ, że naszym celem nie jest minimalizacja $m$, a jedynie zmniejszenie tej charakterystyki.





\section{Wejście}

W pierwszej linii standardowego wejścia znajduje się liczba całkowita $n$ ($1\leq n \leq 10^5$), oznaczająca liczbę urządzeń w sieci. W następnych $n-1$ wierszach znajdują się po cztery liczby całkowite $a_i$, $b_i$, $t_i$, $p_i$ ($1\leq a_i, b_i\leq n, 1\leq t_i, p_i \leq 10^4$), oznaczające kolejno numery wierzchołków połączonych $i$-tym kablem, długość kabla oraz koszt zamiany $i$-tego kabla na kabel ultraszybki.



\section{Wyjście}

	

W pierwszej i jednynej linii wyjścia należy wypisać jedną liczbę całkowitą, oznaczającą minimalny koszt zamiany kabli, pozwalający na zmniejszenie charakterystyki $m$.



\makecompactexample



\end{tasktext}

\end{document}
