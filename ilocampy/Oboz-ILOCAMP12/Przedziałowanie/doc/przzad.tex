\documentclass[zad,zawodnik,utf8]{sinol}

\title{Przedziałowanie}
\id{prz}
\author{Jacek Tomasiewicz} % Autor zadania
\pagestyle{fancy}
\iomode{stdin}
\konkurs{XII obóz informatyczny}
\etap{początkująca}
\day{2}
\date{19.01.2016}
\RAM{32}

\begin{document}
  \begin{tasktext}% Ten znak % jest istotny!
Dla danego zbioru $n$ liczb całkowitych, wyznacz minimalną liczbę przedziałów, które go pokrywają i nie pokrywają żadnej liczby całkowitej spoza tego zbioru.

  \section{Wejście}
W pierwszym wierszu wejścia znajduje się jedna liczba całkowita $t$ ($1 \leq t \leq 20$), oznaczająca liczbę zestawów testowych. Dalej opisywane są zestawy danych.

Pierwszy wiersz zestawu zawiera jedną liczbę całkowitą $n$ $(1\leq n \leq 500\,000$), oznaczającę liczbę elmentów zbioru.
Drugi wiersz wejścia zawiea $n$ liczb całkowitych $l_1, l_2, \ldots, l_n$ ($1 \leq l_i \leq 10^6$), gdzie $l_i$ oznacza wartość $i$-tego elementu.

  \section{Wyjście}
Wyjście powinno zawierać $t$ wierszy będących odpowiedziami dla kolejnych zestawów danych. W każdym wierszu powinna być jedna liczba całkowita, równa minimalnej liczbie przedziałów, pokrywającej zbiór liczb.


     \makecompactexample


  \end{tasktext}
\end{document}
