\documentclass[zad,zawodnik,utf8]{sinol}

\title{Kolorowe drzewo}
\id{kol}
\author{Mateusz Puczel} % Autor zadania
\pagestyle{fancy}
\iomode{stdin}
\konkurs{XII obóz informatyczny}
\etap{zaawansowana}
\day{4}
\date{21.01.2016}
\RAM{128}
 
\begin{document}
\begin{tasktext}%
Przemek narysował $n$-wierzchołkowe drzewo, a następnie każdy z wierzchołków pomalował na pewien kolor. Teraz zastanawia się,
jaka jest długość najdłuższej ścieżki w tym drzewie, która zaczyna i kończy się w wierzchołkach o różnym kolorze. Każda krawędź może
w tej ścieżce pojawić się tylko raz.

  \section{Wejście}
W pierwszym wierszu wejścia znajduje się jedna liczba całkowita $n$ ($1 \leq n \leq 10^6$), oznaczająca liczbę wierzchołków w drzewie.

W każdym z kolejnych $n - 1$ wierszy znajdują się dwie liczby całkowite $a$, $b$ ($1 \leq a \neq b \leq n$), oznaczające, że w drzewie istnieje
nieskierowana krawędź pomiędzy wierzchołkami o numerach $a$ i $b$.

W ostatnim wierszu znajduje się $n$ liczb całkowitych $c_1, c_2, \ldots, c_n$ ($1 \leq c_i \leq 10^9$), oznaczające kolory kolejnych wierzchołków drzewa.
  \section{Wyjście}
Na wyjściu powinna pojawić się jedna liczba całkowita -- długość najdłuższej ścieżki pomiędzy wierzchołkami o różnych kolorach.\newline
Jeżeli taka ścieżka nie istnieje, należy wypisać $0$.  
  \makecompactexample

\end{tasktext}
\end{document}