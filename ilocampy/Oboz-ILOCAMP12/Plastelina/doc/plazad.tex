\documentclass[zad,zawodnik,utf8]{sinol}

\title{Plastelina}
\id{pla}
\author{} % Autor zadania
\pagestyle{fancy}
\iomode{stdin}
\konkurs{XII obóz informatyczny}
\etap{początkująca}
\day{1}
\date{18.01.2016}
\RAM{32}

\begin{document}
  \begin{tasktext}% Ten znak % jest istotny!
Bitoasia ulepiła z plasteliny prostokąt o wymiarach $a \times b$. Teraz może rozciągać prostokąt wzdłuż dowolnego z~boków. Rozciągnięcie polega na zwiększeniu dowolnego boku o~1. Przykładowo mając prostokąt o~wymiarach $3 \times 6$, za pomocą jednego rozciągnięcia można utworzyć prostokąt o~wymiarach $4 \times 6$ lub $3 \times 7$. 

Bitoasia chciałby za pomocą minimalnej liczby rozciągnięć dostać prostokąt o~powierzchni równej \textit{co~najmniej}~$x$.
 
  \section{Wejście}
 Pierwsza linia standardowego wejścia zawiera liczbę całkowitą $z$ ($1 \leq z \leq 100\,000$), oznaczająca liczbę dni w których Bitoasia lepi prostokąty z plasteliny. Następnych $z$ wierszy opisuje kolejne dni.

Każdy wiersz zawiera trzy liczby całkowite $a, b, x$ ($1 \leq a, b \leq 10^6, x \leq 10^{18})$, oznaczające odpowiednio wymiary prostokątu i~oczekiwaną przez Bitoasię powierzchnię.

  \section{Wyjście}
Standardowe wyjście powinno zawierać $z$ wierszy, będących odpowiedziami dla kolejnych dni. Każdy wiersz powinien zawierać jedną liczbę całkowitą, równą minimalnej liczbie ruchów, po których Bitoasia może uzyskać prostokąt o powierzchni równej \textit{co~najmniej} $x$.

     \makestandardexample

  \end{tasktext}
\end{document}