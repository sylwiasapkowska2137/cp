\documentclass[zad,zawodnik,utf8]{sinol}

\title{Zakreślanka}
\id{zak}
\author{Mateusz Chołołowicz} % Autor zadania
\pagestyle{fancy}
\iomode{stdin}
\konkurs{XII obóz informatyczny}
\etap{początkująca}
\day{1}
\date{18.01.2016}
\RAM{64}
 
\begin{document}
  \begin{tasktext}% Ten znak % jest istotny!
Stworek ma tablicę dwuwymiarową o wysokości $n$ i szerokości $m$. W każdej komórce tablicy zapisana jest
jedna liczba całkowita.
Stworek postanowił zakreślić pewne komórki tablicy. Chce wybrać komórki do zakreślenia tak, aby
suma liczb w zakreślonych komórkach była jak największa. Ponadto jeśli w $i$-tym od góry rzędzie chłopiec zakreśla
komórki o indeksach od $a$-tego do $b$-tego włącznie, to w wierszu $i+1$-szym również musi zakreślić komórki
o tych indeksach (być może jeszcze jakieś dodatkowe).
Pomóż Stworkowi obliczyć największą możliwą sumę zakreślonych komórek, jaką może uzyskać.

 \section{Wejście}
    
W pierwszym wierszu wejścia znajdują się dwie liczby całkowite $n$ i $m$ ($1 \leq n, m \leq 1000$), oznaczające odpowiednio
wysokość i szerokość tablicy Stworka.

Każdy z kolejnych $n$ wierszy zawiera $m$ liczb całkowitych 
$a_{i,j}$ ($-10^9 \leq a_{i,j} \leq 10^9$ dla $1 \leq i \leq n$, $1 \leq j \leq m$).
Reprezentują one zawartość tablicy Stworka w kolejności od najwyższego do najniższego wiersza.

  \section{Wyjście}
    Na wyjściu powinna znaleźć się jedna liczba całkowita, oznaczająca maksymalną możliwą do uzyskania sumę zakreślonego obszaru.

     \makecompactexample

  \end{tasktext}
\end{document}