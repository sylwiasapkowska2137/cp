\documentclass[zad,zawodnik,utf8]{sinol}

\title{Dzięcioły}
\id{dzi}
\author{Mateusz Chołołowicz} % Autor zadania
\pagestyle{fancy}
\iomode{stdin}
\konkurs{XIV obóz informatyczny}
\etap{zaawansowana}
\day{1}
\date{16.01.2017}
\RAM{256}
 
\begin{document}
\begin{tasktext}%
W lesie Lothl\'{o}rien od zarania dziejów rosło potężne, magiczne drzewo, które miało dla elfów olbrzymie znaczenie kulturowe. Niestety mroczne siły zesłały nań 
parszywe, jadowite robaki, które zaczęły zatruwać drzewiastą królowę. Bezsilne elfy postanowiły zawrzeć pakt z dzięciołami i wprowadzić je do niektórych dziupli
chorego drzewa. 

Dziuple znajdują się na każdym rozwidleniu gałęzi. Każda gąłąź, która prowadzi między dwiema dzuplami, ma wyceniony przez elfy stopień choroby.
Jeżeli w pewnej dziupli zamieszka dzięcioł, wydziubiuje on robaki ze wszystkich gałęzi, które bezpośrednio wychodzą z rozwidlenia, w którym ta dziupla się znajduje,
lecząc w ten sposób każdą z nich. Jeżeli na żadnym z dwóch końców gałęzi nie mieszka dzięcioł, robaki sieją na niej zarazę pogarszając stan całego drzewa
o podany przez elfy stopień choroby tej gałęzi. Władca dzięciołów - Dziobomir - ostrzegł jednak Galadierę, że jego podwładni są bardzo waleczni i wprowadzenie dzięciołów 
do dwóch sąsiednich dziupli wywoła u nich bój o krawędź pomiędzy ich schronieniami. W konsekwencji żaden z nich nie wyleczy wspomnianej gałęzi i stan drzewa się pogorszy.
Dziobomir zniecierpliwił się naradami elfów omawiającymi liczbę dzięciołów potrzebnych do wyleczenia drzewa i wysłał oddział liczący $k$ dzięciołów, 
wymuszając na elfach zakwaterowanie ich w drzewiastej królowej, nawet jeśli jest ich za dużo. Oczywiście w jednej dziupli drzewa może mieszkać co najwyżej jeden dzięcioł.

Elfy nie były szkolone w zakresie matematyki i programowania, a obawiają się, że bez tych umiejętności nie uda im się tak rozmieścić dzięciołów, aby zminimalizować
poziom choroby drzewa. Pomóż im, a na pewno uraczą Cię wyborniejszym posiłkiem niż szef kuchni Albatrosa. 

  \section{Wejście}
W pierwszym wierszu wejścia znajdują się dwie liczby całkowite $n$ i $k$ ($1 \leq k \leq n \leq 1000$), oznaczające odpowiednio liczbę dziupli w drzewie oraz 
liczbę dzięciołów, które wysłał Dziobomir. Dziuple dla uproszczenia ponumerowane są liczbami od $1$ do $n$.

W każdym z kolejnych $n-1$ wierszy znajdują się trzy liczby całkowite $a_i$, $b_i$ oraz $w_i$ ($1 \leq a_i, b_i \leq n$, $a_i~\neq~b_i$, $1 \leq w_i \leq 10^9$),
reprezentujące kolejne gałązie drzewa, gdzie $a_i$ i $b_i$ to numery dziupli pomiędzy którymi prowadzi gałąź, a $w_i$ to współczynnik choroby tej gałęzi.
Gałęzie znajdujące się na samym szczycie drzewa (czyli takie, które nie prowadzą pomiędzy dwiema dziuplami) nie wymagają leczenia, gdyż można je uciąć, dlatego
nie pojawiają się na wejściu.

 \section{Wyjście}
Na standardowe wyjście należy wypisać jedną liczbę całkowitą będącą minimalnym stopniem choroby drzewa jaki można uzyskać przy optymalnym rozmieszczeniu dzięciołów
w dziuplach.

\makecompactexample

\end{tasktext}
\end{document}