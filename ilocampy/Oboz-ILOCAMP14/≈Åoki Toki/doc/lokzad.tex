\documentclass[zad,zawodnik,utf8]{sinol}

\title{Łoki Toki}
\id{lok}
\author{Mateusz Chołołowicz} % Autor zadania
\pagestyle{fancy}
\iomode{stdin}
\konkurs{XIV obóz informatyczny}
\etap{zaawansowana}
\day{3}
\date{18.01.2017}
\RAM{256}

\usepackage{enumitem}
 
\begin{document}
\begin{tasktext}%
W klasie Pszemka panuje moda na Łoki Toki. Uczniowie bawią się nim bez opamiętania podczas godziny wychowawczej. Każde Łoki Toki może w jednej chwili
wysłać sygnał do kilku innych odbiorników. Jeśli jednak do urządzenia dociera w pewnym momencie więcej sygnałów, niektóre z nich zakłócają się i neutralizują, co w 
konsekwencji sprawia, że właściciel tego Łoki Toki może nie zdawać sobie sprawy, że ktoś próbował się z nim skontaktować. Dokładniej, sygnały
przed dotarciem do odbiornika parują się i wyciszają, zatem jeśli ich liczba jest parzysta, do Łoki Toki w tej chwili nie dotrze żaden z nich.
Jeśli zaś liczba połączeń przychodzących jest nieparzysta, jeden z nich, który nie został sparowany i zneutralizowany, wywołuje wibrację w Łoki Toki
adresata, a ten może nawiązać połączenie. 

Jak w każdej klasie, w klasie Pszemka nie wszyscy się lubią. Uczniowie podczas zabawy Łoki Toki próbują się łączyć wyłącznie z osobami, z którymi się lubią.
Podczas pierwszej sekundy godziny wychowawczej Pszemek wysłał po jednym sygnale do wszystkich swoich kolegów. Pszemek jest w klasie informatycznej, dlatego
zachowanie uczniów, które potem miało miejsce, było bardzo specyficzne. Jeżeli w danej sekundzie któryś uczeń otrzymał połączenie przychodzące 
(nastąpiła wibracja jego Łoki Toki), to w następnej sekundzie wysłał on po jednym sygnale do wszystkich swoich kolegów (gdyż nie wiedział, ilu z nich tak 
naprawdę chciało się z nim połączyć). Jeśli zaś w pewnej sekundzie uczeń nie doczekał się połączenia, popadał w smutek i w ciągu następnej sekundy
nie próbował nawiązać połączenia z żadnym ze swoich kolegów.

Może o tym nie wiesz, ale jesteś wychowawcą klasy Pszemka, którego nikt nie lubi. Nic dziwnego, skoro wstawiasz uczniom uwagę do dziennika za każdą
wibrację w jego Łoki Toki. Za karę, że nie jesteś prawilnym nauczycielem, musisz policzyć ile łącznie wibracji Łoki Toki miało miejsce podczas
pierwszych $s$ sekund godziny wychowawczej w klasie Pszemka.

  \section{Wejście}
W pierwszym wierszu wejścia znajdują się dwie liczby całkowite $n$ i $s$ ($1 \leq n \leq 25, 0 \leq s \leq 10^{18}$), oznaczające kolejno
liczbę uczniów w klasie Pszemka oraz czas, który masz za zadanie monitorować. Dla uproszczenia ponumerujmy uczniów liczbami naturalnymi od $1$ do $n$,
rozpoczynając od Pszemka. 

$I$-ty z następnych $n$ wierszy opisuje relacje koleżeńskie osoby o numerze $i$. Każdy wiersz rozpoczyna się jedną liczbą całkowitą $m_i$ 
($0 \leq m_i \leq n-1$), po której następuje $m_i$ parami różnych liczb naturalnych, oznaczających numery osób, które lubi $i$-ty uczeń. 
Relacje koleżeńskie w klasie Pszemka są obustronne, to znaczy, że jeżeli Mateusz lubi Maćka, to Maciek lubi też Mateusza. 

Możesz założyć, że na liście osób, które lubi $i$-ta osba nie pojawi się liczba $i$.


 \section{Wyjście}
Na standardowe wyjście należy wypisać jedną liczbę całkowitą równą liczbie wszystkich wibracji Łoki Toki, które nastąpiły w ciągu pierwszych $s$ sekund 
godziny wychowawczej.

\makecompactexample

\medskip
\noindent
\textbf{Wyjaśnienie do przykładu:} Na koniec pierwszej sekundy wibracja nastąpiła w Łoki Toki uczniów o numerach $2$ i $3$, na koniec drugiej sekundy
zawibrowały urządzenia uczniów o numerach $2$, $3$ i $5$, a na koniec trzeciej sekundy - uczniów o numerach $3$ i $5$.

\end{tasktext}
\end{document}