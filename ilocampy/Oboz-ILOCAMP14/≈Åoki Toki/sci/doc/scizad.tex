\documentclass[zad,zawodnik,utf8]{sinol}
	\title{Ścieżki}
	\id{sci}
	\signature{xxx0000}
	\author{Karol Farbiś}

	\pagestyle{fancy}
	\iomode{stdin}
 	\konkurs{WIC 2015, dzień świra}
	
	\day{???}
	\date{22.08.2015}

	\RAM{128}


\begin{document}
\begin{tasktext}%
	\emph{Grafem zorientowanym} nazywamy parę zbiorów $\left(V,\, E\right)$ taką, że elementami $E$ są
	uporządkowane pary elementów~$V$. Elementy zbioru $V$ nazywamy wierzchołkami,
	elementy $E$ -- krawędziami. \emph{Ścieżką} długości $k$ nazywamy ciąg wierzchołków $\left(v_1,\, v_2,\, \ldots
	,\, v_k,\, v_{k+1}\right)$ taki, że dla $i=1,\, 2,\, \ldots,\, k$ zachodzi $\left(v_i,\, v_{i+1}\right)\in E$
	(innymi słowy, każde dwa kolejne wierzchołki są połączone krawędzią).

	Dany jest graf $G$. Twoim zadaniem jest odpowiadanie na zapytania postaci: \textit{czy od wierzchołka $u$ do $v$
	istnieje ścieżka długości dokładnie $k$?}.

\section{Wejście}
	W~pierwszym wierszu wejścia znajdują się dwie liczby całkowite $n$ i $m$ ($1\leq n\leq 20$, $0\leq m\leq n(n-1)$),
	oznaczające liczbę wierzchołków i krawędzi grafu. Wierzchołkami grafu są liczby całkowite od $0$ do $n-1$.
	W~$m$~kolejnych wierszach znajduje się po dwie liczby całkowite $u$ i $v$
	($0\leq u,\, v < n$, $u\neq v$), oznaczające istnienie krawędzi
	$\left(u,\, v\right)$. Żadne dwa wiersze nie będą opisywały tej samej krawędzi.

	W~następnym wierszu znajduje się pięć liczb całkowitych $q$, $S$, $A$, $B$ i $L$ ($1\leq q\leq 10^7$, $0\leq S,\, A,\, B,\, L < 2^{64}$),
	oznaczające liczbę zapytań ($q$) i parametry generatora liczb pseudolosowych (reszta). Zapytania są generowane za~pomocą
	generatora w~następujący sposób:
	\vspace{-2ex}
	\begin{verbatim}
def rand():
    S := S * A + B  #obliczenia są wykonywane modulo 2^64
    return S

def next_query():
    u := rand() mod n
    v := rand() mod n
    k := rand() mod L

    return (u, v, k)
	\end{verbatim}
	\vspace{-2ex}

	W~około połowie testów zachodzi: $L \leq 100$, a w~około połowie (być może częściowo innej): $q\leq 10^4$.

\section{Wyjście}
	Na~wyjście należy wpisać jedną liczbę całkowitą w~systemie szesnastkowym taką, że $i$-ta cyfra w~zapisie binarnym
	odpowiada na~$i$-te zapytanie (\texttt{1} -- odpowiedź twierdząca, \texttt{0} -- odpowiedź negatywna).

\section{Przykład}
	\makeexample 0

\paragraph{Wyjaśnienie do przykładu} Zapytania w~teście przykładowym to:
(2,\,~4,\,~3),
(3,\,~4,\,~3),
(1,\,~2,\,~1),
(2,\,~2,\,~4),
(4,\,~2,\,~2),
(3,\,~1,\,~0),
(2,\,~3,\,~0),
(2,\,~0,\,~3),
(0,\,~2,\,~1).
Odpowiedzi to: $\text{000100011}_2$ = $\text{23}_{16}$.

\end{tasktext}
\end{document}
