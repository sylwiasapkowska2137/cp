\documentclass[zad,zawodnik]{sinol}
\usepackage[utf8x]{inputenc}
  \title{Największe NWD}
  \id{nwd}
  \signature{???}
  \author{Mateusz Puczel} % Autor zadania
  \pagestyle{fancy}
  \iomode{stdin}
  % HINT: Pola konkurs, etap, day, date uzupelnia kierownik konkursu.
  \etap{początkująca}
  \day{2}
  \date{17.01.2017}
  \RAM{128} % HINT: To pole uzupelnia opracowujacy
 
\begin{document}
  \begin{tasktext}%
Dany jest ciąg $n$ liczb ${a_1}, {a_2}, ..., {a_n}$. Twoim zadaniem jest znaleźć w nim dokładnie $k$ elementów o największym NWD.
  \section{Wejście}
Pierwszy wiersz wejścia zawiera 2 liczby całkowite $n$, $k$ ($1 \leq k \leq n \leq 10^6$), 
oznaczające odpowiednio długość ciągu oraz liczbę szukanych elementów.
Kolejny wiersz zawiera $n$ liczb całkowitych dodatnich, oznaczające kolejne elementy ciągu. Elementy ciągu nie przekraczają $10^6$.

Możesz założyć, że w testach wartych około $70\%$ punktów zachodzi dodatkowy warunek $n \leq 10^5$ oraz elementy ciągu nie przekraczają $5\cdot10^5$.
  \section{Wyjście}
Pierwszy i jedyny wiersz wyjścia powinien zawierać jedną liczbę całkowitą, 
równą największemu NWD znalezionego podciągu $k$-elementowego.

     \makecompactexample    

  \end{tasktext}
\end{document}