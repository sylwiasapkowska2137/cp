\documentclass[zad,zawodnik,utf8]{sinol}

\title{Trzy napisy}
\id{trz}
\author{Mateusz Puczel} % Autor zadania
\pagestyle{fancy}
\iomode{stdin}
\konkurs{XIV obóz informatyczny}
\etap{początkująca}
\day{3}
\date{18.01.2017}
\RAM{128}
 
\begin{document}
\begin{tasktext}%

Przemek na obozie informatycznym dowiedział się niedawno, co to jest porządek leksykograficzny. Dokładniej, mówimy, że napis $s$ jest mniejszy
leksykograficznie od napisu $t$, jeśli na pierwszej pozycji, na której te napisy się różnią, znak w napisie $s$ jest mniejszy od znaku w napisie $t$.
Jeżeli taka pozycja nie istnieje, to $s$ jest mniejszy leksykograficznie od $t$, jeśli jest od niego krótszy.

Przemek napisał dwa napisy $s$ i $t$ równej długości, takie że $s$ jest mniejszy leksykograficznie od $t$.  Chciałby teraz znaleźć trzeci napis $z$,
taki że $s$ jest mniejszy leksykograficznie od $z$, a $z$ jest mniejszy leksykograficznie od $t$ oraz $z$ ma taką samą długość jak $s$ i $t$.
Ponieważ napisy są długie, potrzebuje programu, który mu to obliczy.

  \section{Wejście}
W pierwszym wierszu wejścia znajduje się jedno słowo składające się z małych liter alfabetu angielskiego $s$ ($1 \leq |s| \leq 10^5$, gdzie $|s|$ oznacza długość słowa $s$).

W drugim wierszu wejścia znajduje się jedno słowo składające się z małych liter alfabetu angielskiego $t$ ($1 \leq |t| \leq 10^5, |s| = |t|$, gdzie $|t|$ oznacza długość słowa $t$).

  \section{Wyjście}
Na wyjściu powinno znaleźć się jedno słowo składające się z małych liter alfabetu angielskiego, które spełnia wymagania Przemka. Jeżeli takie słowo nie istnieje,
należy wypisać \texttt{NIE}. Jeżeli takich słów jest wiele, należy wypisać dowolne z nich.

  \section{Przykład}
   \twocol{%
       \noindent Dla danych wejściowych:
       \includefile{../in/\ID0a.in}
     }{%
       \noindent jednym z poprawnych wyników jest:
       \includefile{../out/\ID0a.out}
     }
   \twocol{%
       \noindent natomiast dla danych wejściowych:
       \includefile{../in/\ID0b.in}
     }{%
       \noindent poprawnym wynikiem jest:
       \includefile{../out/\ID0b.out}
     }

\end{tasktext}
\end{document}