\documentclass[zad,zawodnik,utf8]{sinol}
\usepackage{hyperref}

\title{Przeszkody}
\id{pre}
\author{Mateusz Puczel} % Autor zadania
\pagestyle{fancy}
\iomode{stdin}
\konkurs{XIV obóz informatyczny}
\etap{zaawansowana}
\day{2}
\date{17.01.2017}
\RAM{32}
 
\begin{document}
\begin{tasktext}%

Przemek po wielu godzinach wyczerpującego ulepszania sprawdzarki postanowił pouprawiać jakieś zajęcia na świeżym powietrzu w celu doznania relaksu.
Postanowił pokonać tor przeszkód na swoim placu zabaw.

Tor przeszkód ma kształt prostokąta o wysokości $n$ i szerokości $m$. Wyróżniamy w nim $(n+1)\cdot (m+1)$ punktów o współrzędnych całkowitych,
z punktem $(0, 0)$ w lewym górnym rogu i punktem $(n, m)$ w prawym dolnym rogu.

Na torze, w punktach o współrzędnych całkowitych mogą znajdować się środki przeszkód w kształcie koła. Każda przeszkoda ma pewien promień, może ona wystawać
poza tor oraz przecinać się niepusto z innymi przeszkodami.

Przemek zaczyna pokonywać tor przeszkód z dowolnego punktu na górnej krawędzi toru i chce go zakończyć w dowolnym punkcie na dolnej krawędzi toru.
Problem w tym, że podczas pokonywania toru przeszkód nie może dotknąć żadnej przeszkody. Dokładniej, jego odległość od środka każdej przeszkody
musi być większa niż promień tej przeszkody. Odległość mierzymy w metryce euklidesowej, tzn. odległość między dwoma punktami o współrzędnych
$(x_A, y_A)$ i $(x_B, y_B)$ wyrażona jest wzorem $\sqrt{(x_A-x_B)^2 + (y_A-y_B)^2}$.
Przemek jest bardzo chudy, więc z lotu ptaka wygląda jak punkt.
Pomoż Przemkowi stwierdzić, czy to w ogóle możliwe.

  \section{Wejście}
W pierwszej linii wejścia znajduje się jedna liczba naturalna $t$ ($1 \leq t \leq 10$) oznaczająca liczbę zestawów testowych. Następnie opisywane są kolejne zestawy.

W pierwszym wierszu zestawu znajdują się trzy liczby całkowite $n, m, k$ ($1 \leq n, m, k \leq 1\,000$) oznaczające odpowiednio rozmiar toru przeszkód oraz liczbę przeszkód.

W każdym z kolejnych $k$ wierszy znajdują się trzy liczby całkowite $x, y, r$ ($0 \leq x \leq n$, $0 \leq y \leq m$, $1 \leq r \leq 1\,000$), oznaczające odpowiednio współrzędne kolejnych przeszkód i ich promienie.
  \section{Wyjście}
Na standardowe wyjście należy wypisać $t$ wierszy, a w każdym z nich odpowiedź dla odpowiedniego zestawu danych: słowo \texttt{TAK}, jeżeli Przemek
może pokonać tor przeszkód, \texttt{NIE} w przeciwnym przypadku.

\makecompactexample
\end{tasktext}
\end{document}
