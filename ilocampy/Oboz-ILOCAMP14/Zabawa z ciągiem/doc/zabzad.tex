\documentclass[zad,zawodnik,utf8]{sinol}

\title{Zabawa z ciągiem}
\id{zab}
\author{Mateusz Puczel} % Autor zadania
\pagestyle{fancy}
\iomode{stdin}
\konkurs{XIV obóz informatyczny}
\etap{zaawansowana}
\day{?}
\date{??.01.2017}
\RAM{64}
 
\begin{document}
\begin{tasktext}%

Przemek lubi bawić się ciągami. Tym razem napisał ciąg składający się z $2n - 1$ liczb całkowitych i chciałby zmaksymalizować jego sumę.
Jedyna operacja, którą może wykonać, to wybrać dowolne $n$ liczb z tego ciągu i odwrócić znak każdej z tych liczb. Przemek może wykonać tę operację
nieskończenie wiele razy.

  \section{Wejście}
W pierwszym wierszu wejścia znajduje się jedna liczba całkowita $n$ ($1 \leq n \leq 500\,000$), oznaczająca długość ciągu Przemka.
W kolejnym wierszu wejścia znajduje się $2n - 1$ liczb całkowitych $a_1, a_2, \cdots, a_n$ ($-10^9 \leq a_i \leq 10^9$), oznaczające wartości kolejnych liczb w ciągu.

  \section{Wyjście}
Na wyjściu powinna znaleźć się jedna liczba całkowita, oznaczająca maksymalną sumę całego ciągu, jaką można uzyskać po zastosowaniu pewnej liczby opisanej operacji na podanym ciągu.
  
\makecompactexample

\end{tasktext}
\end{document}