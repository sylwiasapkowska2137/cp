\documentclass[zad,zawodnik,utf8]{sinol}

\title{Marcin Rajder}
\id{mar}
\author{Mateusz Radecki} % Autor zadania
\pagestyle{fancy}
\iomode{stdin}
\konkurs{XIV obóz informatyczny}
\etap{olimpijska}
\day{2}
\date{17.01.2017}
\RAM{512}
 
\begin{document}
\begin{tasktext}%

Przemek gra w nową grę $Marcin Rajder 5002$. Jest nią bardzo podekscytowany. Gra działa następująco.

Po uruchomieniu gry gracz wykonuje kolejne $podejścia$. Każde podejście kończy się wynikiem punktowym, który jest nieujemną liczbą całkowitą. Gracz rozpoczynając grę posiada poziom $skilla$, który rośnie wraz z grą. Na początku skill gracza wynosi $0$. Każde podejście kończy się wynikiem z przedziału $[0, aktualny\_skill]$. Gdy gracz zdobędzie ścisle więcej punktów w danym podejściu niż w jego poprzednim podejściu (bezpośrednio poprzednim), to jego skill zwiększa się o $1$. Zakładamy, że również po pierwszym podejściu graczowi wzrośnie skill o $1$.

Całą rozgrywkę składającą się z $n$ podejść nazwiemy $miłą$, jeśli nie istnieją takie liczby całkowite $i$, $j$ i $l$ ($1 \leq i < j < l \leq n$) takie, że przy $i$--tym podejściu gracz uzyska wynik $x$, przy $j$--tym uzyska wynik większy od $x$ (przez co gracz robi się szczęśliwy), zaś przy $l$--tym wynik mniejszy od $x$ (przez co gracz nagle się demotywuje, robi nieszczęśliwy i wyłącza grę w złości).

Przemek zastanawia się ile jest możliwych rozgrywek, składających się z $n$ podejść, które są miłe. Dwie rozgrywki uważamy za różne, jeśli istnieje taki numer podejścia, że podczas tych dwóch rozgrywek gracz uzyskał podczas tego podejścia różne liczby punktów.

  \section{Wejście}
W pierwszym wierszu wejścia znajduje się liczba całkowita $n$ ($1 \leq n \leq 32$), oznaczająca liczbę podejść.

  \section{Wyjście}
W pierwszym wierszu wyjścia powinna znaleźć się jedna liczba całkowita, oznaczająca liczbę rozgrywek składających się z $n$ podejść, które są miłe.
  
\makecompactexample

  \section{Wyjaśnienie do przykładu}
Ciągi punktów zdobywanych w kolejnych podejściach w możliwych rozgrywkach to:
\vskip 1mm
$(0, 0, 0, 0)$
\vskip 1mm
$(0, 0, 0, 1)$
\vskip 1mm
$(0, 0, 1, 0)$
\vskip 1mm
$(0, 1, 0, 0)$
\vskip 1mm
$(0, 0, 1, 1)$
\vskip 1mm
$(0, 1, 0, 1)$
\vskip 1mm
$(0, 1, 1, 0)$
\vskip 1mm
$(0, 1, 1, 1)$
\vskip 1mm
$(0, 1, 0, 2)$
\vskip 1mm
$(0, 1, 1, 2)$
\vskip 1mm
$(0, 1, 2, 2)$
\vskip 1mm
$(0, 0, 1, 2)$
\vskip 1mm
$(0, 1, 2, 1)$
\vskip 1mm
$(0, 1, 2, 3)$

\end{tasktext}
\end{document}
