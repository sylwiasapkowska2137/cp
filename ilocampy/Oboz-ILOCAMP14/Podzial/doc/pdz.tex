\documentclass[zad,zawodnik,utf8]{sinol}

\title{Podział}
\id{pdz}
\author{Maciej Hołubowicz} % Autor zadania
\pagestyle{fancy}
\iomode{stdin}
\konkurs{XIV obóz informatyczny}
\etap{olimpijska}
\day{3}
\date{18.01.2017}
\RAM{512}
 
\begin{document}
\begin{tasktext}%

Przemo narysował drzewo o $n$ wierchołkach. Każdy wierzchołek pomalował na jeden spośród $k$ kolorów. Teraz zastanawia się, na ile sposobów może
podzielić narysowane drzewo na spójne poddrzewa o równej wielkości (poprzez usunięcie niektórych krawędzi) tak, aby każde z nich było $wyważone$. 
Dla danego poddrzewa i koloru o numerze $i$ oznaczmy przez $r_i$ liczbę wierzchołków w tym poddrzewie, które są pokolorowane $i$-tym kolorem.
Poddrzewo takie jest wyważone wtedy, gdy $r_1 = r_2 = \dots = r_k$. 

Dwa podziały drzewa uznajemy zaś za różne wtedy, kiedy podzbiór krawędzi
usuniętych przy pierwszym podziale jest różny od podzbioru krawędzi usuniętych przy drugim podziale.

Gdy Przemo pozna odpowiedź na nurtujące go pytanie, zamierza dokonać paru transformacji na drzewie, polegających na przemalowaniu niektórych wierzchołków
na inny kolor. Po każdej transformacji chciałby rozważyć wyżej opisany problem na nowo powstałym drzewie.

Przemo jest zajęty jedzeniem $Prince Polo$, więc poświęć chwilę i pomóż mu znaleźć odpwowiedzi na jego pytania, jeśli zależy Ci na zdobyciu większej
liczby punktów podczas dzisiejszego contestu.

  \section{Wejście}
W pierwszym wierszu wejścia znajdują się trzy liczby całkowite $n$, $k$, $q$ ($1 \leq n, k, q \leq 10^5$), oznaczające kolejno liczbę wierzchołków
w drzewie Przema, liczbę kolorów, którymi dysponuje oraz liczbę transformacji, które zamierza wykonać. Wierzchołki drzewa numerujemy liczbami od $1$ do $n$.
W każdym z kolejnych $n-1$ wierszy znajdują się dwie liczby całkowite $a_i$ i $b_i$ ($1 \leq a_i, b_i \leq n$), które oznaczają, że w narysowanym przez Przema
drzewie istnieje krawędź pomiędzy wierzchołkami o numerach $a_i$ oraz $b_i$.

Następny wiersz wejścia zawiera $n$ liczb całkowitych $c_1, c_2, \dots, c_n$ ($1 \leq c_i \leq k$). $I$-ta z nich oznacza kolor, którym został oryginalnie pomalowany
wierzchołek o numerze $i$.

$Q$ kolejnych wierszy opisuje transformacje, których zamierza dokonać Przemo po znalezieniu odpowiedzi dla pierwotnie postawionego problemu.
Każdy wiersz składa się z dwóch liczb całkowituch $a_i$ oraz $d_i$ ($1 \leq a_i \leq n$, $1 \leq d_i \leq k$), które oznaczają, że w $i$-tej transformacji drzewa
Przemo zamierza przemalować wierzchołek o numerze $a_i$ na kolor o numerze $d_i$.

  \section{Wyjście}
Na standardowe wyjście należy wypisać $q+1$ liczb całkowitych. Pierwsza z nich powinna być równa liczbie sposobów na które można podzielić drzewo
narysowane przez chłopca tak, aby uzyskać pewną liczbę wyważonych poddrzew o równej wielkości. $I$-ta z kolejnych $q$ liczb powinna być odpowiedzią
na powyższy problem postawiony wobec drzewa uzyskanego po dokonaniu $i$-tej transformacji z wejścia.
  
\makecompactexample

\end{tasktext}
\end{document}
