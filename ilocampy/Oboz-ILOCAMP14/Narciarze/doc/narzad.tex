\documentclass[zad,zawodnik,utf8]{sinol}
  \title{Narciarze}
  \id{nar}
  \signature{jtom???}
  \author{Piotr Gawryluk} % Autor zadania
  \pagestyle{fancy}
  \iomode{stdin}
  % HINT: Pola konkurs, etap, day, date uzupelnia kierownik konkursu.
  \etap{początkująca}
  \konkurs{XIV obóz informatyczny}
  \day{2}
  \date{17.01.2017}
  \RAM{128} % HINT: To pole uzupelnia opracowujacy
 
\begin{document}
  \begin{tasktext}%

W Górach Bajtolandii znajduje się ogromny stok narciarski, podzielony na fragmenty. Każdy fragment ma swoją ustaloną trudność.
Trasą zjazdu nazwiemy spójny fragment stoku. Przeprowadzono badania, z których wynika, że aby trasa zjazdu była dochodowa, nie może być zbyt monotonna,
to znaczy musi zawierac co najmniej dwa fragmenty o różnych stopniach trudności.
Tak więc trasa składająca się z fragmentów o trudności $3, 3, 3$ nie jest dochodowa, a $3, 2, 3$ już jest.
W szczególności trasa składająca się z jednego fragmentu nigdy nie jest dochodowa'.

Zostałeś poproszony o pomoc właścicielowi w obliczeniu na ile sposobów klienci mogą wybrać dochodową trasę zjazdu.

  \section{Wejście}

Pierwszy wiersz wejścia zawiera jedną liczbę całkowitą $n$ ($1 \leq n \leq 10^{6}$), 
oznaczającą ilość fragmentów stoku narciarskiego. W drugiej linii wejścia znajduje się $n$ liczb całkowitych 
$x_{i}$ ($ 1 \leq x_{i} \leq n, 1 \leq i \leq 10^{6} $), oznaczających odpowiednio poziom trudności i-tego 
fragmentu stoku.

  \section{Wyjście}

Pierwszy i jedyny wiersz wyjscia powinien zawierać jedną liczbę całkowitą, równą ilości różnych spójnych 
tras zjazdu (zaczynających się i kończących na dowolnym fragmencie stoku), zawierających fragmenty o 
różnej trudności.

     \makecompactexample

  \end{tasktext}
\end{document}
