\documentclass[zad,zawodnik,utf8]{sinol}

\title{Arktyka}
\id{ark}
\author{Mateusz Radecki} % Autor zadania
\pagestyle{fancy}
\iomode{stdin}
\konkurs{XIV obóz informatyczny}
\etap{zaawansowana}
\day{0}
\date{15.01.2017}
\RAM{256}
 
\begin{document}
\begin{tasktext}%

W Arktyce trwa lato. Latem część pokrywy lodowej topi się i pojawiają się wody Oceanu Arktycznego. Sytuację wykorzystują foki, które chętnie wskakują na przepływającą krę i na niej wypoczywają. Foki są jednak wybredne, dlatego wybierają tylko te kwadratowe akweny, w których liczba pływających kawałków kry wynosi dokładnie $l$. Policz, ile jest takich obszarów.

  \section{Wejście}
W pierwszym wierszu wejścia znajdują się trzy liczby naturalne $n$, $m$ i $l$ ($1 \leq n,m \leq 10^3; 0 \leq l \leq nm $), oznaczające odpowiednio wysokość i szerokość mapy oceanu oraz liczbę kawałków kry, jaka może pływać w kwadratowym akwenie. W kolejnych liniach znajduje się mapa oceanu, gdzie . oznacza wodę, zaś \# – krę.

  \section{Wyjście}
Wypisz liczbę takich kwadratowych akwenów, w których znajduje się dokładnie $l$ kawałków kry.
  
\makecompactexample

\end{tasktext}
\end{document}
