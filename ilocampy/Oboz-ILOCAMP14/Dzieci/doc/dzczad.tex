\documentclass[zad,zawodnik,utf8]{sinol}

\title{Dzieci}
\id{dzc}
\author{Mateusz Puczel} % Autor zadania
\pagestyle{fancy}
\iomode{stdin}
\konkurs{XIV obóz informatyczny}
\etap{początkująca}
\day{3}
\date{18.01.2017}
\RAM{128}
 
\begin{document}
\begin{tasktext}%

W rzędzie stoi $n$ dzieci ponumerowanych kolejnymi liczbami naturalnymi od $1$ do $n$. Przemek chce im rozdać cukierki.
Dziecko o numerze $i$ chce dostać co najmniej $a_i$ cukierków.

Przemek rozdaje cukierki w nastepujący sposób. Zaczyna od dziecka na początku rzędu i po kolei podchodzi do każdego dziecka i daje mu paczkę, w której jest $m$ cukierków.
Jeżeli dziecko dostało już co najmniej tyle cukierków, ile chciało, to idzie do domu. W przeciwnym wypadku staje na końcu rzędu i dalej oczekuje na kolejną paczkę.
Przemek rozdaje cukierki tak długo, aż wszyscy pójdą do domu.

Twoim zadaniem jest wyznaczyć, które dziecko pójdzie do domu ostatnie.

  \section{Wejście}
W pierwszym wierszu wejścia znajdują się dwie liczby całkowite $n, m$ ($1 \leq n \leq 10^6$, $1 \leq m \leq 10^9$), oznaczające odpowiednio
liczbę dzieci w rzędzie oraz liczbę cukierków w jednej paczce.

W kolejnym wierszu wejścia znajduje się $n$ liczb całkowitych $a_1, a_2, \cdots, a_n$ ($1 \leq a_i \leq 10^9$), oznaczające liczbę cukierków, jaką chcą otrzymać kolejne dzieci.

  \section{Wyjście}
Na wyjściu powinna znaleźć się jedna liczba całkowita, oznaczająca numer dziecka, które jako ostatnie pójdzie do domu.
  
\makecompactexample

\end{tasktext}
\end{document}