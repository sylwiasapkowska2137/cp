\documentclass[zad,zawodnik,utf8]{sinol}

\title{Młot obozu}
\id{zaz}
\author{Mateusz Radecki} % Autor zadania
\pagestyle{fancy}
\iomode{stdin}
\konkurs{XIV obóz informatyczny}
\etap{olimpijska}
\day{3}
\date{18.01.2017}
\RAM{512}
 
\begin{document}
\begin{tasktext}%
    
    Masz dany graf zawierający $n$ wierzchołków i początkowo $0$ krawędzi, oraz $q$ zapytań. Zapytania są jednego z trzech typów:
\vskip 1mm
$1:$ dodaj krawędź nieskierowaną między wierzchołkami $a$ i $b$ o wadze $x$.
\vskip 1mm
$2:$ usuń krawędź nieskierowaną spomiędzy wierzchołków $a$ i $b$.
\vskip 1mm
$3:$ powiedz, czy między wierzchołkami $a$ i $b$ istnieje ścieżka (niekoniecznie prosta, może przechodzić wielokrotnie przez wierzchołki i krawędzie), której xor wag krawędzi jest równy $x$.

  \section{Wejście}

  W pierwszej linii wejścia znajdują się dwie liczby całkowite $n$ i $q$ ($1 \leq n,q \leq 200\,000$), oznaczające liczbę wierzchołków i liczbę zapytań. W kolejnych $q$ wierszach znajdują się zapytania, każde jest w jednej z podanych form:
\vskip 1mm
$1 ~ a ~ b ~ x$ -- oznacza dodanie krawędzi między $a$ i $b$ o wadze $x$ ($1 \leq a,b \leq n;~ a \neq b;~ 0 \leq x < 1024 $).
\vskip 1mm
$2 ~ a ~ b$ -- oznacza usunięcie krawędzi łączącej wierzchołki $a$ i $b$ ($1 \leq a,b \leq n;~ a \neq b$).
\vskip 1mm
$3 ~ a ~ b ~ x$ -- oznacza zapytanie o istnienie ścieżki o xorze $x$ między wierzchołkami $a$ i $b$ ($1 \leq a,b \leq n;~ 0 \leq x < 1024 $).
\vskip 4mm
Gwarantujemy, że graf w żadnym momencie nie będzie zawierał multikrawędzi, pętli, oraz że zawsze usuwamy istniejącą krawędź.
\vskip 4mm
W testach wartych przynajmniej $20\%$ punktów wagi wszystkich krawędzi będą równe $0$.

  \section{Wyjście}
   
   Dla każdego zapytania trzeciego typu wypisz w osobnej linii \texttt{TAK}, lub \texttt{NIE}, w zależności od tego czy istnieje żądana ścieżka.
  
\makecompactexample

\end{tasktext}
\end{document}
