\documentclass[zad,zawodnik,utf8]{sinol}

\title{Ukorzenianie}
\id{uko}
\author{Karol Waszczuk} % Autor zadania
\pagestyle{fancy}
\iomode{stdin}
\konkurs{XIV obóz informatyczny}
\etap{olimpijska}
\day{0}
\date{15.01.2017}
\RAM{64}
 
\begin{document}
\begin{tasktext}%


Dane jest $n$-wierzchołkowe drzewo. Krawędzie są jednak skierowane w losowy sposób. Chcemy je tak poodwracać, aby wszystkie prowadziły w kierunku jednego wierzchołka – korzenia. Każda krawędź należy do jednej z $k$ grup. Odwracać można tylko całe grupy krawędzi (tj. odwrócenie grupy oznacza zmianę kierunków wszystkich należących do niej krawędzi na przeciwne). 

Zadanie polega na znalezieniu najmniejszego numeru wierzchołka, który można uczynić korzeniem poprzez odpowiednie poodwracanie grup krawędzi.

  \section{Wejście}
  
W pierwszym wierszu standardowego wejścia znajdują się dwie liczby całkowite $n$ i $k$ ($1 \leq n, k \leq 10^5$). 

W kolejnych $n - 1$ wierszach znajduje się opis drzewa. W $i + 1$ wierszu znajdują się trzy liczby całkowite $a_i$ , $b_i$ oraz
$g_i$. Oznaczają one, że od wierzchołka $a_i$ prowadzi skierowana krawędź do wierzchołka $b_i$ należąca do grupy $g_i$. Wierzchołki są ponumerowane od $1$ do $n$, zaś grupy od $1$ do $k$.

W testach wartych przynajmniej 10\% punktów zachodzą dodatkowe warunki $n \leq 10$, $k \leq 10$. 

W testach wartych przynajmniej 20\% punktów zachodzą dodatkowe warunki $n \leq 100, k \leq 100$. 

W testach wartych przynajmniej 30\% punktów zachodzi dodatkowy warunek $n \leq 1\ 000$.

\section{Wyjście}
W pierwszym wierszu standardowego wyjścia należy wypisać jedną liczbę całkowitą – najniższy możliwy numer korzenia. Dane wejściowe są tak dobrane, że zawsze istnieje taki wierzchołek.

\makecompactexample
\end{tasktext}
\end{document}
