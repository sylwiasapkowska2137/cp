\documentclass[zad,zawodnik,utf8]{sinol}

\title{Gra w klocki}
\id{gwk}
\author{Karol Waszczuk} % Autor zadania
\pagestyle{fancy}
\iomode{stdin}
\konkurs{XIV obóz informatyczny}
\etap{początkująca}
\day{1}
\date{16.01.2017}
\RAM{32}
 
\begin{document}
\begin{tasktext}%

Przemek gra z jego kolegą Jakubem w pewną grę. Jej przebieg prezentuje się następująco. 

Na dwóch nieskończenie długich planszach o szerokości 1, w taki sam sposób ustawianych jest $n$ klocków na polach $a_1$, $a_2$, ..., $a_n$. Każdy z klocków posiada pewną wysokość $h_i$. Jedna z tych planszy należy do Jakuba, a druga do Przemka. Chłopcy mogą przewrócić każdy z klocków w lewo lub prawo, a kiedy to nastąpi upada on na pola z przedziału [$a_i$ - $h_i$, $a_i$] lub [$a_i$, $a_i$ + $h_i$]. Klocek może też pozostać nietknięty i w takim przypadku zajmuje on jedynie pole $a_i$. Zabronione jest takie przewrócenie klocka, którego efektem będzie jego upadek na pole zajmowane przez inny klocek. Wygrywa ten kto przewróci więcej klocków.

Znając wynik Jakuba, Przemek chciałby wiedzieć czy ma szansę pokonać go w danej partii, jeśli uzyska najwyższy do osiągnięcia wynik. Należy pamiętać, że z Jakuba jest cwany lis i czasem zdarza mu się oszukiwać, podając niemożliwą do uzyskania liczbę punktów! W takim przypadku wygrywa Jakub otrzymując walkower.

  \section{Wejście}

W pierwszym wierszu standardowego wejścia znajdują się dwie liczby całkowite $n$ oraz $k$ ($1 \leq n \leq 3 \cdot 10^5$, $1 \leq k \leq 10^9$) oznaczające kolejno liczbę klocków na planszach oraz uzyskany przez Jakuba wynik.

W każdym z kolejnych $n$ wierszy znajdują się dwie liczby całkowitych $a_i$ oraz $h_i$ ($1 \leq a_i \leq 10^6$, $1 \leq h_i \leq 10^9$, $ a_i < a_{i+1}$), oznaczające numer pola na którym stoi $i$-ty klocek oraz jego wysokość. Żadne dwa klocki nie stoją na tym samym polu.

  \section{Wyjście}
Na standardowe wyjście należy wypisać jedno słowo \texttt{NIE}, jeśli Przemek nie może wygrać z Jakubem lub słowo \texttt{TAK} i liczbę $w$ będącą maksymalnym do uzyskania wynikiem, jeśli Przemek ma szansę pokonać Jakuba.

\makecompactexample 


\hspace{1cm} \textbf{\newline Wyjaśnienie do przykładu:} Wynik 4 możemy uzyskać w następujący sposób: 
\begin{itemize}
	\item Pierwszy klocek przewracamy w lewo  - teraz zajmuje on pola $[-3, 2]$
	\item Drugi klocek przewracamy w prawo - teraz zajmuje on pola $[3, 6]$
	\item Trzeci klocek przewracamy w lewo - teraz zajmuje on pola $[7, 9]$
	\item Czwarty klocek pozostawiamy nietknięty - zajmuje on jedynie pole $11$
	\item Piąty klocek przewracamy w prawo - teraz zajmuje on pola $[14, 16]$
\end{itemize}
\end{tasktext}
\end{document}
