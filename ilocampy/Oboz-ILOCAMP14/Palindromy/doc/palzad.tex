\documentclass[zad,zawodnik,utf8]{sinol}
  \title{Palindromy}
  \id{pal}
  \signature{jtom???}
  \author{Jacek Tomasiewicz} % Autor zadania
  \pagestyle{fancy}
  \iomode{stdin}
  \konkurs{XIV obóz informatyczny}
  \etap{zaawansowana}
  \day{3}
  \date{18.01.2017}
  \RAM{64} % HINT: To pole uzupelnia opracowujacy
 
\begin{document}
  \begin{tasktext}%
Adrian ma bardzo długi tekst $t$. Chciałby teraz wybrać taki spójny fragment tego tekstu, aby po usunięciu dokładnie jednej litery, fragment był palindromem. Przypomnijmy, że palindrom, to słowo, które czytane od przodu i od tyłu jest takie same.

Adrian chciałby dodatkowo, aby wybrany przez niego fragment był jak największy. Pomóż mu i podaj maksymalną długość jaką może znaleźć.

  \section{Wejście}
Pierwszy wiersz wejścia zawiera jedna liczbę całkowitą $n$ ($1 \leq n \leq 10^6$), oznaczającą długość tekstu $t$.

Kolejny wiersz zawiera tekst $t$, złożony z $n$ małych liter alfabetu angielskiego.

Możesz założyć, że w testach wartych $50\%$ puntków zachodzi dodatkowy warunek $n \leq 10^4$, a w testach wartych $30\%$ puktów zachodzi $n \leq 500$.

  \section{Wyjście}
Pierwszy i jedyny wiersz wyjścia powinien zawierać jedną liczbę całkowitą, równą maksymalnej długość fragmentu, który powinien wybrać Adrian.

     \makecompactexample    

  \end{tasktext}
\end{document}