\documentclass[zad,zawodnik,utf8]{sinol}

\title{Wstaw \#}
\id{wst}
\author{Maciej Hołubowicz} % Autor zadania
\pagestyle{fancy}
\iomode{stdin}
\konkurs{XIV obóz informatyczny}
\etap{olimpijska}
\day{2}
\date{17.01.2017}
\RAM{256}
 
\begin{document}
\begin{tasktext}%

Dane jest słowo $s$ o długości $n$ składające się wyłącznie z małych liter języka angielskiego. Podsłowo słowa $s$ jest jedwabiste wtedy, gdy liczba jego występień w $s$ 
jest większa niż jeden. Musisz zamienić jedną literę $s$ na znak \# tak, aby długość najdłuższego jedwabistego podsłowa w nowopowstałym słowie była jak największa.

  \section{Wejście}
W pierwszym wierszu wejścia znajduje się jedna liczba naturalna $n$ ($1 \leq n \leq 10^5$), oznaczająca długość słowa $s$, które podane będzie w drugim wierszu wejścia.

  \section{Wyjście}
Na stadardowe wyjście należy wypisać dwie liczby całkowite $l$ i $p$ oznaczające kolejno długość najdłuższego jedwabistego podsłowa po zamianie jednej litery słowa $s$
na \# oraz indeks litery, na której należy dokonać zamiany. Jeżeli istnieje wiele takich indeksów, należy wypisać najmniejszy z nich. Litery słowa numerujemy od $1$ do $l$.
  
\makecompactexample
\noindent
\textbf{Wyjaśnienie do przykładu:} Wynikowe podsłowo jedwabiste to $aab$.

\end{tasktext}
\end{document}
