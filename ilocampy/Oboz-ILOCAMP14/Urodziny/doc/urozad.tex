\documentclass[zad,zawodnik, utf8]{sinol}
  \title{Urodziny}
  \author{Jacek Tomasiewicz}
  \pagestyle{fancy}
  \signature{jtom???}
  \id{uro}
  \iomode{stdin}
  \konkurs{XIV obóz informatyczny}
  \etap{zaawansowana}
  \day{0}
  \date{15.01.2017}
  \RAM{64}
  \history{JTom, pomysł i redakcja}{2012.07.07}{1.00}

\begin{document}
  \begin{tasktext}% Ten znak % jest istotny!
W dniu urodzin króla Bajtazara zorganizowano przyjęcie. Zaproszono na nie jedynie małżeństwa zgodnie z króla życzeniem. Wszystkie pary mają zasiąść przy długim stole, przy którym znajduje się tyle samo miejsc z jednej strony i drugiej strony stołu. 

Każdą osobę charakteryzuje pewien wiek. Król zażyczył sobie, aby różnica między sumą wieku wszystkich osób zasiadających po jednej stronie stołu, a sumą wieku wszystkich osób zasiadających po drugiej stronie stołu była jak najmniejsza. Chciałby także, aby osoby z tego samego małżeństwa nie siedziały po tej samej stronie stołu. 

Znajdź wartość tej różnicy, wiedząc, że w każdym małżeństwie różnica wieku nie przekracza stu lat. 

  \section{Wejście}
Pierwszy wiersz standardowego wejścia zawiera jedną liczbę całkowitą $n$ ($1 \leq n \leq 500$), oznaczającą liczbę małżeństw zaproszonych na przyjęcie. 

Kolejnych $n$ wierszy zawiera po dwie liczby całkowite $m_i, k_i$ ($1\leq m_i, k_i \leq 10^9, |m_i-k_i| \leq 100$), oznaczających odpowiednio wiek mężczyzny i wiek kobiety w $i$-tym małżeństwie.

  \section{Wyjście}
	
Pierwszy i jedyny wiersz standardowego wyjścia powinien zawierać jedną liczbę całkowitą, równą minimalnej różnicy między wiekiem osób siedzących po jednej stronie, a wiekiem osób siedzących po drugiej stronie.

     \makecompactexample


  \end{tasktext}
\end{document}