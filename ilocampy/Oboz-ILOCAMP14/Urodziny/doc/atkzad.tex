\documentclass[zad,zawodnik,utf8]{sinol}
\usepackage{hyperref}

\title{Atak hakerski}
\id{atk}
\author{Karol Waszczuk} % Autor zadania
\pagestyle{fancy}
\iomode{stdin}
\konkurs{XIV obóz informatyczny}
\etap{zaawansowana}
\day{?}
\date{??.01.2017}
\RAM{32}
 
\begin{document}
\begin{tasktext}%

W trakcie ostatniego obozu ILOCampowa serwerownia, za sprawą geniuszu kadry, zamieniła się w istną maszynę do robienia pieniędzy (o czym możecie przeczytać w \href{http://www.ilocamp14.ilo.pl/pliki/srw.pdf}{TYM} zadaniu). Po otrzymaniu finansowego raportu kwartalnego z działania serwerowni kadra, podubodowana znakomitymi osiągami ich pomysłu, postanwiła rozbudowac i odnowić serwerownie, aby działała ona jeszcze lepiej i mogła walczyć o miano międzynarodowego potentata branży przechowywania danych.

Serwerownia przeszła gruntowny remont. Z podłogi pozbyto się resztek przypalonych tostów, ztarto wszelkie kurze oraz wprowadzono szereg optymalizacji w działaniu serwerów jak i zmieniono zupełnie strukturę ich połączeń. Korzeniem pozostał serwer numer $1$, ale od teraz serwery i ich połączenia tworzą pełne drzewo binarne o głębokości $n$, w którym traktując $i$-ty serwer jako węzeł drzewa, serwery $2i$ oraz $2i + 1$ są jego kolejno lewym i prawym dzieckiem. 

Wydawało się, że wszystko zmierza w jak najlepszą stronę, kiedy to haker, działający pod intrygującym pseudonimem bOudroX\_skY, przeprowadził atak na ILOCampową serwerownie! Zdołał on złamać zabezpieczenia $k$-tego serwera i umieścił na nim wirusa, który w każdej jednostce czasu roznosi się do sąsiednich serwerów, co skutkuje jego zniszczeniem. Kadra nie pozostała bierna wobec ataku, od razu ruszając z kontrofensywą. W każdej chwili po zarażeniu nowych serwerów może ona zabezpieczyć dowolny inny serwer dzięki czemu wirus na pewno się już na niego nie dostanie. Pomóż kadrze wygrać walkę z czasem i podstępnym hakerem, tym samym ratując ich złoty interes, i znajdź minimalną liczbę serwerów, które zostaną zniszczone pod wpływem wirusa. 

  \section{Wejście}
W pierwszym wierszu znajduje się jedna liczba całkowita $t$ ($1 \leq t \leq 10^6$), oznaczająca liczbę zestawów testowych.

W każdym z kolejnych $t$ wierszy znajdują się dwie liczby całkowite $n$ oraz $k$ ($1 \leq n \leq 60, 1 \leq k \leq 2^n - 1$) oznaczające kolejno głębokość drzewa oraz numer zarażonego wierzchołka. 

  \section{Wyjście}
Na standardowe wyjście należy wypisać $t$ wierszy, a w każdym z nich jedną liczbę całkowitą będącą maksymalną liczbą niezarażonych węzłów w kolejno rozpatrywanych zestawach testowych.

\makecompactexample
\end{tasktext}
\end{document}
