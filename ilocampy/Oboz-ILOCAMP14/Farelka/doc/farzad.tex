\documentclass[zad,zawodnik,utf8]{sinol}
\usepackage{hyperref}

\title{Farelka}
\id{far}
\author{Karol Waszczuk} % Autor zadania
\pagestyle{fancy}
\iomode{stdin}
\konkurs{XIV obóz informatyczny}
\etap{zaawansowana}
\day{2}
\date{17.01.2017}
\RAM{32}
 
\begin{document}
\begin{tasktext}%

Dany jest ciąg $n$ liczb naturalnych $a_1, a_2, \dots, a_n$. Dla każdej liczby z ciągu, chcielibyśmy poznać maksymalną sumę liczb takiego przedziału podanego ciągu, 
że znajduje się w nim ta liczba oraz że największy wspólny dzielnik wszystkich liczb z tego przedziału jest różny od $1$.

  \section{Wejście}

W pierwszym wierszu znajduje się jedna liczba całkowita $n$ ($1 \leq n \leq 10^6$), oznaczająca długość ciągu.

W drugim wierszu znajduje się $n$ liczb całkowitych $a_1, a_2, \dots, a_n$ ($1 \leq a_i \leq 10^6$), oznaczających kolejne liczby ciągu.

  \section{Wyjście}
Na standardowe wyjście należy wypisać $n$ liczb całkowitych oddzielonych pojedynczymi spacjami. $I$-ta z nich powinna być wynikiem dla $i$-tej liczy z ciągu.

\makecompactexample
\end{tasktext}
\end{document}