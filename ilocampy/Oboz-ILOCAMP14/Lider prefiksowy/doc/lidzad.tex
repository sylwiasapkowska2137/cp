\documentclass[zad,zawodnik,utf8]{sinol}
  \title{Lider prefiksowy}
  \id{lid}
  \signature{jtom???}
  \author{Jacek Tomasiewicz} % Autor zadania
  \pagestyle{fancy}
\konkurs{XIV obóz informatyczny}
  \iomode{stdin}
  % HINT: Pola konkurs, etap, day, date uzupelnia kierownik konkursu.
  \etap{początkująca}
  \day{1}
  \date{16.01.2017}
  \RAM{32} % HINT: To pole uzupelnia opracowujacy
 
\begin{document}
  \begin{tasktext}%
	\textit{Liderem} nazywamy element, który występuje więcej niż $\frac{k}{2}$ razy, gdzie $k$ jest liczbą rozpatrywanych elementów.
    \textit{Liderem prefiksowym} nazywamy element, który jest liderem w więcej niż $\frac{n}{2}$ prefiksach rozpatrywanego ciągu, 
    gdzie prefiks to każde $i$ pierwszych elementów ciągu ($1 \leq i \leq n$).
    
    Twoim zadaniem jest znaleźć dla zadanego ciągu wartość lidera prefiksowego.

  \section{Wejście}

	Pierwszy wiersz wejścia zawiera jedną liczbę całkowitą $n$ ($1 \leq n \leq 500\,000$), oznaczającą liczbę elementów ciągu. 
    Drugi wiersz wejścia zawiera $n$ liczb całkowitych $a_0, a_1, \ldots, a_{n-1}$ ($-10^9~\leq~a_i~\leq~10^9$), 
    gdzie $a_i$ oznacza $i$-ty element ciągu.

  \section{Wyjście}

	Pierwszy i jedyny wiersz wyjścia powinien zawierać jedną liczbę całkowitą, 
    równą wartości lidera prefiksowego lub jedno słowo \texttt{NIE}, jeżeli lider prefiksowy nie istnieje.

     \makecompactexample

  \end{tasktext}
\end{document}