\documentclass[zad,zawodnik,utf8]{sinol}

\title{Tester}
\id{tes}
\author{Michał Majewski} % Autor zadania
\pagestyle{fancy}
\iomode{stdin}
\konkurs{XI obóz informatyczny}
\etap{średnia}
\day{?}
\date{??.09.2015}
\RAM{32}
 
\begin{document}
\begin{tasktext}%
Mateusz jest testerem w dużej korporacji. Niedawno dostał od szefa polecenie stworzenia testów do algorytmów na drzewach. Wszystko starannie przygotował i zadowolony poszedł do domu.
Z samego rana czekała na niego niemiła niespodzianka. Okazało się, że testy miały składać się jedynie z drzew binarnych. Nie zdąży już wygenerować testów od nowa, więc chciałby zmodyfikować istniejące poprzez usunięcie niektórych wierzchołków. Nie chciałby tracić zbyt wiele, tak więc usunie jak najmniej wierzchołków.

Dla danego testu wygenerowanego przez Mateusza oblicz, ile wierzchołków będzie musiał on minimalnie usunąć, aby otrzymać drzewo binarne.

  \section{Wejście}
Na wejściu znajduje się opis spójnego drzewa, czyli nieskierowanego grafu bez cykli.

W pierwszym wierszu wejścia znajduje się jedna liczba całkowita $N$ ($1 \leq N \leq 500\,000$) oznaczająca liczbę wierzchołków drzewa.

W kolejnych $N - 1$ wierszach znajdują się pary liczb $A, B$ ($1 \leq A, B \leq N, A \neq B$), oznaczające krawędź między wierzchołkami $A$ i $B$.

  \section{Wyjście}
Na wyjściu powinna znaleźć się jedna liczba całkowita, oznaczająca ile minimalnie wierzchołków należy usunąć z drzewa, aby było ono drzewem binarnym.

\makecompactexample

\end{tasktext}
\end{document}