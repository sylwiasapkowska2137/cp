\documentclass[zad,zawodnik,utf8]{sinol}

\title{Budynki}
\id{bud}
\author{Maciej Hołubowicz} % Autor zadania
\pagestyle{fancy}
\iomode{stdin}
\konkurs{XIV obóz informatyczny}
\etap{olimpijska}
\day{1}
\date{16.01.2017}
\RAM{128}

\usepackage{graphicx}
 
\begin{document}
\begin{tasktext}%

 Na płaszczyźnie wprowadźmy układ współrzędnych kartezjanskich. Wielkie Góry Bajtowe zajmują na tej płaszczyźnie prostoktokąt o przeciwległych wierzchołkach w punktach ($0,0$) i ($w,h$), gdzie $w$ i $h$ są dodatnimi liczbami całkowitymi. W Górach jest n szczytów, a kazdy z nich jest położony w jednym z punktów kratowych prostokąta (punkt kratowy to punkt o całkowitych współrzędnych). Coraz więcej turystów odkrywa piękno Wielkich Gór Bajtowych i kazdy z nich chciałby zbudować sobie w nich budynek. Jednak zgodnie z prawem Narodowego Parku Wielkich Gór Bajtowych, w kazdym punkcie kratowym prostokąta można zbudowac co najwyżej jeden budynek i nie można budować budynków na górskich szczytach. Tak więc jest dokładnie ($w + 1$)~*~($h + 1$)$-n$ miejsc, w których można postawić budynek. 
 
 Niektóre z tych miejsc są uwazane za bardziej atrakcyjne od innych. Powiemy, że punkt ($x,y$) ma północnego sąsiada, jezeli istnieje górski szczyt w pewnym punkcie ($x,y + d$) dla dodatniej liczby całkowitej $d$. Podobnie definiujemy południowego, wschodniego i zachodniego sąsiada. W ten sposób każdy punkt kratowy niebędący górskim szczytem posiada od 0 do 4 sąsiadów. Naturalnie im więcej sąsiadów, tym punkt jest bardziej atrakcyjny (z powodu lepszego widoku na góry). 

Dyrektor Parku chciałby znac maksymalny zysk, jaki mógłby uzyskać ze sprzedaży ziemi pod budowę budynków. Pomóz mu i policz ile punktów kratowych (z wyłączeniem górskich szczytów) posiada $0, 1, 2, 3 i 4$ sąsiadów.

  \section{Wejście}
Pierwszy wiersz wejscia zawiera trzy liczby całkowite $w, h i n$ ($1 \le w,h \le 10^9, 1 \le n \le 5 * 10 ^ 5$), pooddzielane pojedynczymi odstępami. Pozostałe $n$ wierszy opisuje połozenie górskich szczytów w Parku. Każdy z nich zawiera dwie liczby całkowite $x$ i $y$ ($0 \le x \le w, 0 \le y \le h$), oddzielone pojedynczym odstępem. Nie ma dwóch szczytów znajdujących się w tym samym punkcie.

 \section{Wyjście}
Pierwszy i jedyny wiersz wyjscia powinien zawierać 5 liczb całkowitych, pooddzielanych pojedynczymi odstępami i oznaczających liczby punktów kratowych (z wyłączeniem górskich szczytów), mających odpowiednio 0, 1, 2, 3 i 4 sąsiadów.

\makecompactexample

\end{tasktext}
\end{document}
