\documentclass[zad,zawodnik,utf8]{sinol}

\title{Urwipołeć}
\id{urw}
\author{Mateusz Puczel} % Autor zadania
\pagestyle{fancy}
\iomode{stdin}
\konkurs{XIV obóz informatyczny}
\etap{olimpijska}
\day{3}
\date{18.01.2017}
\RAM{128}
 
\begin{document}
\begin{tasktext}%

Przemek to urwipołeć. Dziś po raz kolejny postanowił pożartować sobie z uczestników obozu w trakcie zawodów.

Jakub wysłał dziś na zawodach $n$ zgłoszeń do zadań. Każde zgłoszenie jest zaakceptowane lub odrzucone. Przemek pomyślał, że byłoby śmiesznie,
gdyby pozamieniał rezultaty niektórych zgłoszeń, więc postanowił wypłatać Jakubowi figiel.
Figiel polega na wybraniu dowolnych $k$ zgłoszeń i odwróceniu ich rezultatów -- zgłoszenie zaakceptowane
stanie się odrzucone, a zgłoszenie odrzucone stanie się zaakceptowane. Przemek lubi, gdy Jakub ma dużo zaakceptowanych zgłoszeń, więc
przedsięweźmie swój figiel, jeśli na koniec Jakub będzie miał ich jak najwięcej.

Przemek po wypłataniu swojego figla odczuł ochotę na więcej, ale nie wie dokładnie, ile jeszcze powinien ich wykonać.
Zaczął się zastanawiać, dla każdego $i$ ($1 \leq i \leq n$), jaka jest maksymalna liczba zgłoszeń zaakceptowanych, jaką może uzyskać
Jakub po wykonaniu dokładnie $i$ figlów przez Przemka na początkowym ciągu rezultatów zgłoszeń Jakuba.

  \section{Wejście}
W pierwszym wierszu wejścia znajdują się dwie liczby całkowite $n, k$ ($1 \leq n \leq 500\,000$, $1 \leq k \leq n$), oznaczające odpowiednio
liczbę zgłoszeń Jakuba oraz liczbę zgłoszeń, których rezultaty odwróci Przemek w jednym figlu.

W kolejnym wierszu wejścia znajduje się $n$ liczb całkowitych $a_1, a_2, \cdots, a_n$ ($a_i \in \{0, 1\}$), oznaczających
rezultaty kolejnych zgłoszeń Jakuba ($0$ oznacza zgłoszenie odrzucone, a $1$ zaakceptowane).

  \section{Wyjście}
Na wyjściu powinno znaleźć się $n$ wierszy. $i$-ty z nich powinien zawierać jedną liczbę całkowitą, oznaczającą maksymalną liczbę zaakceptowanych zgłoszeń
po wykonaniu figla (odwrócenie rezultatu dowolnych $k$ zgłoszeń) dokładnie $i$ razy przez Przemka.
  
\makecompactexample

\end{tasktext}
\end{document}