\documentclass[zad,zawodnik,utf8]{sinol}

\title{Przemek rozrabiaka}
\id{pro}
\author{Karol Waszczuk} % Autor zadania
\pagestyle{fancy}
\iomode{stdin}
\konkurs{XIV obóz informatyczny}
\etap{początkująca}
\day{4}
\date{19.01.2017}
\RAM{32}
 
\begin{document}
\begin{tasktext}%

Przemek, znany obozowy rozrabiaka, ma ciężki orzech do zgryzienia. Wczoraj był bardzo niegrzeczny i za karę pani Ilona zadała mu $n$ zadań z matematyki, które musi rozwiązać do końca obozu! Na szczęście wszystkie polegają na tym samym. Mając pewną liczbę $k$ należy stwierdzić czy możliwe jest stworzenie ciągu o 4 elementach będacych liczbami całkowitymi, w którym każdy kolejny różni się od poprzedniego o 1, a ich suma wynosi dokładnie $k$.

Ręczne rozwiązanie wszystkich zadań zajmie Przemkowi cały obóz, przez co nie będzie już miał czasu na figle, psoty i żarty! \textit{Co za hekatomba!} -- pomyslał nasz urwis, momentalnie posępniawaszy. Przywróć uśmiech na twarzy Przemka i pomóż mu z zadaniami! 

  \section{Wejście}

W pierwszym wierszu standardowego wejścia znajduje się jedna liczba całkowita $n$ ($1 \leq n \leq 10\ 000$) oznaczająca liczbę zadań, które Przemek otrzymał za karę.
 
Każdy z kolejnych $n$ wierszy zawiera jedną liczbę całkowitą $k$ z treści zadania. ($1 \leq k \leq 10^{18}$).

  \section{Wyjście}
Na standardowe wyjście należy wypisać $n$ wierszy, będącymi odpowiedziami na kolejne zapytania. Jeśli liczba $k$ umożliwia stworzenie ciągu opisanego w treści zadania wypisz \texttt{TAK}, w przeciwnym przypadku wypisz \texttt{NIE}.
\makecompactexample 
\hspace{1cm} \textbf{\newline Wyjaśnienie do przykładu:} Ciąg o sumie 2 wygląda następująco: $[-1, 0, 1, 2]$.

\end{tasktext}
\end{document}
