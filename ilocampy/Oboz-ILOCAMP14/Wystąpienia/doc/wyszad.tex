\documentclass[zad,zawodnik,utf8]{sinol}

\title{Wystąpienia}
\id{wys}
\author{Mateusz Puczel} % Autor zadania
\pagestyle{fancy}
\iomode{stdin}
\konkurs{XIV obóz informatyczny}
\etap{początkująca}
\day{3}
\date{18.01.2017}
\RAM{128}

\begin{document}
\begin{tasktext}%

Przemek napisał na kartce $n$ liczb w ciągu. Chciałby teraz wskazać liczbę, która występuje w tym ciągu najczęściej. Zanim jednak zacznie jej szukać,
może on wykonać nie więcej niż $k$ razy operację zwiększenia dowolnej z liczb o 1. Chciałby teraz wykonać te operacje w taki sposób,
aby liczba wystąpień najczęściej występujacej liczby była jak największa.

  \section{Wejście}
W pierwszym wierszu wejścia znajdują się dwie liczby całkowite $n, k$ ($1 \leq n \leq 5 \cdot 10^5$, $0 \leq k \leq 10^9$), oznaczające odpowiednio
liczbę liczb na kartce oraz liczbę dozwolonych operacji.

W kolejnym wierszu wejścia znajduje się $n$ liczb całkowitych $a_1, a_2, \cdots, a_n$ ($0 \leq a_i \leq 10^9$), oznaczające kolejne liczby na kartce.

  \section{Wyjście}
Na wyjściu powinny znaleźć się dwie liczby oddzielone spacją -- pierwsza oznacza liczbę wystąpień najczęściej występującego elementu po wykonaniu co najwyżej $k$ operacji,
a druga najmniejszą spośród liczb występujących tyle razy.

\makecompactexample

\medskip
\noindent
\textbf{Wyjaśnienie do przykładu:} Jeżeli zwiększymy liczbę na pozycji drugiej i piątej, to otrzymamy czterokrotne wystąpienie liczby $2$.
Zauważ, że można również zwiększyć dwukrotnie liczbę na drugiej pozycji oraz jednokrotnie liczby na pozycjach 3 i 4. Wówczas otrzymamy również
czterokrotne wystąpienie liczby $3$, jednak szukamy najmniejszej możliwej liczby.

\end{tasktext}
\end{document}