\documentclass[zad,zawodnik,utf8]{sinol}

\title{Telefony}
\id{tel}
\author{Karol Waszczuk} % Autor zadania
\pagestyle{fancy}
\iomode{stdin}
\konkurs{XIV obóz informatyczny}
\etap{początkująca}
\day{1}
\date{16.01.2017}
\RAM{32}
 
\begin{document}
\begin{tasktext}%

Zjednoczone Królestwo Bajtocji i Bitocji ma nowego władcę! Po śmierci wcześniejszego króla jego miejsce zajął jego rodzony brat -- Przemek, który od teraz jest znany jako Jakub Silny. Jedną z pierwszych reform nowego króla jest wprowadzenie telefonów stacjonarnych w całym królestwie. Jednak zanim każdy z mieszkańców państwa będzie mógł wyposażyć się w swój własny telefon, należy ustalić bardzo ważną kwestię, a mianowicie numery telefonów służb ratunkowych.

Jakub postanowił, że w celu łatwiejszego zapamiętania tych jakże ważnych numerów, każdy z nich będzie składał się z trzech cyfr, których suma wynosić będzie $n$. Ponadto po wielu godzinach rozmów z policjantami, lekarzami i strażakami, przychylił się do ich próśb i zarządził, iż numer telefonu policji będzie parzysty, zaś pogotowia nieparzysty. Król Jakub zastanawia się na ile sposobów można wybrać takie trzy numery telefonów, aby spełniały one wszystkie wspomniane wcześniej wymagania.

  \section{Wejście}
W pierwszym wierszu i jedynym wierszu wejścia znajduje się jedna liczba $n$ ($0 \leq n \leq 1\ 000$) z treści zadania.

  \section{Wyjście}
Na standardowe wyjście należy wypisać jedną liczbę całkowitą równą liczbie sposobów na które można wybrać trzy numery telefonów spełniające warunki z treści zadania.

\makecompactexample
\end{tasktext}
\end{document}
