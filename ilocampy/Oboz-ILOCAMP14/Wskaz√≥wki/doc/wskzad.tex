\documentclass[zad,zawodnik]{sinol}
\usepackage[utf8x]{inputenc}
  \title{Wskazówki}
  \author{Jacek Tomasiewicz}
  \pagestyle{fancy}
  \signature{jtom???}
  \id{wsk}
  \iomode{stdin}
  \etap{zaawansowana}
  \day{1}
  \date{16.01.2017}
  \RAM{64}
  \history{2011.02.20}{JTom, pomysł i~redakcja}{1.00}

\begin{document}
  \begin{tasktext}% Ten znak % jest istotny!
Jesteśmy w Bajtockim muzeum, gdzie znajduje się $n$ bardzo starych, okrągłych zegarów. Każdy z nich jest tej samej wielkości i posiada $m$ identycznych wskazówek. 

Na zegarach, ze starości nie widać podziałki z godzinami, która pierwotnie była w zegarach. Podziałka oznacza liczbę miejsc (w równych odstępach), na które mogą wskazywać wskazówki zegara. Jedyne co różni zegary to ułożenie wskazówek. 

W wyniku obrotu zegarów możemy doprowadzić do sytuacji, że dwa zegary będą wyglądały dokładnie tak samo. Chcielibyśmy wiedzieć, ile jest par zegarów, które w wyniku obrotu mogą wyglądać tak samo.
 
  \section{Wejście}
Pierwszy wiersz wejścia zawiera trzy liczby całkowite $n, m, p$ ($1 \leq n, m \leq 1\,000, 1 \leq p \leq 10^9$), oznaczające odpowiednio liczbę zegarów, liczbę wskazówek oraz wielkość podziałki zegarów. 

Następnych $n$ wierszy opisuje kolejne zegary. Każdy wiersz zawiera $m$ liczb całkowitych $w_1, w_2, \ldots, w_m$ ($1 \leq w_i \leq p$), gdzie $w_i$ oznacza miejsce znajdowania się $i$-tej wskazówki.

Możesz założyć, że w testach wartych $50\%$ punktów zachodzi $n, m \leq 300$.

  \section{Wyjście}
Pierwszy i jedyny wiersz wyjścia powinien zawierać jedną liczbę całkowitą, równą liczbie par zegarów, które w wyniku obrotów mogą wyglądać tak samo.

	\exampleimg{wsk0.eps}
     \makestandardexample

\bigskip
\noindent
\textbf{Wyjaśnienie do przykładu:} 
Pary zagarów, które mogą wyglądać tak samo to: $(1,3), (1,4), (2,5), (3,4)$.

  \end{tasktext}
\end{document}