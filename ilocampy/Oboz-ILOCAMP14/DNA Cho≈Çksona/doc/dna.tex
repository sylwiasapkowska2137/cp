\documentclass[zad,zawodnik,utf8]{sinol}

\title{DNA Chołksona}
\id{dna}
\author{Mateusz Chołołowicz} % Autor zadania
\pagestyle{fancy}
\iomode{stdin}
\konkurs{XIV obóz informatyczny}
\etap{zaawansowana}
\day{2}
\date{17.01.2017}
\RAM{64}

\usepackage{enumitem}
 
\begin{document}
\begin{tasktext}%
Szalony Chołkson tym razem postanowił pozabijać humanistów przy użyciu zainfekowanego DNA. Humaniści nie są skomplikowanymi istotami,
dlatego łańcuchy cząstek ich DNA składają się jedynie z dwóch zasad: 

\begin{itemize}
\item zasady W - reprezentującą Wenę Twórczą oraz
\item zasady G - reprezentującą Głupotę.
\end{itemize}

Chołkson tworząc nowe DNA zawsze zaczyna od jednej zasady typu W. Następnie jego apokaliptyczna maszyna co sekundę podwaja długość łańcucha.
Dokładniej, każdą zasadę typu W zastępuje parą zasad GW, a każdą zasadę typu G zastępuje parą zasad WG. Tak więc po pierwszej sekundzie
łańcuch DNA będzie wyglądał tak: GW, a po drugiej następująco: WGGW.

Proces infekowania DNA przez Chołksona następuje dopiero po zakończeniu pracy jego maszyny. Chołkson zastanawia się jednak, ile sekund powinna
działać jego maszyna, aby zwrócić DNA najlepiej nadające się do tuningu. DNA zdaniem szaleńca jest tym lepsze, im więcej istnieje w nim
sąsiadujących ze sobą zasad reprezentujących Głupotę modulo $10^9+7$. Musisz uwierzyć mu na słowo.

Chołkson nadal zmaga się z po-analizowym zboczeniem, dlatego to Ty musisz stwierdzić, na ile hekatombiczne DNA powstanie po $t$ sekundach pracy jego maszyny.

  \section{Wejście}
W pierwszym wierszu wejścia znajduje się jedna liczba całkowita $t$ ($0 \leq t \leq 10^6$) oznaczająca liczbę sekund przez które będzie pracować maszyna
Chołksona.

 \section{Wyjście}
Na standardowe wyjście należy wypisać jedną liczbę całkowitą równą liczbie sąsiadujących ze sobą zasad typu G w łańcuchu DNA
otrzymanym po $t$ sekundach pracy maszyny modulo $10^9+7$.

\makecompactexample

\end{tasktext}
\end{document}