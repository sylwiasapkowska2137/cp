\documentclass[zad,zawodnik,utf8]{sinol}

\title{Maksymalny zbiór k-pokrywający}
\id{mak}
\author{Mateusz Radecki} % Autor zadania
\pagestyle{fancy}
\iomode{stdin}
\konkurs{XIV obóz informatyczny}
\etap{ŻarzMasters}
\day{4}
\date{19.01.2017}
\RAM{256}

\usepackage{indentfirst}

\begin{document}
\begin{tasktext}%
    
	\textbf{Uwaga! To zadanie jest inne niż zwykłe zadania. Jest to zadanie aproksymacyjne. Nie musisz znaleźć optymalnego wyniku. Zaakceptowane zostanie każde rozwiązane, podające jakiś wynik w poprawnym formacie. Im lepszy podasz wynik, tym więcej otrzymasz punktów. Jest to pierwsza taka inicjatywa na obozie i wersja alfa, więc bądźcie wyrozumiali :)}
\vskip 3mm
    Masz dane drzewo z wagami na krawędziach. Chcesz wybrać podzbiór $k$ wierzchołków, żeby zmaksymalizować sumę wag krawędzi incydentnych do przynajmniej jednego wierzchołka.


  \section{Wejście}

   W pierwszej linii wejścia znajdują się dwie liczby całkowite $n$ i $k$ ($1 \leq k \leq n \leq 100000;~ 30000 \leq n$) oznaczające rozmiar drzewa i rozmiar podzbioru wierzchołków, który masz wybrać. W następnych $n-1$ wierszach znajdują się opisy krawędzi w postaci $a~b~x$ ($1 \leq a,b \leq n;~ a \neq b;~ 1 \leq x \leq 10^6$), co oznacza, że istnieje krawędź między wierzchołkami $a$ i $b$ o wadze $x$.

  \section{Wyjście}
   
\vskip 1mm
   W pierwszym wierszu wyjścia powinna znaleźć się jedna liczba całkowita oznaczająca sumę wag krawędzi incydentnych do przynajmniej jednego wybranego przez ciebie wierzchołka.
\vskip 1mm
   W drugim wierszu powinno znaleźć się dokładnie $k$ różnych liczb, oznaczających indeksy wybranych wirzchołków.

  \section{Ocenianie}
   
\vskip 1mm
   Jeśli wypiszesz wynik w niepoprawnym formacie, otrzymasz $0$ punktów za dany test.
\vskip 1mm
   Jeśli format będzie poprawny, to jeśli $A$ to twój wynik, a $B$ to wynik jury, to otrzymasz $10*(\frac{A}{B})^T$ punktów, gdzie $T$ i $B$ mogą się zmieniać w trakcie konkursu.
\vskip 1mm
   Testy nie będą grupowane i będą generowane następującym algorytmem:
   Ręcznie wybierane są parametry $n$, $k$, $d$ oraz przedział z którego losowane są wagi krawędzi. Następnie dla każdego $i$ od $2$ do $n$ dodawana jest krawędź od $i$ do losowego wierzchołka z zakresu od $max(1, i-d)$ do $i-1$.

  
\makecompactexample

\end{tasktext}
\end{document}
