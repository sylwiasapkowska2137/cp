\documentclass[zad,zawodnik,utf8]{sinol}

\title{Inspekcja}
\id{ins}
\author{Mateusz Radecki} % Autor zadania
\pagestyle{fancy}
\iomode{stdin}
\konkurs{XIV obóz informatyczny}
\etap{olimpijska}
\day{4}
\date{19.01.2017}
\RAM{256}
 
\begin{document}
\begin{tasktext}%
    
    W Przemkolandii trzeba przeprowadzić inspekcję dróg. Przemek, znany inspektor, musi dokonać jej osobiście.
Przemkolandia składa się z miast połączonych dwukierunkowymi drogami. Z każdego miasta do każdego innego da się dojechać na dokładnie jeden sposób. Innymi słowy Przemkolandia tworzy drzewo.
\vskip 1mm
   Przemek podczas swojej pracy wykonuje wiele "wypraw". Wyprawa to podróż, która zaczyna się w jakimś mieście, kończy w jakimś innym mieście, korzysta jedynie z dróg i nie przebiega dwa razy przez to samo miasto.
\vskip 1mm
Podczas wyprawy Przemek dokonuje po jednej inspekcji na wszystkich drogach pomiędzy miastem startowym i końcowym. Każda droga ma pewien priorytet, mówiący ile razy należy dokonać na niej inspekcji. Jeśli Przemek nie przeprowadzi inspekcji jakiejś drogi tyle razy ile trzeba, to zapewne zostanie mu to potrącone z wypłaty. Jeśli zaś na jakiejś drodze dokona zbyt wielu inspekcji, to jego przełożeni zaczną podejrzewać, że Przemek coś kombinuje i straci on swoje szanse na awans. Niestety, przejeżdżając przez jakąś drogę Przemek nie potrafi powstrzymać się od przeprowadzenia inspekcji.
\vskip 1mm
Pomiędzy wyprawami Przemek może swobodnie podróżować samolotami, tzn. bez korzystania z dróg. Powiedz na ile conajmniej wypraw musi wyruszyć Przemek.
\vskip 1mm
Co więcej, priorytety dróg cały czas się zmieniają. Po każdej zmianie Przemek chciałby znać odpowiedź dla aktualnego zestawu priorytetów.


  \section{Wejście}

  W pierwszej linii wejścia znajduje się liczba całkowita $n$ ($1 \leq n \leq 500000$). W następnych $n-1$ wierszach znajdują się opisy dróg w postaci $a~b~x$ ($1 \leq a,b \leq n;~ a \neq b;~ 0 \leq x \leq 10^9$), która oznacza, że miasto $a$ jest połączone z miastem $b$ drogą o priorytecie wynoszącym aktualnie $x$.
\vskip 1mm
  W kolejnej linii znajduje się jedna liczba całkowita $q$ ($1 \leq q \leq 500000$) oznaczająca liczbę zmian. W następnych $q$ liniach znajdują opisy zmian w postaci $a~b~x$ ($1 \leq a,b \leq n;~ a \neq b;~ 0 \leq x \leq 10^9$), która oznacza, że od teraz droga łącząca miasta $a$ i $b$ ma priorytet $x$.

  \section{Wyjście}
   
   Wypisz na wyjście $q+1$ linii, pierwsza z nich powinna być odpowiedzią na pytanie Przemka przed jakimikolwiek zmianami, zaś $q$ następnych odpowiedziami na pytania po zmianach priorytetów.
  
\makecompactexample

\end{tasktext}
\end{document}
