\documentclass[zad,zawodnik,utf8]{sinol}

\title{Podciąg Przekuba}
\id{pod}
\author{Mateusz Radecki} % Autor zadania
\pagestyle{fancy}
\iomode{stdin}
\konkurs{XIV obóz informatyczny}
\etap{olimpijska}
\day{2}
\date{17.01.2017}
\RAM{512}
 
\begin{document}
\begin{tasktext}%
    
    Przemek i Jakub są bliźniakami, którzy nie są w stanie ogarnąć liczb większych od $k$. Każdy z nich podczas dzisiejszego śniadania, zgodnie z zaleceniem lekarza,
    napisał na kartce pewien ciąg liczb naturalnych, aby utrwalić pisownię liczb, które znają. Ich starszy brat Przekub jest nieugiętym rozkminiaczem.
    Pod nieuwagę bliźniaków zabrał im kartki i teraz rozkminia, jaki jest najkrótszy ciąg liczb taki, że nie jest on podciągiem 
    żadnego z ciągów zapisanych na kartkach oraz wszystkie liczby, z których się składa, są znane przez bliźniaków.
    
    Niefortunnie Przekub nie umie tego rozkminić, zatem musisz zrobić to Ty.

  \section{Wejście}

  W pierwszej linii wejścia znajduje się jedna liczba całkowita $k$ ($1 \leq k \leq 2000$), oznaczająca największą liczbę naturalną, którą znają chłopcy. W kolejnym wierszu znajduje się liczba
  $n$ ($1 \leq n \leq 2000$), oznaczająca długość ciągu $a_1, a_2, \dots, a_n$ ($1 \leq a_i \leq k$) zapisanego na kartce przez Przemka. Ciąg ten znajduje się w następnym wierszu wejścia. 
  Czwarty wiersz zawiera liczbę całkowitą $m$ ($1 \leq m \leq 2000$), równą długości ciągu $b_1, b_2, \dots, b_m$ ($1 \leq b_i \leq k$) zapisanego przez Jakuba, podanego w piątym
  wierszu wejścia.

  \section{Wyjście}
   
   W pierwszym wierszu standardowego wyjścia powinna znaleźć się jedna liczba całkowita $l$ równa długości ciągu, który wykminiłeś za Przekuba. 
   W drugim wierszu wyjścia powinieneś wypisać przykładowy ciąg o tej długości, który spełnia wymogi zadania.
  
\makecompactexample

\end{tasktext}
\end{document}
