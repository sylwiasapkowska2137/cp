\documentclass[zad,zawodnik,utf8]{sinol}

\title{Plansza}
\id{pla}
\author{Mateusz Puczel} % Autor zadania
\pagestyle{fancy}
\iomode{stdin}
\konkurs{XIV obóz informatyczny}
\etap{początkująca}
\day{1}
\date{16.01.2017}
\RAM{32}
 
\begin{document}
\begin{tasktext}%
Na nieskończonej planszy, opisanej przez układ współrzędnych kartezjańskich, na polu o współrzędnych $(0,~0)$ stoi pionek. Pionek można w jednym ruchu przesunąć o jedno pole w jednym z czterech kierunków: góra, dół, lewo lub prawo.
Przemek chce przesunąć pionek na pole o współrzędnych $(a,~b)$ wykonując jedynie dozwolone ruchy. Czy jest możliwe, aby zrobił to wykonując dokładnie $m$ ruchów?

Zakładamy, że pierwsza współrzędna opisuje położenie w poziomie (ruch w prawo ją zwiększa o 1, a w lewo zmniejsza o 1),
a druga współrzędna opisuje położenie w pionie (ruch w górę zwiększa o 1, ruch w dół zmniejsza o 1).

  \section{Wejście}
Na wejściu znajdują się trzy liczby całkowite $a$, $b$, $m$ ($-10^9 \leq a, b \leq 10^9$, $0 \leq m \leq 2 \cdot 10^9$), oznaczające odpowiednio współrzędne
końcowe pionka oraz liczbę ruchów, które należy wykonać, aby się tam dostać z punktu $(0,~0)$.

  \section{Wyjście}
Jeżeli Przemek może przesunąć pionek z pola (0, 0) na pole $(a,~b)$ wykonując dokładnie $m$ ruchów, należy na standardowe wyjście wypisać \texttt{TAK}.
W przeciwnym wypadku należy wypisać \texttt{NIE}.
  
\section{Przykład}
  \twocol{%
      \noindent Dla danych wejściowych:
      \includefile{../in/\ID0a.in}
    }{%
      \noindent poprawnym wynikiem jest:
      \includefile{../out/\ID0a.out}
    }
  \twocol{%
      \noindent natomiast dla danych wejściowych:
      \includefile{../in/\ID0b.in}
    }{%
      \noindent poprawnym wynikiem jest:
      \includefile{../out/\ID0b.out}
    }

\end{tasktext}
\end{document}