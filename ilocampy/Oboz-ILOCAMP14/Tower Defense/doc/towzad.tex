\documentclass[zad,zawodnik,utf8]{sinol}

\title{Tower Defense}
\id{tow}
\author{Mateusz Chołołowicz} % Autor zadania
\pagestyle{fancy}
\iomode{stdin}
\konkurs{XIV obóz informatyczny}
\etap{początkująca}
\day{2}
\date{17.01.2017}
\RAM{64}
 
\begin{document}
\begin{tasktext}%
Król Karol (ten z pokoju 303) w ramach rozwijania swoich umiejętności stategicznych, często gra w grę Tower Defense. 
Polega ona na ustawieniu $k$ wież na prostokątnej planszy o $n$ wierszach i $m$ kolumnach w taki sposób, aby jak najlepiej chronić
królestwo, które reprezentuje plansza. Dokładniej, gra polega na zminimalizowaniu powierzchni największego obszaru niechronionego przez żadną z wież.
W każdej kolumnie i w każdym wierszu można postawić co najwyżej jedną wieżę - wówczas wieża ta chroni wszystkie pola w wierszu 
i kolumnie w których się znajduje. 

Król Karol rozstawił wszystkie $k$ wież na planszy zgodnie z zasadami gry i zastanawia się jakie jest pole największego prostokątnego obszaru, takiego
że każde z jego pól nie jest chronione przez którąkolwiek z wież. Pomóż mu odpowiedzieć na to pytanie!

  \section{Wejście}
W pierwszym wierszu wejścia znajdują się trzy liczby całkowite $n$, $m$ i $k$ ($1 \leq n, m \leq 10^9, 1 \leq k \leq 10^6$, $k \leq min(n, m)$), 
oznaczające odpowiednio wymiary planszy oraz liczbę wież, ktore ustawił król Karol. W następnych $k$ wierszach znajdują się opisy
kolejnych wież, składające się z dwóch liczb całkowitych $y_i$, $x_i$ ($1 \leq y_i \leq n$, $1 \leq x_i \leq m$), które oznaczają odpowiednio
numer wiersza i kolumny pól, na których stoją kolejne wieże na planszy.
 \section{Wyjście}
Na standardowe wyjście należy wypisać jedną liczbę całkowitą równą polu największego obszaru niechronionego przez wieże rozstawione
przez króla Karola.

\makecompactexample

\end{tasktext}
\end{document}