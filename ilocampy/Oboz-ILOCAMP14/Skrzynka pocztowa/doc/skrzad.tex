\documentclass[zad,zawodnik,utf8]{sinol}

\title{Skrzynka pocztowa}
\id{skr}
\author{Mateusz Puczel} % Autor zadania
\pagestyle{fancy}
\iomode{stdin}
\konkurs{XIV obóz informatyczny}
\etap{początkująca}
\day{2}
\date{17.01.2017}
\RAM{128}
 
\begin{document}
\begin{tasktext}%

Przemek codziennie przegląda swoją pocztę elektroniczną. Od jakiegoś czasu zajmuje mu to bardzo dużo czasu, więc postanowił tym razem przyspieszyć ten proces.

W skrzynce odbiorczej Przemka znajduje się $n$ e-maili. Niektóre z nich są już przeczytane, a niektóre wciąż nie.
Program do przegladania poczty składa się z dwóch widoków: widok z listą wszystkich e-maili uporządkowanych po czasie dostarczenia oraz
widok szczegółowy dla danego e-maila. Po wejściu w widok szczegółowy dany e-mail staje się przeczytany i tylko tam można go przeczytać.

Z widoku listy e-maili można za pomocą jednego kliknięcia przejść do widoku szczegółowego dowolnego e-maila. Z widoku szczegółowego można za pomocą jednego kliknięcia
wrócić do listy e-maili, przejść do e-maila następnego lub poprzedniego (w sensie czasu dostarczenia).

Przemek chciałby przeczytać wszystkie dotychczas nieprzeczytane e-maile. Chciałby zrobić to za pomocą minimalnej liczby kliknięć, aby maksymalnie skrócić
czas przeglądania poczty. Napisz program, który mu to ułatwi.

  \section{Wejście}
W pierwszym wierszu wejścia znajduje się jedna liczba całkowita $n$ ($1 \leq n \leq 10^6$), oznaczająca liczbę e-maili w skrzynce Przemka.
W kolejnym wierszu wejścia znajduje się opis wszystkich e-maili w kolejności od najwcześniej dostarczonego do najpóźniej dostarczonego.
Dokładniej, znajduje się w nim $n$ liczb całkowitych $a_1, a_2, \cdots, a_n$ ($0 \leq a_i \leq 1$), oznaczające stan kolejnych e-maili.
Jeśli $a_i = 1$, to $i$-ty e-mail nie został przeczytany, a jeśli $a_i = 0$, to e-mail został przeczytany.

  \section{Wyjście}
Na wyjściu powinna znaleźć się jedna liczba całkowita, oznaczająca minimalną liczbę kliknięć potrzebną do przeczytania wszystkich nieprzeczytanych e-maili.
  
\makecompactexample

\end{tasktext}
\end{document}