\documentclass[zad,zawodnik,utf8]{sinol}

  \title{Sprawdzian}
  \author{Jacek Tomasiewicz}
  \pagestyle{fancy}
  \signature{jtom???}
  \konkurs{XIII obóz informatyczny}
  \id{spr}
  \iomode{stdin}
  \etap{początkująca}
  \day{1}
  \date{26.09.2016}
  \RAM{32}

\begin{document}
  \begin{tasktext}% Ten znak % jest istotny!
Ania dostała na sprawdzianie bardzo proste zadanie: ile liczb z przedziału od $a$ do $b$ jest  podzielnych przez $k$? Pomożesz Ani?

  \section{Wejście}
Pierwszy i jedyny wiersz wejścia zawiera trzy liczby całkowite $a, b, k$ ($1 \leq a \leq b \leq 2 \cdot 10^9, 1  \leq k \leq 2 \cdot 10^9$), oznaczające odpowiednio początek i koniec przedziału oraz liczbę przez  którą dzielimy.

  \section{Wyjście}
Wyjście powinno zawierać jedną liczbę całkowitą, równą ilości liczb, które się dzielą przez $k$.

     \makecompactexample

  \end{tasktext}
\end{document}