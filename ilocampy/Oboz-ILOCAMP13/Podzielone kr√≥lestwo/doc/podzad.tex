\documentclass[zad,zawodnik,utf8]{sinol}

\title{Podzielone królestwo}
\id{pod}
\author{Mateusz Puczel} % Autor zadania
\pagestyle{fancy}
\iomode{stdin}
\konkurs{XIII obóz informatyczny}
\etap{zaawansowana}
\day{1}
\date{26.09.2016}
\RAM{128}

\begin{document}
\begin{tasktext}%
Pewien król miał królestwo, w którym było $2n$ miast oraz dwóch synów -- Przemka i Jakuba. Król w swoim testamencie przekazał po $n$ miast każdemu z nich.

Pewnego razu nastąpił podział królestwa na dwie części. Pierwszą część stanowiły miasta przepisane Przemkowi, a drugą miasta przepisane
Jakubowi. Zostały też zniszczone wszystkie bezpośrednie połączenia pomiędzy miastami z dwóch różnych części.

Nikt właściwie nie wie, dlaczego królestwo zostało podzielone, ponieważ Przemek i Jakub bardzo się lubią i
mają ze sobą wiele wspólnego. Właściwie to niczym
się nie różnią. Dlatego bracia postanowili odbudować i połączyć oba królestwa.

Królestwo Przemka składa się z $n$ miast ponumerowanych liczbami naturalnymi od $1$ do $n$,
połączonych pewną liczbą jednokierunkowych dróg. Analogicznie królestwo Jakuba.
Przemek mieszka w mieście o numerze $p$ w swoim królestwie, a Jakub w mieście o numerze $j$ w swoim.

Bracia chcą odbudować swoje królestwo w taki sposób, aby Przemek mógł dostać się do mieszkania Jakuba.
Jednocześnie chcieliby przeprowadzić odbudowę jak najmniejszym kosztem.
Niestety infrastruktura królestwa pozwala jedynie na zbudowanie \textbf{dwukierunkowego} mostu pomiędzy $i$-tym miastem w królestwie Przemka,
a $i$-tym miastem w królestwie Jakuba.
Innymi słowy, można wybudować most z dowolnego miasta do miasta w drugim królestwie o takim samym numerze.
Koszt wybudowania mostu wynosi~$1$.

Pracujesz w zarządzie dróg królewskich i twoim zadaniem jest wyznaczyć minimalny koszt odbudowy królestwa według ustaleń braci.

  \section{Wejście}
W pierwszym wierszu wejścia znajdują się trzy liczby całkowite $n$, $p$, $j$ ($1 \leq p, j \leq n \leq 500\,000$), oznaczające
odpowiednio liczbę miast w obu królestwach oraz numery miast, w których mieszkają Przemek i Jakub.

W drugim wierszu wejścia znajduje się jedna liczba całkowita $m_p$ ($0 \leq m_p \leq 500\,000$), oznaczająca liczbę jednokierunkowych
dróg w królestwie Przemka.

W każdym z kolejnych $m_p$ wierszy znajdują się dwie liczby całkowite $a$, $b$ ($1 \leq a, b \leq n$, $a \neq b$), oznaczające,
że w królestwie Przemka z miasta $a$ prowadzi jednokierunkowa droga do miasta $b$.

W kolejnym wierszu wejścia znajduje się jedna liczba całkowita $m_j$ ($0 \leq m_j \leq 500\,000$), oznaczająca liczbę jednokierunkowych
dróg w królestwie Jakuba.

W każdym z kolejnych $m_j$ wierszy znajdują się dwie liczby całkowite $a$, $b$ ($1 \leq a, b \leq n$, $a \neq b$), oznaczające,
że w królestwie Jakuba z miasta $a$ prowadzi jednokierunkowa droga do miasta $b$.

  \section{Wyjście}
Na wyjściu powinna znaleźć się jedna liczba całkowita, oznaczająca minimalny koszt odbudowy królestwa.
Jeżeli odbudowa królestwa nie jest możliwa, wypisz \texttt{NIE}.

\makecompactexample

\end{tasktext}
\end{document}