\documentclass[zad,zawodnik,utf8]{sinol}

\title{Klocki}
\id{klo}
\author{Jacek Tomasiewicz} % Autor zadania
\pagestyle{fancy}
\iomode{stdin}
\konkurs{XIII obóz informatyczny}
\etap{początkująca}
\day{2}
\date{27.09.2016}
\RAM{32}
 
\begin{document}
\begin{tasktext}%
Adrian buduje wieżyczki z klocków. Ułożył już $n$ wieżyczek, każdą z pewnej liczby  klocków. Po pewnym czasie Adrian stwierdził, że wybudowane przez niego wieżyczki  wyglądają po prostu brzydko.

Adrian chciałby, aby każda wieżyczka zbudowana była z tej samej liczby klocków.  Postanowił, że może brać klocki z czubków wieżyczek i przekładać na inne wieżyczki. Nie może natomiast zmniejszyć łącznej liczby klocków budowli oraz nie może zmniejszyć liczby wieżyczek (wieżyczka musi być zbudowana z co najmniej jednego klocka).

Pomoż Adrianowi i powiedz, ile minimalnie klocków musi przestawić, aby wszystkie wieżyczki składały się z takiej samej liczy klocków lub stwierdź, że nie jest to  możliwe.

  \section{Wejście}
Pierwszy wiersz wejścia zawiera jedną liczbę całkowitą $n$ ($1 \leq n \leq 400\,000$),  oznaczającą liczbę wieżyczek. Kolejny wiersz zawiera $n$ liczb całkowitych $w_1, w_2,  \ldots, w_n$ ($1 \leq w_i \leq 10^6$), gdzie $w_i$ oznacza liczbę klocków, z których  zbudowana jest $i$-ta wieżyczka.

  \section{Wyjście}
Pierwszy i jedyny wiersz wyjścia powinien zawierać jedną liczbę całowitą, równą  minimalnej liczbie klocków, które Adrian musi przestawić lub \texttt{NIE}, gdy nie jest  możliwe uzyskanie równych wieżyczek.

\makecompactexample
  
  \section{Wyjaśnienie do przykładu}
Adrian może przestawić z drugiej wieżyczki 2 klocki na  pierwszą wieżyczkę oraz z drugiej wieżyczki 1 klocek na czwartą wieżyczkę.

\end{tasktext}
\end{document}