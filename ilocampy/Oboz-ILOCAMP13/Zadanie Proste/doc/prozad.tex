\documentclass[zad,zawodnik,utf8]{sinol}

\title{Zadanie Proste}
\id{pro}
\author{Maciej Hołubowicz} % Autor zadania
\pagestyle{fancy}
\iomode{stdin}
\konkurs{XIII obóz informatyczny}
\etap{olimpijska}
\day{1}
\date{26.09.2016}
\RAM{128}
 
\begin{document}
\begin{tasktext}%
Na płaszczyźnie znajduje się $n$ punktów. Chcemy znaleźć dokładnie $k$ prostych, takich że każdy punkt leży na którejś z nich.

  \section{Wejście}
W pierwszym wierszu wejścia znajdują się dwie liczby całkowite $n, k$ ($1 \leq n \leq 200\,000, 1 \leq k \leq 5$), oznaczające odpowiednio liczbę punktów na płaszczyźnie
oraz liczbę poszukiwanych prostych.

W każdym z kolejnych $n$ wierszy znajdują się dwie liczby całkowite $x, y$ ($-10^9 \leq x, y \leq 10^9$), oznaczające współrzędne kolejnych punktów.
  
  \section{Wyjście}
W pierwszym wierszu wyjścia powinien znaleźć się napis \texttt{TAK}, jeżeli da się znaleźć $k$ prostych na płaszczyźnie takich, że każdy punkt leży na którejś z nich
lub \texttt{NIE}, jeżeli to niemożliwe.

Jeżeli da się ułożyć $k$ prostych, to w każdym z kolejnych $k$ wierszy należy wypisać 4 liczby całkowite $x_0, y_0, x_1, y_1$ ($-10^9 \leq x_0, y_0, x_1, y_1 \leq 10^9$),
oznaczające współrzędne dowolnych dwóch \textbf{różnych} punktów, przez które przechodzi kolejna prosta.

\makecompactexample

  \section{Wyjaśnienie do przykładu}
Pierwsze dwa punkty o współrzędnych (-1, 0) oraz (1, 0) reprezentują prostą przechodzącą przez oś $OX$, która pokrywa pierwszy i drugi punkt z wejścia.
Kolejne dwa punkty o współrzędnych (0, -1) oraz (0, 1) reprezentują prostą przechodzącą przez oś $OY$, która pokrywa drugi i trzeci punkt z wejścia, zatem
wszystkie punkty są pokryte którąś prostą.

\end{tasktext}
\end{document}