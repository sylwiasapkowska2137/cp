\documentclass[zad,zawodnik,utf8]{sinol}

\title{Magik}
\id{mag}
\author{Maciej Hołubowicz} % Autor zadania
\pagestyle{fancy}
\iomode{stdin}
\konkurs{XIII obóz informatyczny}
\etap{początkująca}
\day{4}
\date{29.09.2016}
\RAM{64}
 
\begin{document}
  \begin{tasktext}% Ten znak % jest istotny!
  Franek jest początkującym magikiem, ma $a$ amarantowych $b$ burgundowych i $c$ karmazynowych piłeczek. Może zamienić dwie piłeczki takiego samego koloru na jedną piłeczke innego koloru. Po swojej magiczniej sztuczce chciałby mieć prznajmniej $x$ amarantowych, $y$ burgundowych i $z$ karmazynowych piłeczek. Czy jest w stanie to zrobić?  
  
 \section{Wejście}
    
  W pierwszym wierszu wejścia są trzy liczby $a$, $b$ i $c$ ($0 \leq a, b, c \leq 10^6$) oznaczające ilości piłeczek które ma Franek.
  W drugim wierszu wejścia są trzy liczby $x$, $y$ i $z$ ($0 \leq x, y, z \leq 10^6$) oznaczające ilości piłeczek które chce mieć Franiu po wykonaniu sztuczki.

  \section{Wyjście}
    Na wyjście należy wypisać 'TAK' jeśli Franek jest w stanie po wykonaniu sztuczki uzyskać oczekiwane ilości piłeczek lub 'NIE' w przeciwnym wypadku.
    
     \makecompactexample

  \end{tasktext}
\end{document}
