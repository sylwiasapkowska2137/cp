\documentclass[zad,zawodnik,utf8]{sinol}

\title{Wąsacz}
\id{was}
\author{Karol Waszczuk} % Autor zadania
\pagestyle{fancy}
\iomode{stdin}
\konkurs{XIII obóz informatyczny}
\etap{zaawansowana}
\day{4}
\date{29.09.2016}
\RAM{32}
 
\begin{document}
\begin{tasktext}%

W ostatnim czasie Maciej uznał, że jego wygląd nie jest zbyt męski i aby to zmienić postanowił zahodować pieknego wąsa, nie jest jednak pewien czy posiadanie go przypadnie mu do gustu, więc zdecydował się hodować go przez $k$ dni, po których podejmie decyzję co z nim dalej robić.

Maciek jest też niesamowitym podrywaczem i w najbliższym czasie umówił się na dokładnie $n$ randek, z których każda odbywa się w południe innego dnia. Niestety niektóre kobiety, z którymi spotka się Maciek, nie przepadają za zbyt bujnym wąsem i jeśli na spotkaniu wąs naszego bohetara będzie zbyt długi, momentalnie dostanie on kosza. 

Po dokładnych analizach swojego zarostu Maciej odkrył, że każdej nocy jego wąs rośnie dokładnie $c$ jednostek. Może jednak temu zapobiec i wieczorem skrócić go o $c$ jednostek, jeśli waś jest niezerowej długości. Podrażni to wąsa i sprawi, że tej nocy nie urośnie ani o milimetr. 

Maciej posiada nienaganą reputację podrywacza, więc nie może sobie pozwolić na odrzucenie na jakimkolwiek spotkaniu. Młodzieniec ciekaw jest jaką maksymalną długość może osiągnąć jego wąs w w ciągu $k$ dni, przy założeniu że żadna jego randka nie może zakończyć się dostaniem kosza. 

  \section{Wejście}
W pierwszym wierszu wejścia znajdują się trzy liczby całkowite $n$, $k$ oraz $c$ ($1 \leq n \leq 10^6$, $1 \leq k \leq 10^9$, $1 \leq c \leq 10^9$) z treści zadania.

W kolejnych $n$ wierszach znajdują się dwie liczby całkowite $t_i$ oraz $w_i$ ($1 \leq t_i \leq k$, $0 \leq w_i \leq 10^9$), oznaczające kolejno dzień w którym odbędzie się $i$-ta randka oraz maksymalną długość wąsą jaką Maciek może na niej posiadać.

  \section{Wyjście}
Na standardowe wyjście należy wypisać jedną liczbę całkowitą, będącą maksymalną długością wąsa, jakiego może zahodować Maciek równocześnie spełniając oczekiwania na wszystkich umówionych randkach lub \texttt{NIE}, gdy to niemożliwe.
\makecompactexample 

\end{tasktext}
\end{document}