\documentclass[zad,zawodnik,utf8]{sinol}

\title{Osiedle}
\id{osi}
\author{Mateusz Puczel} % Autor zadania
\pagestyle{fancy}
\iomode{stdin}
\konkurs{XIII obóz informatyczny}
\etap{zaawansowana}
\day{3}
\date{28.09.2016}
\RAM{128}
 
\begin{document}
\begin{tasktext}%
Przemek jest nieoficjalnie nazywany królem swojego osiedla. Każdego wieczoru wybiera się na spacer po nim, podczas którego pilnuje porządku
oraz rozwiązuje problemy innych ludzi. Dziś Przemek też wybierze się na spacer.

Osiedle Przemka składa się z $n$ bloków mieszkalnych. Patrząc na osiedle Przemka z lotu ptaka, każdy z tych bloków ma kształt prostokąta o
bokach równoległych do osi układu współrzędnych. Bloki nie przecinają się, ale mogą stykać się brzegami. Żaden blok nie zawiera się w innym.

Spacer rozpoczyna się w pewnym punkcie, a kończy w innym. Przemek chciałby, aby przebyta droga była jak najkrótsza.
Stojąc w pewnym punkcie, Przemek może przejść w linii prostej do dowolnego innego punktu, o ile na tym odcinku nie ma żadnego bloku, tj.
odcinek między dwoma punktami nie może przecinać się z żadnym blokiem. Może natomiast dotykać zewnętrznych krawędzi bloku. Odległość
między dwoma punktami jest mierzona w metryce euklidesowej, tj. odległość między punktami $A$ i $B$ wyrażona jest wzorem $\sqrt{(x_A - x_B)^2 + (y_A - y_B)^2}$.
Podsumowując, spacer Przemka jest ciągiem przejść w linii prostej między pewnymi punktami, a łączna długość spaceru jest sumą odległości między
kolejnymi punktami.

Pomoż Przemkowi wyznaczyć minimalną długość spaceru.

  \section{Wejście}
W pierwszym wierszu wejścia znajduje się 5 liczb całkowitych $n$, $x_p$, $y_p$, $x_k$, $y_k$ ($1 \leq n \leq 700, -10^6 \leq x_p, y_p, x_k, y_k \leq 10^6$),
oznaczające odpowiednio liczbę bloków mieszkalnych oraz współrzędne punktu początkowego oraz punktu końcowego. Punkty początkowe i końcowe nie leżą wewnątrz ani na brzegu
żadnego bloku.

W każdym z kolejnych $n$ wierszy wejścia znajdują się 4 liczby całkowite $x_0, y_0, x_1, y_1$ ($-10^6 \leq x_0, y_0, x_1, y_1 \leq 10^6, x_0 < x_1, y_0 < y_1$),
oznaczające odpowiednio współrzędne lewego dolnego rogu oraz prawego górnego rogu kolejnego bloku.
  
  \section{Wyjście}
Na wyjściu powinna pojawić jedna liczba rzeczywista, będąca minimalną długością spaceru Przemka. Wynik zostanie zaliczony, jeśli nie różni się
od poprawnego o więcej niż $10^{-6}$ oraz liczba cyfr po przecinku nie przekracza 10. Jeśli nie da się dojść z punktu początkowego do końcowego, wypisz \texttt{NIE}.
\makecompactexample

\end{tasktext}
\end{document}