\documentclass[zad,zawodnik,utf8]{sinol}

\title{Oceny}
\id{oce}
\author{Mateusz Puczel} % Autor zadania
\pagestyle{fancy}
\iomode{stdin}
\konkurs{XIII obóz informatyczny}
\etap{zaawansowana}
\day{2}
\date{27.09.2016}
\RAM{64}
 
\begin{document}
\begin{tasktext}%
Przemek chodzi do szkoły, a jego ulubionymi przedmiotami są informatyka oraz wychowanie fizyczne. Pewnego razu Przemek zauważył, że dziennik klasowy
jest niepilnowany, więc postanowił zajrzeć do środka i poprawić swoje wyniki w nauce.

W dzienniku znajduje się $n$ ocen z informatyki i tyle samo z wuefu. Niedługo zbliża się wywiadówka, a Przemek wie, że jego rodzice
będą zadowoleni z jego ocen z danego przedmiotu, jeśli będą one tworzyły ciąg ściśle rosnący, tj. $a_i < a_{i + 1}$ dla każdej pary sąsiednich ocen.
Oznacza to, że nastąpił progres w jego nauce.
Postanowił więc zmienić niektóre oceny, aby zadowolić swoich rodziców. Przemek jest świadomy, że jeżeli znacznie zmieni swoje oceny, wówczas zauważą
to jego nauczyciele, a wtedy dostanie zakaz rozwiązywania zadań programistycznych. Jedyną operacją, która nie wzbudzi podejrzeń nauczycieli, jest zamiana
miejscami $i$-tej oceny z informatyki z $i$-tą oceną z wuefu. Taką operację można wykonać wiele razy, lecz Przemek chce wykonać ich jak najmniej, ponieważ
lada moment zjawi się nauczyciel.

Ratuj Przemka! Wyznacz minimalną liczbę operacji, po wykonaniu których oceny z informatyki i wuefu zadowolą jego rodziców lub stwierdź, że jest to niemożliwe.

  \section{Wejście}
W pierwszym wierszu wejścia znajduje się jedna liczba całkowita $n$ ($1 \leq n \leq 500\,000$), oznaczająca liczbę ocen z informatyki i wuefu.

W kolejnym wierszu znajduje się $n$ liczb całkowitych $a_1, a_2, \dots, a_n$ ($-10^9 \leq a_i \leq 10^9$), oznaczających oceny Przemka z informatyki.

W kolejnym wierszu znajduje się $n$ liczb całkowitych $b_1, b_2, \dots, b_n$ ($-10^9 \leq b_i \leq 10^9$), oznaczających oceny Przemka z wuefu.

  \section{Wyjście}
Na wyjściu powinna znaleźć się jedna liczba całkowita, oznaczająca minimalną liczbę zamian potrzebną do zadowolenia rodziców Przemka po wywiadówce
lub słowo \texttt{NIE}, jeśli Przemek nie jest w stanie ustawić obu ciągów w sposób ściśle rosnący za pomocą dozwolonych operacji.

\makecompactexample0

  \section{Wyjaśnienie do przykładu}
Przemek może zamienić miejscami oceny na pozycjach 2 oraz 4 uzyskując ciągi: 5, 6, 7, 9, 10 oraz 1, 3, 6, 7, 9. 

\end{tasktext}
\end{document}