\documentclass[zad,zawodnik,utf8]{sinol}

\title{Dominacja}
\id{dom}
\author{Jacek Tomasiewicz} % Autor zadania
\pagestyle{fancy}
\iomode{stdin}
\konkurs{XIII obóz informatyczny}
\etap{początkująca}
\day{1}
\date{26.09.2016}
\RAM{16}
 
\begin{document}
\begin{tasktext}%
Mamy danych $n$ liczb całkowitych. 
Przy czym jedna wartość jest \textit{dominująca} i występuje  dokładnie $n-1$ razy, a inna występuje tylko jeden raz. 

Chcielibyśmy znaleźć wartość elementu dominującego.
  \section{Wejście}
Pierwszy wiersz wejścia zawiera jedną liczbę całkowitą ($3 \leq n \leq 15\,000\,000$), oznaczającą ilość liczb.

Drugi wiersz wejścia zawiera $n$ liczb całkowitych $l_1, l_2, \ldots, l_n$ ($0 \leq l_i \leq 9$), gdzie $l_i$ oznacza wartość $i$-tej liczby.
  \section{Wyjście}
\makecompactexample

\end{tasktext}
\end{document}