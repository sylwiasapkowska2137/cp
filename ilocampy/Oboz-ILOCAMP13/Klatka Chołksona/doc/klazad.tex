\documentclass[zad,zawodnik,utf8]{sinol}

\title{Klatki Chołksona}
\id{kla}
\author{Mateusz Chołołowicz} % Autor zadania
\pagestyle{fancy}
\iomode{stdin}
\konkurs{XIII obóz informatyczny}
\etap{zaawansowana}
\day{1}
\date{26.09.2016}
\RAM{64}
 
\begin{document}
  \begin{tasktext}% Ten znak % jest istotny!

Szalony Chołkson zamierza odwiedzić $m$ miast i w każdym z nich chce pozamykać humanistów w swoich klatkach. Humaniści w $i$-tym mieście ponumerowani
są liczbami naturalnymi od $1$ do $n_i$. 
Chołkson cierpi na po-analizowe zboczenie $k$-tego stopnia i boi się zamknąć w jednym mieście dwóch takich humanistów, że numer jednego z nich jest $k$ razy większy od numeru drugiego. 
Zależnie od strefy klimatycznej w której znajduje się miasto, zboczenie Chołksona nasila się lub słabnie.

Pomóż Chołksonowi i oblicz dla każdego miasta maksymalną liczbę humanistów, których może zamknąć w swoich klatkach.

 \section{Wejście}
    
W pierwszym wierszu wejścia znajduje się jedna liczba całkowita $m$ ($1 \leq m \leq 10^5$), oznaczająca liczbę miast, w których dojdzie do hekatomby.

W każdym z kolejnych $m$ wierszy, znajdują się dwie liczby całkowite $n_i$ i $k_i$ oddzielone spacją, oznaczające liczbę humanistów w $i$-tym mieście oraz stopień nasilenia
zboczenia Chołksona w tym mieście ($1~\leq~n_i,~k_i~\leq~10^{18}$).

  \section{Wyjście}
Na standardowe wyjście należy wypisać $m$ wierszy. $I$-ty wiersz powinien składać się z jednej liczby całkowitej, równej maksymalnej liczbie humanistów, których
może zmknąć Chołkson w $i$-tym mieście.
    
\makecompactexample

  \end{tasktext}
\end{document}
