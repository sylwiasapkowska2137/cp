\documentclass[zad,zawodnik,utf8]{sinol}

\title{Ciąg Jacusia}
\id{cia}
\author{Mateusz Chołołowicz} % Autor zadania
\pagestyle{fancy}
\iomode{stdin}
\konkurs{XIII obóz informatyczny}
\etap{początkująca}
\day{4}
\date{29.09.2016}
\RAM{64}
 
\begin{document}
  \begin{tasktext}% Ten znak % jest istotny!

Jacuś miał zapisany na kartce \textbf{ściśle} rosnący ciąg $n$ liczb całkowitych dodatnich, tj. $a_i < a_{i + 1}$ dla każdej pary sąsiednich liczb. Jacuś niechcący wylał
trochę wody na kartkę i część liczb na wskutek zamoczenia stała się niewidoczna. Jacuś nie pamięta niestety jakie
liczby znajdowały się w brakujących miejscach przed rozlaniem wody. 

Tak naprawdę nie jest to istotne -- Jacuś chciałby mieć po prostu znowu ściśle rosnący ciąg o długości $n$. Pomóż mu
i podaj przykładowy zestaw liczb, które mogły znajdować się w miejscach zalanych przez wodę.

 \section{Wejście}
    W pierwszym wierszu wejścia znajduje się jedna liczba całkowita $n$ ($1 \leq n \leq 500~000$).
    W kolejnym wierszu znajduje się $n$ liczb całkowitych $a_1, a_2, \dots, a_n$ ($1 \leq a_i \leq 10^9$), oznaczających kolejne liczby ciągu
    Jacusia. Jeśli $i$-ta liczba ciągu została zalana wodą, wówczas $a_i = -1$.
    
 
  \section{Wyjście}
    W pierwszym i jedynym wierszu wyjścia należy wypisać $n$ liczb całkowitych, które mogły tworzyć oryginalny ciąg Jacusia
    lub jedna liczba $-1$, jeśli Jacusiowi coś się pomyliło i jego ciąg zapisany na kartce wcale nie był ściśle rosnący.
    Zaproponowane liczby muszą być dodatnimi liczbami całkowitymi, a ich wartości nie mogą przekraczać $10^9$.
    
    \makecompactexample

  \end{tasktext}
\end{document}
