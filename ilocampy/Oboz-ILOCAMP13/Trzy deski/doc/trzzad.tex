\documentclass[zad,zawodnik,utf8]{sinol}

\title{Trzy deski}
\id{trz}
\author{Pewnie PJK} % Autor zadania
\pagestyle{fancy}
\iomode{stdin}
\konkurs{XIII obóz informatyczny}
\etap{olimpijska}
\day{2}
\date{27.09.2016}
\RAM{64}
 
\begin{document}
\begin{tasktext}%
Jaco jest właścicielem wielkiej posiadłości w Serwach. Niestety niedawno jego okolice odwiedziła burza z
piorunami i podziurawiła cały dach (dziury w dachu są przedstawione jako punkty na osi współrzędnych).
Jaco sam zabrał się za naprawę dachu.
Jaco postanowił przykryć wszystkie dziury dokładnie trzema identycznymi kwadratowymi deskami. Deski
mogą się pokrywać i powinny być położone równoległe do osi układu współrzędnych. Powiedz Jacowi jaka jest
najmniejsza długość boków desek, tak aby było to możliwe.
Zakładamy, że dach Jaca ma bardzo dużą powierzchnię, tak dużą, że przyjmujemy, że jest nieskończony.
  \section{Wejście}
W pierwszym wierszu wejścia, znajduje się jedna liczba całkowita $n$ ($1 \leq n \leq 2 \cdot 10^5$), oznaczająca liczbę dziur
w dachu Jaca. Kolejne $n$ wierszy zawiera współrzędne kolejnych dziur $x_i$, $y_i$ ($-10^9 \leq x_i, y_i \leq 10^9$).

  \section{Wyjście}
W jedynym wierszu wyjścia wypisz najmniejszą długość boku deski, tak aby można było przykryć wszystkie
dziury trzema takimi deskami.

\makecompactexample

\end{tasktext}
\end{document}