\documentclass[zad,zawodnik,utf8]{sinol}

\title{Palindromy na okręgu}
\id{pal}
\author{Przemysław Jakub Kozłowski} % Autor zadania
\pagestyle{fancy}
\iomode{stdin}
\konkurs{XIII obóz informatyczny}
\etap{olimpijska}
\day{0}
\date{25.09.2016}
\RAM{32}
 
\begin{document}
\begin{tasktext}%
Palindrom to symetryczny napis, tzn. taki napis, który czytany od lewej do prawej i od prawej do lewej
jest taki sam.
Palindrom na okręgu to taki napis umieszczony na okręgu, który czytany od pewnej pozycji jest palindromem (jego rotacja jest palindromem). 
Np. napis \texttt{aacc} jest palindromem na okręgu, ponieważ czytany od pozycji drugiej jest palindromem: \texttt{acca}. 
Przemek napisał na okręgu pewien napis i teraz zastanawia się, ile minimalnie liter musi do niego wstawić, aby był palindromem na okręgu.

  \section{Wejście}
W pierwszej linii standardowego wejścia znajduje się jedna liczba naturalna $n$ ($1 \leq n \leq 2000$), oznaczająca długość napisu napisanego przez Przemka.
W drugiej linii standardowego wejścia znajduje się ten napis czytany od pewnej pozycji. Napis składa się z małych liter alfabetu angielskiego.

  \section{Wyjście}
Jedna liczba całkowita oznaczająca minimalną liczbę liter, które trzeba wstawić, aby napis wymyślony przez Przemka był palindromem na okręgu.

\makecompactexample

\end{tasktext}
\end{document}