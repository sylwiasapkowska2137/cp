\documentclass[zad,zawodnik,utf8]{sinol}

\title{Bilety}
\id{bil}
\author{Jacek Tomasiewicz} % Autor zadania
\pagestyle{fancy}
\iomode{stdin}
\konkurs{XIII obóz informatyczny}
\etap{zaawansowana}
\day{0}
\date{25.09.2016}
\RAM{64}
 
\begin{document}
  \begin{tasktext}% Ten znak % jest istotny!
Pan Jan wyruszył autem w trasę składającą się z $n$ nastêpujących po sobie odcinków. Przed wjazdem na każdy z odcinków znajduje się kasa biletowa, w której trzeba skasować
1 bilet, aby można było udać się w dalszą podróż. Dodatkowo w każdej kasie można kupić pewną liczbę biletów, które można kasować przed wjazdem na dowolny odcinek. 
Ceny biletów w kasach mogą się różnić, a pojedyncza osoba może kupić ich ograniczoną liczbę. 

Pan Jan chce zaplanować kupno biletów w taki sposób, aby koszt przejazdu wyszedł jak najmniejszy. Pomóż mu, i powiedz ile będzie musiał minimalnie wydać na bilety. 

 \section{Wejście}
    
Pierwszy wiersz wejścia zawiera jedną liczbę całkowitą $n$ ($1 \leq n \leq 1\,000\,000$), oznaczającą liczbę odcinków, z których składa się trasa. Kolejnych $n$ 
wierszy zawiera opis kolejnych kas. Każdy z wierszy składa się z dwóch liczb całkowitych $c_i, x_i$ ($1 \leq c_i, x_i \leq 10^6$), oznaczających odpowiednio cenę i liczbę
dostępnych biletów w $i$-tej kasie. 

  \section{Wyjście}
    Pierwszy i jedyny wiersz wyjścia powinien zawierać jedną liczbę całkowitą, równą minimalnemu kosztowi przejechania całej trasy przez Pana Jana. 
    
    \makecompactexample

  \end{tasktext}
\end{document}
