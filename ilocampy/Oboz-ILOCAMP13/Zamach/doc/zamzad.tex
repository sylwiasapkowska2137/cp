\documentclass[zad,zawodnik,utf8]{sinol}

\title{Zamach}
\id{zam}
\author{Mateusz Chołołowicz} % Autor zadania
\pagestyle{fancy}
\iomode{stdin}
\konkurs{XIII obóz informatyczny}
\etap{olimpijska}
\day{2}
\date{27.09.2016}
\RAM{64}
 
\begin{document}
  \begin{tasktext}% Ten znak % jest istotny!
Franekstein planuje zamach na Block City. Miasto składa się z $n$ bloków ustawionych bezpośrednio jeden za drugim. Każdy z bloków ma pewną liczbę pięter.
Piętra o tym samym numerze w dwóch sąsiadujących ze sobą blokach przylegają do siebie. Podłożenie bomby na $i$-tym piętrze w pewnym bloku sprawia, 
że siła zniszczenia rujnuje wszystkie piętra w tym bloku, a dodatkowo przemieszcza się do sąsiednich bloków, demolując pomieszczenia znajdujące się 
w nich na $i$-tym piętrze. Następnie fala uderzeniowa ponownie rozchodzi się w obie strony masakrując $i$-te piętra w kolejnych blokach. Fala zatrzymuje się w momencie, gdy
sąsiadujący blok ma mniej niż $i$ pięter, ponieważ gruz i pył ulatują w powietrze, dzięki czemu dalej położone bloki zostają nietknięte tym wybuchem. Bloki w Block City to nie byle jakie bloki. Posiadają niesamowicie grube i solidne ściany.
Naruszenie pojedynczym wybuchem dowolnego piętra, zabija wszystkich ludzi, którzy się na nim znajdują, jednak nie zakłóca konstrukcji bloku. Franekstein postanowił
to wykorzystać i nie zamierza tak podkładać bomb, aby któraś ściana, podłoga bądź sufit oberwały dwukrotnie. Budynek zrównałby się z ziemią, a Franekstein
nie mógłby go wówczas obłupić. Siła rażenia przemieszczając się w pionie w obrębie jednego budynku, narusza jedynie poziome płyty konstrukcji (podłogi i sufity),
natomiast przechodząc z jednego bloku do drugiego, niszczy wyłącznie ściany pomiędzy nimi.

Franekstein dysponuje $k$ bombami z opóźnionym zapłonem. Zupełnie przypadkowo, włączył zapłon we wszystkich z nich dzisiaj o godzinie 9:00. Franekstein
lubi kombinatorykę i pomimo, że za 5 godzin, równo o 14:00, bomby wybuchną, postanowił, że policzy na ile różnych sposobów może rozmieścić wszystkie bomby 
w pewnych pomieszczeniach bloków Block City, tak aby nie zaburzyć konstrukcji budynków. Oczywiście zadanie to przerosło zamachowca, a że czasu jest mało,
a Ty siedzisz przed komputerem, poprosił Cię o pomoc.

 \section{Wejście}
    
W pierwszym wierszu wejścia znajdują się dwie liczby całkowite $n$ i $k$ ($1 \leq k \leq n \leq 1000$), oznaczające odpowiednio liczbę bloków w mieście oraz liczbę 
bomb które posiada Franekstein. W kolejnym wierszu wejścia znajduje się $n$ liczb całkowitych $h_1, h_2, \dots, h_n$, oznaczających liczbę pięter w kolejnych budynkach Block City
($0 \leq h_i \leq 10^9$). 

Możesz założyć, że w testach wartych łącznie $50\%$ punktów $h_i \leq n$.

  \section{Wyjście}
    Na standardowe wyjście należy wypisać jeden wiersz zawierający jedną liczbę całkowitą równą liczbie sposobów, na które Franekstein może rozmieścić $k$ bomb
    w opisanym na wejściu mieście, modulo $10^9 + 7$. Zakłdamy, że wszystkie bomby są nierozróżnialne.
    
    \makecompactexample

  \end{tasktext}
\end{document}
