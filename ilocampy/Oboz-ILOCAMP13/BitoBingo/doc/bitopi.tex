\documentclass[opr]{sinol}

\usepackage{graphicx}
\usepackage{amsthm}
\newcounter{def}
\newcounter{tw}
\newcounter{lem}
\newcounter{wn}
\newcounter{obs}

\usepackage{listings}
\lstset{
  basicstyle=\small,
  keywordstyle=\bfseries,
  numbers=left,
  language=pascal,
  xleftmargin=1.2em,
  frame=TBLR,
  mathescape=true,
  numberstyle=\footnotesize,
 }

\newcommand{\definicja}[1]{\vskip 0.2cm \noindent {\bf Definicja.\thedef.} \stepcounter{def} \emph{#1} \vskip 0.3cm}
\newcommand{\twierdzenie}[1]{\vskip 0.2cm \noindent {\bf Twierdzenie.\thetw.} \stepcounter{tw} \emph{#1} \vskip 0.3cm}
\newcommand{\lemat}[1]{\noindent {\vskip 0.2cm \bf Lemat.\thelem.} \stepcounter{lem} \emph{#1} \vskip 0.3cm}
\newcommand{\wniosek}[1]{\noindent {\vskip 0.2cm \bf Wniosek.\thewn.} \stepcounter{wn} \emph{#1} \vskip 0.3cm}
\newcommand{\obserwacja}[1]{\noindent {\vskip 0.2cm \bf Obserwacja.\theobs.} \stepcounter{obs} \emph{#1} \vskip 0.3cm}
\newcommand{\intuicja}[1]{\noindent {\vskip 0.2cm \bf Intuicja.} #1 \vskip 0.3cm}
\newcommand{\dowod}[1]{\begin{proof} #1 \end{proof}}
\begin{document}
 \signature{jtom???}
  \id{bit}
  \title{BitoBingo}
  \author{Jacek Tomasiewicz}
  \iomode{stdin}
  \etap{pocz�tkuj�ca}
  \day{4}
  \date{26.11.2011}
  \RAM{32}
  \history{2011.12.30}{Jacek Tomasiewicz, przygotowanie rozwi�zania}{1.00}



\begin{tasktext}%

\setcounter{def}{1}
\setcounter{tw}{1}
\setcounter{wn}{1}
\setcounter{obs}{1}

\section{Rozwi�zanie wzorcowe $O(m)$}

Przy ka�dym ustawieniu krzy�yka zwi�kszamy liczb� krzy�yk�w w odpowiednim wierszu i kolumnie. Je�li liczba krzy�k�w w kt�rym� momencie b�dzie wynosi�a $n$ w pewnym wierszu (kolumnie), to istnieje w ca�o�ci wype�niony wiersz (kolumna). Sprawdzanie to wystraczy, �e b�dziemy wykonywa� tylko w miejscach ostatnich ustawie� krzy�yk�w.

\begin{lstlisting}
wczytaj(n, m)
for i := 1 to m do
   wczytaj(x, y)
   kolumna[x] := kolumna[x] + 1
   wiersz[y] := wiersz[y] + 1
   if (kolumna[x] == n) or (wiersz[y] == n) then
      wypisz_zwyciezce(i)
      zakoncz_program()
wypisz(-1)
\end{lstlisting}

Z�o�ono�� czasowa takiego rozwi�zania to $O(m)$.


\end{tasktext}
\end{document}