\documentclass[zad,zawodnik,utf8]{sinol}

\title{Wycieczka}
\id{wyc}
\author{Maciej Hołubowicz} % Autor zadania
\pagestyle{fancy}
\iomode{stdin}
\konkurs{XIII obóz informatyczny}
\etap{Olimpijska}
\day{3}
\date{28.09.2016}
\RAM{128}
 
\begin{document}
  \begin{tasktext}% Ten znak % jest istotny!
  W Bajtocji trwa remont dróg, a Przemek który uwielbia podróżować nie może czekać ze swoimi wycieczkami bo skończą mu się wakacje!
  
  Jak wszyscy dobrze wiemy Bajtocja składa się z $n$ miast połączonych $m$ dwukierunkowymi drogami. W Bajtocji podczas wakacji Przemka w jednym momencie jest remontowana dokładnie jedna droga. Droga na czas remontu jest nieprzejezdna. Przemek chciałbym wybrać się na wycieczkę pomiędzy dwoma miastami. Jednakże z powodu remonu może się okazać, że nie da się przejechać pomiędzy niektórymi parami miast. 
  
  Przemek chciałby zawczasu wiedzieć czy jego pomysł na wycieczke może być zrealizowany. Potrzebuje więc programu który będzie odpowiadał na jego pytania czy wycieczka pomiędzy miastami $a$ i $b$ jest możliwa, gdy remontowana jest krawędz pomiędzy miastami $x$ i $y$! Sytuacja (jak i zachcianki Przemka) zmieniają się bardzo dynamicznie więc program może dostać wiele zapytań.
  
  Przemek, jako że ma wakacje i $"$$nie$ $chce$ $mu$ $się$ $kminić"$ poprosił Ciebie, dobrego kumpla którego niegdyś uczył kodzenia o pomoc! Czy możesz napisać dla niego program którego potrzebuje?
  
 \section{Wejście}
    
W pierwszym wierszu wejścia znajdują się trzy liczby całkowite $n$, $m$ i $q$ ($1 \leq n, m, q \leq 10^6$), oznaczająca kolejno liczbę miast, dróg oraz liczbę zapytań.

W każdym z kolejnych $m$ wierszy, znajdują się dwie liczby całkowite $a$ i $b$ oddzielone spacją, oznaczające że istnieje droga dwukierunkowa pomiędzy miastami $a$ i $b$, ($1 \leq a, b \leq n$, $a \neq b$).

W kolejnych $q$ wierszach będą cztery liczby całkowite $a$, $b$, $x$ i $y$ ($1 \leq a, b, x, y \leq n$, $a \neq b$) oddzielone spacją oznaczające zapytania Przemka, czy da się przejechać pomiędzy wierzchołkami $a$ i $b$ gdy droga pomiędzy wierzchołkami $x$ i $y$ jest remontowana.

%TODO: czy graf jest spójny
Remontowane drogi z zapytań musiały wcześniej być podane jako istniejące drogi w Bajtocji. Pomiędzy parą miast może istnieć co najwyżej jedna droga.

  \section{Wyjście}
    Na wyjście należy wypisać $q$ wierszy. $i$-ty wiersz powinnien być odpowiedzią na kolejne zapytanie z wejścia i powinnien zawierać TAK jeśli da się zrealizować $i$-tą wycieczke a NIE jeśli jest to niemożliwe.
    
     \makecompactexample

  \end{tasktext}
\end{document}
