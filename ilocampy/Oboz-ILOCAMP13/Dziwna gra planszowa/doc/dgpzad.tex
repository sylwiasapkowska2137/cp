\documentclass[zad,zawodnik,utf8]{sinol}

\title{Dziwna gra planszowa}
\id{dgp}
\author{Mateusz Puczel} % Autor zadania
\pagestyle{fancy}
\iomode{stdin}
\konkurs{XIII obóz informatyczny}
\etap{początkująca}
\day{3}
\date{28.09.2016}
\RAM{32}
 
\begin{document}
\begin{tasktext}%
Przemek i Jakub kupili ostatnio dziwną grę planszową. Składa się ona z planszy, która ma $n$ wierszy oraz $m$ kolumn.
Każde pole na planszy jest wolne. Dodatkowo w zestawie są dwa pionki -- jeden należy do Przemka, a drugi do Jakuba.

Początkowo oba pionki stoją na pewnych polach planszy. Celem Przemka jest dojść pionkiem do ostatniej kolumny planszy,
a celem Jakuba jest nie dopuszczenie do tego. Dodatkowo gracz przegrywa, gdy nie może wykonać poprawnego ruchu.

Gracze wykonują ruchy na przemian. W każdym ruchu
gracz \textbf{musi} przesunąć swój pionek w jednym z czterech kierunków: góra, dół, lewo, prawo. Wykonując ruch gracz nie może
wyjść poza planszę ani nie może stanąć na polu, które bezpośrednio sąsiaduje (ma wspólną krawędź) z polem, na którym stoi pionek przeciwnika.
Gra kończy się, gdy Przemek postawi swój pionek na dowolnym polu ostatniej kolumny.
Przemek wykonuje pierwszy ruch.

Zanim gra się zacznie, Przemek zastanawia się, czy może zwycięsko zakończyć grę niezależnie od ruchów wykonywanych przez Jakuba.
Ponieważ plansza jest bardzo duża, napisz program, który rozwieje jego wątpliwości.

  \section{Wejście}
W pierwszym wierszu wejścia znajduje się jedna liczba całkowita $t$ ($1 \leq t \leq 500\,000$), oznaczająca liczbę zestawów danych.

W każdym z kolejnych $t$ wierszy znajduje się 6 liczb całkowitych $n$, $m$, $r_p$, $c_p$, $r_j$, $c_j$ 
($3 \leq r_p, r_j \leq n \leq 10^9$, $3 \leq c_p, c_j \leq m \leq 10^9$), oznaczających odpowiednio liczbę wierszy oraz kolumn planszy,
numer wiersza oraz kolumny pola, na którym stoi pionek Przemka, numer wiersza oraz kolumny pola,
na którym stoi pionek Jakuba.

Wiersze ponumerowane są od góry do dołu kolejnymi liczbami naturalnymi od $1$ do $n$, a kolumny ponumerowane
są od lewej do prawej kolejnymi liczbami naturalnymi od $1$ do $m$.

Możesz założyć, że oba pionki nie stoją na tych samych polach ani na polach bezpośrednio sąsiadujacych.
  \section{Wyjście}
Na wyjściu powinno znaleźć się $t$ wierszy. W każdym z nich powinna pojawić się odpowiedź do odpowiedniego zestawu danych - 
słowo \texttt{TAK}, jeśli Przemek jest w stanie wygrać niezależnie od ruchów Jakuba lub \texttt{NIE} w przeciwnym wypadku.

\makecompactexample

  \section{Wyjaśnienie do przykładu}
W pierwszym zestawie danych Przemek może poruszać się cały czas w prawo, a Jakub nie jest w stanie mu przeszkodzić, ponieważ musiałby
stanąć na polu sąsiadującym z pionkiem Przemka.
\\W drugim zestawie danych Jakub może naśladować ruchy Przemka. Dzięki temu Przemek nigdy nie dojdzie do ostatniej kolumny.

\end{tasktext}
\end{document}