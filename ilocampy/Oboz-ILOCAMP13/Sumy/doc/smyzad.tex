\documentclass[zad,zawodnik,utf8]{sinol}

\title{Sumy}
\id{smy}
\author{Jacek Tomasiewicz} % Autor zadania
\pagestyle{fancy}
\iomode{stdin}
\konkurs{XIII obóz informatyczny}
\etap{początkująca}
\day{1}
\date{26.09.2016}
\RAM{32}
 
\begin{document}
\begin{tasktext}%
Mamy dany ciąg $n$ liczb całkowitych $l_1, l_2, \ldots, l_n$. Chcielibyśmy wyliczyć dla każdego elementu, sumę wszystkich innych liczb.
Dokładniej, dla $i$-tego elementu, chcielibyśmy znać sumę $l_1 + l_2 + \ldots l_{i-1} + l_{i+1} + \ldots + l_n$.

  \section{Wejście}
Pierwszy wiersz wejścia zawiera jedną liczbę całkowitą $n$ ($1 \leq n \leq 400\,000$), oznaczającą ilość liczb.
Kolejny wiersz zawiera $n$ liczb całkowitych $l_1, l_2, \ldots, l_n$ ($1 \leq l_i \leq 10^3$), gdzie $l_i$ oznacza wartość $i$-tej liczby.

  \section{Wyjście}
Wyjście powinno zawierać $n$ liczb całkowitych będących wyliczonymi sumami dla każdego z elementów.

\makecompactexample

\end{tasktext}
\end{document}