\documentclass[zad,zawodnik,utf8]{sinol}
  \title{Antena}
  \author{Jacek Tomasiewicz}
  \pagestyle{fancy}
  \signature{jtom???}
  \id{ant}
  \iomode{stdin}
  \konkurs{XIII obóz informatyczny}
  \etap{początkująca}
  \day{0}
  \date{25.09.2016}
  \RAM{32}
  \history{2011.02.20}{JTom, pomysł i~redakcja}{1.00}

\begin{document}
  \begin{tasktext}% Ten znak % jest istotny!
W maleńkiej wiosce Bajtowszczyźnie, znajdują się 3 domki jednorodzinne. Sołtys zastanawia się, w którym miejscu ustawić antenę, tak aby wszystkie domki były w jej zasięgu. Chce przy tym, aby antena miała jak najmniejszy zasięg. Pomóż mu i powiedz, jaki zasięg powinna mieć antena.

Zakładamy, że zasięg anteny to jej promień, którego długość może być tylko całkowita. Domki leżą w linii prostej, a ich położenie utożsamiamy z punktami na osi. Antenę można ustawić w dowolnym punkcie o \textbf{nieujemnych} współrzędnych, gdyż tylko w tych miejscach sołtys ma pozwolenie na budowę.

  \section{Wejście}
Pierwszy wiersz wejścia zawiera trzy liczby całkowite $x_1, x_2, x_3$ ($-10^9 \leq x_i \leq 10^9$), gdzie $x_i$ oznacza współrzędną $i$-tego domku. Zakładamy, że jest to współrzędna $x$-owa, a domki leżą na osi \textit{OX}, czyli wszystkie posiadają współrzędną $y = 0$.

  \section{Wyjście}
Wyjście powinno zawierać jedną liczbę całkowitą, równą minimalnemu zasięgowi anteny.

     \makecompactexample

  \medskip
  \noindent
  \textbf{Wyjaśnienie do przykładu:} 
Domki mają współrzędne $(2, 0), (6, 0), (10, 0)$. Antenę ustawiamy w punkcie $(6,0)$, czyli tam gdzie drugi domek jednorodzinny. Jej zasięg ustawiamy na wartość 4.

  \end{tasktext}
\end{document}