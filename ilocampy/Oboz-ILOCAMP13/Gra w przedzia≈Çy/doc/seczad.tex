\documentclass[zad,zawodnik,utf8]{sinol}

\title{Gra w przedziały}
\id{sec}
\author{Mateusz Puczel} % Autor zadania
\pagestyle{fancy}
\iomode{stdin}
\konkurs{XIII obóz informatyczny}
\etap{olimpijska}
\day{4}
\date{29.09.2016}
\RAM{256}
 
\begin{document}
\begin{tasktext}%
Podczas zawodów programistycznych na Ilocampie, Przemek i Jakub po ciężkiej nocy przygotowywania zadań, postanowili odprężyć się odrobinę
i zagrać w nową, wymyśloną przez nich grę.

Na kartce znajduje się $n$ liczb całkowitych, wypisanych w rzędzie. Gra składa się z dwóch ruchów -- ruchu Przemka i ruchu Jakuba.
Najpierw Przemek wycina pewien spójny fragment z ciągu liczb, następnie sklejane są pozostałe części, po czym Jakub również wycina pewien
spójny fragment nowootrzymanego ciągu. Po wykonaniu obu ruchów sumujemy wartości z dwóch wyciętych fragmentów. Przemek chce, aby uzyskana suma była
jak największa, a Jakub -- jak najmniejsza.

Zakładając optymalną grę obu graczy, wyznacz wynik gry, czyli uzyskaną sumę.
  \section{Wejście}
W pierwszym wierszu wejścia znajduje się jedna liczba całkowita $n$ ($1 \leq n \leq 10^6$), oznaczająca liczbę liczb napisanych na kartce.

W drugim wierszu wejścia znajduje się $n$ liczb całkowitych $a_1, a_2, \dots, a_n$ ($-10^9 \leq a_i \leq 10^9$), oznaczających wartości
kolejnych liczb na kartce.
  \section{Wyjście}
Na wyjściu powinna znaleźć się jedna liczba całkowita, oznaczająca wynik gry przy założeniu optymalnej gry Przemka i Jakuba.
\makecompactexample

  \section{Wyjaśnienie do przykładu}
Przemek może wyciąć fragment od pozycji 7 do 9 włącznie, czyli fragment \texttt{12 -5 8}.
Następnie Jakub wytnie fragment od pozycji 6 do 7 włącznie z nowopowstałego ciągu, czyli fragment \texttt{-2 -7}.

\end{tasktext}
\end{document}