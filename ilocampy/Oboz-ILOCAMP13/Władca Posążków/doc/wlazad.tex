\documentclass[zad,zawodnik,utf8]{sinol}

\title{Władca Posążków}
\id{wla}
\author{Mateusz Chołołowicz} % Autor zadania
\pagestyle{fancy}
\iomode{stdin}
\konkurs{XIII obóz informatyczny}
\etap{olimpijska}
\day{1}
\date{26.09.2016}
\RAM{64}
 
\begin{document}
  \begin{tasktext}% Ten znak % jest istotny!
Pradawna indiańska legenda głosi, iż każdy z dziewięciu bogów, by uwiecznić swą moc, wykuł posążek ze szlachetnych kamieni, zamykając w nim cząstkę samego siebie.
W posążkach została zaklęta wola rządzenia każdą z ras. Wszyscy bogowie zostali jednak oszukani. Najmroczniejszy z nich - Ktulu - ukuł Jedyny Posążek, rządzący pozostałymi.
Przelał w ów posążek całe swe okrucieństwo, złość i żądzę władania wszelkim życiem. Jeden, by wszystkimi rządzić, jeden, by wszystkie odnaleźć, jeden, by wszystkie 
zgromadzić i w ciemności związać. Jeden, by posiąść Wszechbinarność. 

Indianin Ilek, zwany wśród swego plemienia Binarnym Rozpruwaczem\footnote{Przydomek Binarny Rozpruwacz Ilek otrzymał ze względu na swoją ponadprzeciętną umiejętność 
rozpruwania liczb na postać binarną.},
nieoczekiwanie wszedł w posiadanie Jedynego Posążka.
Warto wspomnieć, że posążki indiańskich bogów, podczas wykuwania, były zabezpieczane zaklęciem ochronnym. Posiadacz posążka, zanim stawał się jego panem,
musiał wypowiedzieć wyraźnie wspomniane zaklęcie (o ile szeryf nie zamknął go tej nocy w więzieniu). 
Jak łatwo się domyślić, zaklęcie do Jedynego Posążka nie jest powszechnie znane. Binarny Rozpruwacz w celu jego poznania rozpruł już wiele liczb i wypatroszył tysiące
indiańskich szamanów. Ostatecznie od pewnego pastora z pobliskiego miasteczka dowiedział się, że Ktulu pozostawił wskazówkę do zaklęcia, licząc, że pewnego dnia
jakaś bystra istota okiełzna jego wszechbinarną moc i razem spowiją mrokiem całą Ziemię. Zaklęciem okazał się zbiór $n$ liczb całkowitych, które Ktulu zapisał niegdyś
na Magicznym Okręgu. By zapanować nad posążkiem należy wyrecytować zgodnie z ruchem wskazówek zegara wszystkie liczby z okręgu, zaczynając od dowolnej z nich.
Pastor (zanim został rozpruty) wyjawił Ilkowi, że wskazówka również składa się z $n$ liczb całkowitych oraz widnieje na Magicznym Okręgu, 
przykrywając swymi liczbami liczby zaklęcia. Legenda głosi, że każda z liczb wskazówki jest sumą $k$ sąsiadujących ze sobą liczb zaklęcia, 
przy czym początek i koniec przedziału sumowanych liczb przesuwa się o jeden zgodnie z ruchem wskazówek zegara dla każdej kolejnej liczby należącej do
podpowiedzi pozostawionej przez Ktulu. 

Posiadając tę wiedzę, Binarny Rozpruwacz udał się do Magicznego Okręgu (liczbę $k$ kupił od Lucky Luke'a, który planował później przejąć posążek, by oddać go na cele charytatywne). 
Ilek jednak nie potrafi wykonywać operacji arytmetycznych na tak wielu liczbach. Ostatniej nocy, jego żona - indiańska szamanka, zwana Iloną - podłożyła Ci ziółka 
i czy chcesz czy nie, musisz pomóc Binarnemu Rozpruwaczowi w odkryciu Legendarnego Kodu władającego Jedynym Posążkiem. Wszechbinarność czeka na Ciebie!

 \section{Wejście}
    
W pierwszym wierszu wejścia znajdują się dwie liczby całkowite $n$ i $k$ ($1 \leq n, k \leq 10^6$), oznaczające liczbę liczb z których składa się zaklęcie do Jedynego Posążka
oraz długość przedziałów zaklęcia, które zostały zsumowane w celu wytworzenia wskazówki. 
W kolejnym wierszu znajduje się $n$ liczb całkowitych, oznaczających liczby widniejące na Magicznym Okręgu, będące wskazówką do zaklęcia. Liczby podane są zgodnie z ruchem
wskazówek zegara i należą do przedziału $\big \langle-10^9, 10^9\big \rangle$, podobnie jak liczby szukanego zaklęcia.

  \section{Wyjście}
    Na standardowe wyjście należy wypisać jeden wiersz zawierający $n$ liczb całkowitych, będących kolejnymi liczbami zaklęcia. Należy przyjąć, że pierwsza wczytana z wejścia
liczba wskazówki powstała poprzez zsumowanie pierwszych $k$ wypisanych na wyjście liczb. Istnieje możliwość, że wskazówka jest niejednoznaczna, to znaczy istnieje wiele
zaklęć, które mogły zostać zaszyfrowane przy jej pomocy. Należy wówczas wypisać dowolne z nich i liczyć na szczęście, że to właśnie te liczby tworzą poszukiwane zaklęcie. 
Możesz założyć, że istnieje przynajmniej jedno zaklęcie składające się z liczb całkowitych z przedziału $\big \langle-10^9, 10^9\big \rangle$, które po zakodowaniu zgodnie
z instrukcjami zasłyszanymi u pastora, utworzy wskazówkę wczytaną z wejścia. 

    \makecompactexample

  \end{tasktext}
\end{document}
