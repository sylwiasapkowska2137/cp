\documentclass[zad,zawodnik,utf8]{sinol}

\title{Smakołyki}
\id{sma}
\author{Jacek Tomasiewicz} % Autor zadania
\pagestyle{fancy}
\iomode{stdin}
\konkurs{XIII obóz informatyczny}
\etap{zaawansowana}
\day{0}
\date{25.09.2016}
\RAM{32}
 
\begin{document}
\begin{tasktext}%
Natalia ustawiła w rzędzie $n$ smakołyków. Każdy smakołyk ma przypisany 
pewien rodzaj. Natalia może teraz wybrać pewną liczbę (od 1 do $n$) sąsiednich  
smakołyków, a następnie je wszystkie zjeść. Jedynym warunkiem jest to, aby żadne dwa 
smakołyki nie były tego samego rodzaju.

Pomóż Natalii i znajdź liczbę sposobów, na które można wybrać sąsiednie smakołyki. 

  \section{Wejście}
Pierwszy wiersz wejścia zawiera dwie liczby całkowite $n, m$ ($1 \leq n, m \leq 300\,000$), 
oznaczające odpowiednio liczbę smakołyków oraz liczbę dostępnych ich rodzajów. Drugi wiersz wejścia zawiera $n$ liczb całkowitych 
$c_1, c_2, \ldots, c_n$ ($1 \leq c_i \leq m$), gdzie $c_i$ oznacza rodzaj 
$i$-tego smakołyka.

  \section{Wyjście}
Pierwszy i jedyny wiersz wyjścia powinien zawierać jedną liczbę całkowitą, 
równą liczbie sposobów, na które Natalia może wybrać sąsiednie smakołyki.

\makecompactexample

\end{tasktext}
\end{document}