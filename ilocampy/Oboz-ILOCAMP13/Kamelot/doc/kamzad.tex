\documentclass[zad,zawodnik,utf8]{sinol}

\title{Kamelot}
\id{kam}
\author{Mateusz Chołołowicz} % Autor zadania
\pagestyle{fancy}
\iomode{stdin}
\konkurs{XIII obóz informatyczny}
\etap{olimpijska}
\day{4}
\date{29.09.2016}
\RAM{64}
 
\begin{document}
  \begin{tasktext}% Ten znak % jest istotny!

  W Państwie Kamelot znajduje się $n$ miast połączonych dwukierunowymi drogami. Z każdego miasta da się dojechać do każdego innego dokładnie na jednen sposób. 
  Innymi słowy, sieć połączeń w Kamelocie tworzy \textit{drzewo}. Każda droga łącząca dwa miasta ma określoną liczbę minut, potrzebnych by ją przejechać i wynika ona 
  z nawierzchni oraz stanu tej drogi. Król Kamelotu - Jakubysław - mieszka w stolicy państwa i zależy mu na tym, aby czas potrzebny na dojechanie ze stolicy do najbardziej 
  odległego miasta był jak najkrótszy. Jakubysław dysponuje $k$ kamefuntami. Robotnicy Kamelotu się cenią - za jeden kamefunt mogą poprawić stan dowolnej drogi tylko o tyle, 
  że czas potrzebny na jej przejechanie zmiejszy się o jedną minutę. Geolodzy Kamelotu zbadali dokładnie litosferę znajdującą się pod drogami państwa, wyznaczając w ten
  sposób minimalny czas do jakiego mogą zejść robotnicy Jakubysława poprawiając stan poszczególnych dróg. 
  
  Król zastanawia się jak optymalnie wykorzystać zasoby finansowe Kamelotu, aby liczba minut jazdy do miasta o najdłuższym czasie dojazdu była możliwie najmniejsza. 
  Przemykub nie uważał na lekcjach matematyki, dlatego jak pewnie się domyślasz - prosi Cię o pomoc. 
  
 \section{Wejście}
    
W pierwszym wierszu wejścia znajdują się dwie liczby całkowite $n$ i $k$ ($1 \leq n \leq 10^5, 1 \leq k \leq 10^9$), oznaczające odpowiednio liczbę miast w Kamelocie
oraz liczbę kamefuntów, którymi dysponuje Jakubysław. W $n-1$ kolejnych wierszach znajdują się opisy kolejnych dróg składające się z czwórek liczb 
$a_i$, $b_i$, $t_i$, $(t_{min})_i$ ($1~\leq~a_i,~b_i~\leq~n,~a_i~\neq~b_i,~1~\leq~(t_{min})_i~\leq~t_i~\leq~10^9$). Reprezentują one kolejno: numery miast, które łączy
dana droga, czas potrzebny do jej przebycia oraz minimalny czas przejadu tą drogą, jaki można uzyskać poprzez modernizację. Dla uproszczenia załóżmy, że miasto będące
stolicą ma numer $1$.


  \section{Wyjście}
    Na standardowe wyjście należy wypisać jedną liczbę całkowitą, równą liczbie minut potrzebnych na dojazd ze stolicy do miasta, do którego czas dojazdu jest najdłuższy,
    przy otymalnym wykorszystaniu zasobów finansowych króla Przemykuba.
    
    
    \makecompactexample

  \end{tasktext}
\end{document}
