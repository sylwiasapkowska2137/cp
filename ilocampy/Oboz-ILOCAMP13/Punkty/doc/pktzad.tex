\documentclass[zad,zawodnik,utf8]{sinol}

\title{Punkty}
\id{pkt}
\author{Maciej Hołubowicz} % Autor zadania
\pagestyle{fancy}
\iomode{stdin}
\konkurs{XIII obóz informatyczny}
\etap{zaawansowana}
\day{2}
\date{27.09.2016}
\RAM{128}
 
\begin{document}
  \begin{tasktext}% Ten znak % jest istotny!
  Masz siatkę $n$ na $m$ z punktami o współrzędnych całkowitych od $(0, 0)$ do $(n, m)$. Twoim zadaniem jest wybranie czterech różnych punktów z siatki i poprowadzenie łamanej między nimi, tak aby łamana była jak najdłuższa. Łamana może mieć samoprzecięcia. 
  
 \section{Wejście}
    
Na wejściu znajdują się dwie liczby $n, m$ ($0 \leq n, m \leq 1000$) oznaczające wymiary siatki. Jest zagwarantowane że każda siatka będzie miała przynajmniej 4 punkty.

  \section{Wyjście}
    Na wyjściu powinny się znaleźć współrzędne 4 punktów oddzielone znakiem nowej linii będące kolejnymi punktami łamanej. Punkty powinny być w postaci $x~y$ gdzie $0 \leq x \leq n$ i $0 \leq y \leq m$. Punkty powinny być parami różne.
    
     \makecompactexample

  \end{tasktext}
\end{document}
