\documentclass[zad,zawodnik,utf8]{sinol}

\title{Zgred Fred}
\id{zgr}
\author{Mateusz Chołołowicz} % Autor zadania
\pagestyle{fancy}
\iomode{stdin}
\konkurs{XIII obóz informatyczny}
\etap{olimpijska}
\day{3}
\date{28.09.2016}
\RAM{64}
 
\begin{document}
  \begin{tasktext}% Ten znak % jest istotny!

  
W Michałczewie organizowany był wczoraj maraton. Biegło w nim $n$ zawodników. Niestety maszyna odpowiedzialna za mierzenie czasów na linii mety
zawiodła i nieznane są dokładne wyniki biegu. Dla każdego biegacza maszyna podała przedział czasowy w jakim zawodnik pojawił się na mecie.
Organizatorzy sprawdzili zachowanie urządzenia i stwierdzili, że dla każdych dwóch podprzedziałów równej długości, prawdopodobieństwo
tego, że faktyczny czas zawodnika się w nich zawierał, jest równe. Innymi słowy, dla każdego momentu z przedziału podanego
przez urządzenie, prawdopodobieństwo, że to właśnie w tym momencie zawodnik zakończył maraton, jest take same.

Twój przyjaciel Zgred Fred biegł w tym maratonie i chciałby wiedzieć, jakie ma szanse na zwycięstwo. Jeszcze nie doszedł do siebie po biegu,
dlatego obliczenie prawdopodobieństwa tego, że wygrał, przypadło Tobie.

 \section{Wejście}
    
W pierwszym wierszu wejścia znajduje się jedna liczba całkowita $n$ ($1 \leq n \leq 300$), oznaczająca liczbę zawodników którzy brali udział w biegu.
Zawodników numerujemy dla uporoszczenia liczbami naturalnymi od $1$ do $n$, przy czym Zgred Fred ma numer 1.
W każdym z kolejnych $n$ wierszy, znajdują się dwie liczby całkowite $a_i$ i $b_i$ ($0 \leq a_i \leq b_i \leq 10^9$), 
oddzielone spacją, oznaczające początek i koniec przedziału podanego przez maszynę dla $i$-tego zawodnika.

  \section{Wyjście}
    Na standardowe wyjście należy wypisać jedną liczbę rzeczywistą równą prawdopodobieństwu tego, że Zgred Fred wygrał maraton. Wynik 
    zostanie zaliczony, jeśli nie będzie się różnić od optymalnego o więcej niż $10^{-5}$ oraz liczba cyfr po przecinku nie przekroczy $10$.
    
    \makecompactexample

  \end{tasktext}
\end{document}
