\documentclass[zad,zawodnik,utf8]{sinol}

\title{Inwersje}
\id{inw}
\author{Nie wiem} % Autor zadania
\pagestyle{fancy}
\iomode{stdin}
\konkurs{XIII obóz informatyczny}
\etap{olimpijska}
\day{2}
\date{27.09.2016}
\RAM{64}
 
\begin{document}
\begin{tasktext}%
Adi lubi ciągi liczb. Nienawidzi zaś w nich inwersji. Inwersja w ciągu liczb $a_1, a_2, \dots, a_n$ jest parą pozycji $i$, $j$ ($1 \leq i < j \leq n$) takich,
że zachodzi $a_i > a_j$.

Dzisiaj są urodziny naszego bohatera i dostał w prezencie od przyjaciół ciąg liczb $p_1, p_2, \dots, p_n$. Adi może pomnożyć wybrane
przez siebie elementy ciągu $p$ przez $-1$. Po wykonaniu tych operacji mnożenia chce uzyskać minimalną możliwą liczbę inwersji w danym ciągu.

  \section{Wejście}
Pierwsza linia wejścia zawiera liczbę całkowitą dodatnią $z$, oznaczającą liczbę zestawów danych. Każdy zestaw danych składa się z dwóch linii,
w pierwszej znajduje się jedna liczba całkowita $n$ ($1 \leq n \leq 100\,000$). Następna linia zestawu danych zawiera $n$ liczb całkowitych -- ciąg,
który chce zmodyfikować Adi w postaci $p_1, p_2, \dots, p_n$ ($|p_i| \leq 100\,000$). Liczby są oddzielone spacjami. Możesz przyjąć, że w $80\%$ testów
$1 \leq n \leq 2000$.
  \section{Wyjście}
Jedna liczba całkowita, równa minimalnej liczbę inwersji w ciągu, jaką może uzyskać Adi.
\makecompactexample

\end{tasktext}
\end{document}