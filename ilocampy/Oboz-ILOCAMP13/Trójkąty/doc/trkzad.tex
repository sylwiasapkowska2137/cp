\documentclass[zad,zawodnik,utf8]{sinol}

\title{Trójkąty}
\id{trk}
\author{Mateusz ChoLOLowicz} % Autor zadania
\pagestyle{fancy}
\iomode{stdin}
\konkurs{XIII obóz informatyczny}
\etap{podstawowa}
\day{3}
\date{28.09.2016}
\RAM{32}
 
\begin{document}
\begin{tasktext}%

Jakubek z okazji swoich urodzin, otrzymał od swojego starszego brata Przemka niesamowity zestaw czterech drewnianych patyków. Chłopiec ogromnie ucieszył się z tego prezentu i już od paru dni nie może się od niego oderwać, wymyślając coraz to nowe zabawy i zastosowania otrzymanych patyków. Dzisiaj postanowił, że ułoży z patyków znajdujących się w zestawie trójkąt o niezerowym polu. Niestety Jakubek jest jeszcze bardzo mały i nie potrafi poradzić sobie z tym zadaniem, więc poprosił Ciebie o pomoc w stwierdzeniu czy stworzenie takiego trójkąta jest w ogóle możliwe.


  \section{Wejście}
W pierwszym i jedynym wierszu wejścia znajdują się 4 liczby całkowite $a$, $b$, $c$ oraz $d$ ($1 \leq a, b, c, d \leq 10000$) oznaczające długości kolejnych patyków Jakubka. 

  \section{Wyjście}
Na wyjściu należy wypisać jedną linię zawierającą słowo \texttt{TAK}, jeśli zbudowanie trójkata o niezerowym polu z patyków Jakubka jest możliwe lub \texttt{NIE} w przeciwnym przypadku.
\makecompactexample 

\end{tasktext}
\end{document}