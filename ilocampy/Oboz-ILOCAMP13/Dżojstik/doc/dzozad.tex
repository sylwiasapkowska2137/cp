\documentclass[zad,zawodnik,utf8]{sinol}

\title{Dżojstik}
\id{dzo}
\author{Maciej Hołubowicz} % Autor zadania
\pagestyle{fancy}
\iomode{stdin}
\konkurs{XIII obóz informatyczny}
\etap{początkująca}
\day{4}
\date{29.09.2016}
\RAM{64}
 
\begin{document}
  \begin{tasktext}% Ten znak % jest istotny!
  Izydor i Beata chcą zagrać na konsoli. Mają dwa dżojstiki, ale jak zwykle Beata zapomniała ładowarki i chce używać tej Izydora. Mają więc tylko jedną ładowarkę. Pierwszy dżojstik jest naładowany w $a_1$ procentach a drugi w $a_2$ procentach. W jednej minucie, dżojstik podłączony do ładowarki ładuje się o 1 procent, a dżojstik nie podłączony rozładowuje się o 2 procent.
  
  Gra trwa dopóki oba dżojstiki mają dodatni poziom naładowania. Jeśli dżojstik ma poziom naładowania 1 musi zostać podłączony do ładowarki, inaczej gra się kończy. Jeśli jakiś dżojstik zostanie rozładowany do 0, gra również się kończy.l
  
  Wyznacz maksymalną liczbę minut przez które gra może trwać. Gra nie może zostać zatrzymana. Jest dozwolone aby dżojstiki miały więcej niż 100 procent naładowania.
  
  
 \section{Wejście}
    
  Na wejściu są dwie liczby $a_1$ i $a_2$ ($ 1 \leq a_1, a_2 \leq 100$) oznaczające poziomy naładowania pierwszego i drugiego dżojstika.

  \section{Wyjście}
    Na wyjście należy wypisać maksymalną liczbę minut przez które gra może trwać.
    
     \makecompactexample

  \end{tasktext}
\end{document}
